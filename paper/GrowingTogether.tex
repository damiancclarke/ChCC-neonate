\documentclass[12pt]{article}
%===============================================================================
%=== (1) Packages
%===============================================================================
\usepackage[english]{babel}
\selectlanguage{english}
%\usepackage[utf8]{inputenc}
\usepackage{libertine}
\usepackage[libertine]{newtxmath} 

\usepackage{amsfonts}
\usepackage{amsmath}
%\usepackage{amsthm}
\usepackage{appendix}
\usepackage{bm}
\usepackage{booktabs}
\usepackage[usenames, dvipsnames]{color}
\usepackage{graphicx}
\usepackage[margin=0.9in]{geometry}
\usepackage{epstopdf}
\epstopdfsetup{update}
\usepackage{helvet}
%\usepackage{hyperref}
\usepackage{indentfirst}
\usepackage{pdflscape}
\usepackage{morefloats}
\usepackage{natbib} \bibliographystyle{aea}
%\bibliographystyle{abbrvnat}\bibpunct{(}{)}{;}{a}{,}{,}
\usepackage{setspace}
\usepackage{subcaption}
\usepackage[capposition=top]{floatrow}
\usepackage{subfloat}
\usepackage[latin1]{inputenc}
%\usepackage[pdf]{pstricks}
\usepackage{pdfpages}
\usepackage{multirow}
\usepackage[bottom]{footmisc}
\usepackage[table]{xcolor}

\usepackage{pgfplotstable}
%\usepackage{filecontents}
\usepackage[hypertexnames=false]{hyperref}
\usepackage[normalem]{ulem}
% \renewcommand{\familydefault}{\sfdefault}
%===============================================================================
%=== (2) Specific Instructions
%===============================================================================
\hypersetup{
    colorlinks=true,      
    linkcolor=BlueViolet, 
    citecolor=BlueViolet, 
    filecolor=BlueViolet, 
    urlcolor=BlueViolet   
}

%\setlength\topmargin{-0.375in}
%\setlength\textheight{8.9in}
%\setlength\textwidth{5.8in}
%\setlength\oddsidemargin{0.4in}
%\setlength\evensidemargin{-0.5in}
\setlength\parindent{0.25in}
\setlength\parskip{0.15in}

\definecolor{blue}{HTML}{84CECC}
\definecolor{gr}{HTML}{375D81}
\DeclareMathOperator{\plim}{plim}
\newcommand{\Var}{\mathrm{Var}}
\newcommand{\Cov}{\mathrm{Cov}}
\newcommand{\Bias}[2]{\frac{\Cov[#1,#2]}{\Var[#1]}}

%The Importance of a Large Early-Life Social
% Inclusion Program on Neonatal Health Outcomes in Latin America

%  the proportion of low birth weight babies, and the proportion
%  of premature births. What's more, we validate micro-level between-mother
%  estimates with difference-in-difference estimates based on time-varying program
%  roll-out at the sub-national level.  Taken together our estimates suggest
%  that program participation increased weight at birth by 12 or 13 grams, at
%  an estimated public cost of \$18 per gram.  These estimates are comparable to
%  those observed in a developed country setting and have important efficiency and
%  equity implications for a developing economy. We show that program participation
%  closed the prevailing early life health gap between targeted program
%  participants and richer non-participants, and that they imply considerable
%  changes in cognitive achievement in the long-run.

%  [ACTUAL NEW SUBTITLE]:
%  
%===============================================================================
%=== (4) Title, authors, data
%===============================================================================
\title{\textbf{Growing Together}:
  Assessing Equity and Efficiency in an Early-Life Health Program in Chile\thanks{
    We are grateful to Rodrigo Alarc\'on, Jeanette Leguas and Felipe Arriet of
         the Ministry of Social Development of Chile and Andr\'es Alvarez of the
         Ministry of Health for providing invaluable data linkages and other
         guidance.  We thank Serafima Chirkova, Rudi Rocha, Gabriel Romero, and
         seminar audiences at UNU-WIDER Mozambique, Universidad de Concepci\'on,
         Chile, and Universidad de la Rep\'ublica, Uruguay
         for very useful comments.  We are
         particularly grateful to Fresia Jara and team at Hospital San Juan de
         Dios for providing interviews and discussion regarding day-to-day
         program functionality.  We gratefully acknowledge financial
         support from the CAF Development Bank's Research Program on Health and
         Social Inclusion in Latin America.  Damian Clarke additionally
         acknowledges the generous support from the Comisi\'on Nacional de
         Investigaci\'on Cient\'ifica y Tecnol\'ogica of the Government of Chile
         (grant reference 11160200).  The results and views in this paper are
         our own. Any errors are our own.}}
\author{Damian Clarke\thanks{Universidad de Santiago
    de Chile, and CSAE Oxford. Contact: Department of Economics, USACH, Avenida Libertador
    Bernardo O'Higgins, 3363, Estaci\'on Central, Chile. E-mail: damian.clarke@usach.cl.}
  \and Gustavo Cort\'es Mend\'ez\thanks{Department of Economics, Universidad de Santiago
    de Chile. Email: gustavo.cortes@usach.cl.}
\and Diego Vergara Sep\'ulveda\thanks{Department of Economics, Universidad de Santiago
    de Chile Email: diego.vergarase@usach.cl.}}
\date{\today}

\setcounter{page}{0}
\begin{document}
\maketitle
\thispagestyle{empty}
\vspace{-1cm}



%-------------------------------------------------------------------------------
\begin{spacing}{1.4}
  %126 words
\begin{abstract}
  We estimate the impact of participation in \emph{Chile Crece Contigo} (ChCC),
  Chile's flagship early-life health and social welfare program, on neonatal
  health outcomes. We use administrative birth data matched to social benefits
  usage, and the staggered program roll-out, to identify the impact of
  participation.  %We validate results using within mother-variation in program
  %access across time.
  We find that this targeted social program had significant effects on birth
  weight (approximately 10 grams) and other early life human capital measures.
  These benefits are largest among the most socially vulnerable groups, however
  shift outcomes toward the middle of the distribution of health at birth. We
  show that program is efficient when compared to other successful neonatal
  health programs around the world, and find some evidence to suggest that
  maternal nutrition and increased links to the social safety net are important
  mechanisms of action.
\end{abstract}
\noindent \textbf{JEL codes}: H23, O15, I14, H43, O38, H51. \\
\noindent \hspace{1mm} \textbf{Keywords}: Public health; neonatal health; social
security; efficiency; early life investments. \\

\clearpage
\section{Introduction}
The importance of early life health over the entire life course of an
individual has been extensively recognised in the economic (and
non-economic) literature \citep{Almondetal2017,AlmondCurrie2011,Barker1990}.
This has lead to public investments in fetal and infant health in a wide
range of contexts.  Spending on infant and maternal health during the
pre-natal period often makes up a central pillar of the social safety net in
developed countries (see for example discussion in \citet{BitlerKaroly2015}
with respects to the US), while considerable public spending is focused on
technological innovation and remedial investments to improve neonatal health
outcomes \citep{Almondetal2010,Bharadwajetal2013}.  Influential work points
to the importance of health as a determinant of equality within countries
\citep{Deaton2003}, and document the long-shadow of early life insults to
health in the developing world \citep{CurrieVogl2012}.   The social
determinants of health starting
\emph{in utero} have lead to the recent design and implementation of
many large, targeted early-life social safety-net programs throughout
the developing world \citep{Monteiroetal2015}.

An important motivation of these early-life health policies owes
to the dynamic complementary between the efficiency of investments in health
early in life and investments later in life.  In an influential series of
papers, \citet{HeckmanCunha2007,CunhaHeckman2009,Cunhaetal2010} argue that
early-life remedial investments are not only efficient, but need not face
equity--eficiency trade-offs implicit in later life remedial investments.
%This has lead, at times explicitly, to policies targeting early-life health
%outcomes as a basic column of the social safety net across the developed and
%the developing world.\footnote{For example,
%  in the context of Chile Crece Contigo---the policy we propose to examine in
%  this paper---the policy's design explicitly references
%  \citet{HeckmanCarneiro2003} as support for the implementation of a
%  large-scale early-life social program (see the official materials in
%  \citet{Arrietetal2013}).}

In this paper we study the equity and efficiency implications of a large
targeted public health program.  We examine the program \emph{Chile Crece
  Contigo} (hereafter ChCC): a national-level health program 
explicitly designed to target early-life health in vulnerable groups.
ChCC was implemented in Chile in 2007, offering a basket of medical
and social services, information and supplies to all expectant mothers
enrolled in the public health system, as well as their children once they
are born.  As well
as a transversal series of benefits, an additional series of means-tested
benefits were provided to families classified as part of the 60\% of most
vulnerable families in the country.  ChCC also has a stated
aim of addressing divergent health outcomes in socially excluded groups,
releasing materials in both Spanish and native indigenous languages, given
the well-documented health disparities among indigenous people across the
world, and in Chile \citep{Andersonetal2016}.\footnote{Chile's population
  is 4.58\% indigenous, the majority of whom are Mapuche, and this group
  has been documented as having poorer birth, neonatal and child health
  outcomes \citep{Andersonetal2016}.}%\textsuperscript{,}\footnote{In order
  %to provide an idea of the program's scope, we provide a brief list of
  %program benefits in appendix \ref{ascn:benefits} of this paper.}

ChCC is the flagship early life health program in the Chile, and one of
the largest social safety net programs of any type in the country.  It is
presented as a successful case of scaling-up development interventions in
highly cited work
\citep{Richteretal2017}, and has been replicated, largely unchanged, in
other contexts.\footnote{For example, \citet{Marroigetal2017} describe the
  program Uruguay Crece Contigo, which was designed following ChCC.} Despite
the size and scope of ChCC, as well as the attention paid to its rollout and
scale-up, few rigourous or well identified studies have been
conducted on the program's effectiveness, and none %, as far as we are aware,
have examined the policy's effect on birth outcomes or survival during
gestation. The headline results from our paper suggest that this program
\emph{has} been successful in improving neonatal health in Chile, suggesting
the attention paid to the program is not unwarranted.  We find
that the effect of program participation on birth weight is approximately a
10 gram increase, and observe some evidence to suggest that the program
may also have reduced rates of fetal death and other health outcomes at
birth.

\paragraph{Assessing Program Equity}
However, beyond mean impacts of the program, we are interested in studying
the program's distributional impacts on the population of Chile.  ChCC
is universally available in the public health system, however has means
tested components designed to close health and developmental gaps which
open early in life. In particular, we consider two equity considerations
relating to ChCC's impacts.  Firstly, we examine whether the program
impacts the most vulnerable population groups, and secondly we examine
from what part of the health distribution at birth mean program impacts
flow.  In terms of the first consideration, we do find that ChCC has
largest effects among vulnerable (targeted) families.  However, when
considering the impact of ChCC across the distribution of health at birth
we find that the largest impacts come towards the middle of the distribution.
While we do observe universally positive impacts of ChCC participation on
both birth weight and weeks of gestation across their distribution, we
estimate that these impacts do not become
statistically significant until 2,000 grams and 36 weeks respectively, and
are largest when considering babies weighing 3,500 grams, and born at full
term.  These results suggest that targeting poor health may be significantly
more challenging than targeting vulnerable families.  Nonetheless, it is
recognised that health improvements even above the median outcome have
considerable long-term impacts \citep{Royer2009}

 %\footnote{As well as searching the economics
  %literature, we conducted a search on PubMed using the keywords
  %``Chile Crece Contigo'' AND ``Child Health'' OR ``Chile Crece Contigo''
  %AND ``Neonatal Health'' OR ``Chile Crece Contigo'' AND ``Birth weight''
  %OR ``Chile'' AND ``Neonatal Health'' over the span 2006 to 2014
%(ie after the design of the program).}

\paragraph{Assessing Program Efficiency}
In terms of total cost, ChCC is one of the largest health or welfare
programs in Chile.  Recent figures suggest that ChCC spending currently
accounts for almost 1\% of the national budget. And in terms of coverage,
this program is substantial, currently reaching between 75 to 80\% of all
newborns in the country.
%We also observe that program participation reduces the
%proportion of low birth weight babies.  Similarly, we observe an
%increase in gestational length.  Notably, these impacts are focused on
%the most vulnerable families who receive a richer set of program
%benefits, including home visits in some cases.
%We find that the program has positive impacts across the birth weight
%distribution, but that these are most substantial after the 60th
%percentile of birth weights.  In general, when considering health
%outcomes of both birth weight and gestational length, although effects
%are universally positive, these only become large and statistically
%significant when moving away from the bottom of the health
%distribution.
%reduction in the frequency
%of premature births.  Given the large coverage of the ChCC program,
%these results are noteable in national level averages, and appear to
%\emph{eliminate} the birth weight differential between the poorer
%program participants and the less-poor non-participants.
To put the program's estimated effects in context, we calculate the inferred cost
of producing a gram of birth weight, and the implications of this to
educational attainment later in life.  When combined with the cost of
running Chile Crece Contigo, our estimates suggest that the government
spends around \$11 per gram of birth weight---a figure that is comparable
to other large successful neonatal health programs, including those in
developed countries, (such as the Special Supplemental Nutrition Program
for Women, Infants, and Children, or WIC, in the US).  Our estimates
suggest that ChCC is efficient when compared to other programs which
explicitly target health at birth, and that the cost per gram of birth
weight is considerably lower than programs which do not explicitly target
health at birth, but which have been documented to have unintended positive
impacts on these outcomes
(such non-targeted programs include a poverty alleviation program in
Uruguay and the Food Stamp Program in the US). What's more, given the well-known
positive effects of birth weight on later life outcomes, we estimate that
as an \emph{upper bound} cost, each \$3700 spent on Chile Crece Contigo
results in an additional 0.05 standard deviations of educational attainment
on later life test scores.  These results suggest that targeted public
health and social welfare programs can have large impacts in developing
and emerging economies, and that these impacts may last much longer than
the period in which an individual is enrolled in the program.  

\paragraph{Program Impacts and Mechanisms}
In this study we take advantage of administrative data from vital stastics
and enrollment in public programs to conduct the first study of ChCC's
impact \emph{in utero}, drawing identification from two (different) sources.
The first, and principal method, is based on time- and geographic variation
in program roll-out and intensity in a differernce-in-difference style
setting.  As a consistency check of these results, for a subset of women and
children, we observe the mother's use of public programs, and so exploit
within-mother variation in exposure produced across siblings around the date
of the policy's introduction.

%The importance of investment in health---and early-life health in
%particular---as a driver of individual and national outcomes in the
%developed world has been flagged in various dimensions.  
%Components of ChCC are already being implemented in other contexts,
%including a largely similar program in Uruguay, and ChCC is being
%hailed as a successful example of scaling up early childhood development
%program.
Given that ChCC provides a basket of health and social support services to
participants, after considering the net program impacts, we briefly examine
the mechanisms of program action.  We find suggestive evidence that pre-natal
nutritional supplements
for mothers, and increasing linkages between families and the social
safety net are important drivers of improvements of health at birth.
All in all, the lessons from ChCC suggest that targeted health policies
can have a substantial impact on birth outcomes of their intended
recipients, but point to remaining challenges in shifting very poor
outcomes even with quite intensive investments.  

%In what remains of this paper we briefly describe the ChCC program,
%as well as the matched administrative data that allows us to link
%birth outcomes with ChCC usage and intensity.  We discuss the proposed
%estimation strategies to determine the impact of ChCC on neonatal health,
%discuss estimated results, and in closing estimate the efficiency of
%public spending on this program, benchmarking against other public
%neonatal health programs, as well as the estimated value of improvements
%in health at birth in Chile.

\section{Background}
\label{scn:backgroun}
\subsection{Chile Crece Contigo}
\label{sscn:ChCC}
Chile Crece Contigo is a multidimensional early-life health program,
targeting children from the first pre-natal check-up during gestation,
and following them through the first
four years of their life. From 2018 onwards, this will be extended
to the first seven years of life with the implementation of a mental
health component.  It is the Government of Chile's flagship social
security program for children, reaching in some form approximately
75-80\% of children in the country.  The most comprehensive set of
benefits are targeted to children from the 60\% of most vulnerable
families.\footnote{``Vulnerability'' has historically been measured
  using a deterministic score assigned by Government social workers,
  known as the \emph{Ficha de Protecci\'on Social} (FPS), or Social
  Protection Score.  Families with a FPS inferior to 13,484 points are
  classified as belonging to the 60\% of most vulnerable households.
  Additional details of the FPS can be found in \citet{Herreraetal2010}.}
ChCC is jointly implemented by the Minsitry of Social Development, the
Ministry of Health, the Ministry of Education, and the Ministry of
Labour, and is delivered by a local network of public providers in each
municipality (known as the Chile Crece Contigo Municipal Network).

%\textcolor{red}{[UP TO HERE WITH REVIEW]}\\
The program was implemented gradually throughout the country, starting
in June of 2007.  The yearly expansion in program size, both in terms of
total municipalities covered and the proportion of all pregnancies
nationwide, is displayed in Figure \ref{fig:coverage}. In the first year
the program covered 159 of Chile's 346 municipalities, before being
extended to all municipalities in early-2008.  We provide a description
of the geographic dispersion of roll-out in Appendix Figure \ref{fig:map}.
Early-implementing municipalities were not chosen at random, but rather
were targeted given the availability of key infrastructure and the ability to
manage the program in existing space in hospitals and health clinics
\citep{Arrietetal2013}, explaining the earlier rollout to less-densely
populated regions in the north and south of the country.  We return to
discuss this in section \ref{scn:methods} when discussiong estimation
strategies.

Program participation
among pregnant women also increased in line with geographic coverage.  
The proportion of all births in Chile receiving at least some ChCC
benefits during gestation are displayed as the solid line in Figure
\ref{fig:coverage}.  By the time the program was fully rolled-out, the
program reached approximately 70\% of all births nationwide, a figure
which has remained quite steady over time.  The design of the program
means that there is no explicit demand-side, as all individuals
enrolled in the public health service\footnote{The Chilean health system
  consists of a private and public stream and users nominally choose
  between public or private care. An associated monthly payment is
  automatically deducted from all formal salaries as a previsional payment.
  This payment is either made to the public health insurance (FONASA) or
  a private health insurer known as an ISAPRE.  Any individual unable to
  pay contributions is covered by the public FONASA system.  The private
  system is considerably more costly in terms of out of pocket costs.
  Recent administrative data suggests that 76\% of the population is
  covered by public care.  Nationally, 67\% of beds are in the public
  system and the remaining 33\% are in the private system \citep{DEIS2016}.}
automatically participate in ChCC from their first pre-natal check up,
until the child `graduates out' of the program when entering the primary
schooling system.  Thus, program participation is entirely determined
by the supply-side, which depends on each municipality's date of entry
into ChCC.  The program was institutionalised as a basic pillar of the
Social Security system in 2009, with the approval of a law\footnote{The
  Law 20.379 was passed unanimously by parliament on April 2nd, 2009 to
  ``institutionalise the subsystem of integral protection of infancy,
  Chile Crece Contigo''.} guaranteeing its ongoing existence.

\begin{figure}[htpb!]
  \begin{center}
    \centering
    \caption{Usage of Gestational Component of ChCC by Month}
    \includegraphics[scale=0.97]{../results/ChCCtime.eps}
    \label{fig:coverage}
  \end{center}
  \vspace{-5mm}
  \floatfoot{\textsc{Notes to Figure \ref{fig:coverage}}: Program usage
    by month and municipality, and proportion of all births covered nation-wide
    is calculated from administrative data provided by MDS.  This captures
    the proportion of all mothers giving birth each month who participated in the
    pre-natal components of ChCC prior to giving birth.  Additional details can be
    found in section \ref{scn:data} of this paper.  Geographic distribution of
    municipal roll-out is provided in Appendix Figure \ref{fig:map}.
    %The program continued after 2010 and is due to do so indefinitely,
    %however we only consider the period up until 2010 given that our matched
    %birth data end in this year.
  }
\end{figure}

%In Figure \ref{fig:coverage} we plot administrative figures of
%program usage over the life of the program.  In 2010 the program covered
%greater than 200,000 pregnancies nationwide.\footnote{According to vital
%  statistics data, in 2010 there were 250,643 live births.  We note
%  however that a pregnancy will be contained in the ChCC administrative
%  data plotted in Figure \ref{fig:coverage} even if the mother miscarries
%  or a fetal death is recorded as long as the mother has attended at
%  least one prenatal check-up.}

The program consists of two main pillars.  The first is the Program
Supporting Bio/Psycho/Social Development (PADBP), and the second
is the Program Supporting New-Borns (PARN).  The PADBP pillar begins
at the first pre-natal medical check-up, with the main goal of
supporting fetal and child development by providing information and
ongoing support in periodic check-ups, and in certain circumstances,
home visits.  We outline the principal components of the PADBP pillar
in more detail below.  The second program arm, the PARN, begins at the
birth of the child.  Among other things, this pillar provides a
comprehensive kit of materials to all newborns born in the public
health system including a crib, blankets, baby carrier, toys and
didactic materials, clothing and sanitary products.  In what remains
of this section we provide a description of the components of the PADBP
program, focusing only on the pre-natal components.  We focus on this
program arm in more depth given that we examine ChCC's impact on
health at birth, which can only respond to prenatal investments, rather
than health after birth. We provide additional discussion of the program,
including both pre- and post-natal components, in Appendix
\ref{app:program}.


\paragraph{Pre-Natal Components of ChCC} The design of ChCC called for
an increase in the amount of time spent on pre-natal controls (with
midwives in public health clinics) from 20 minutes per appointment to
40 minutes per appointment.  The increased time was used on newly
incorporated components, such as the application of standardised
tests for pre-partum depression, social support programs, and information
to encourage the participation of fathers or partners in preparations
for having a child.  ChCC targets 7 pre-natal controls in public
health centres.  At the date of the first pre-natal control, families are
supplied with an information kit (in Spanish or one of five indigenous
languages or regional dialects), as well as a (music) CD for pre-natal
stimulation.  Any person meeting a set of pre-defined risk factors%
\footnote{These factors are: a first pre-natal control at 20 weeks or
  later, the pregnant women being aged under 18 years, having 6 or
  fewer years of primary education, insufficient family support,
  ``rejection of the pregnancy'', symptoms of depression, substance
abuse, or suffering from intra-family violence.} also
receives an additional psycho-social evaluation to determine whether
they are referred for immediate additional support.  The ChCC program
also delivers nutritional components to expectant mothers.  This
principally consists of a fortified powdered milk disbursed by the
kilogram at local health centres.  The formula of this product was
also changed during the ChCC program to more accurately meet the
nutritional needs of pregnant women. We return to discuss mechanisms
of the program's action in more depth later in the paper.

Along with these universal benefits, families flagged as 
pertaining to the 60\% most vulnerable of the populuation receive a
series of preferential benefits.  These benefits begin at the
first pre-natal check-up with the definition of a personalised
plan created between municipal health workers and families, as
well as hour long home visits from social workers and paramedical
technicians.\footnote{These home visits are not universally
  offered among the preferential group.  Home visits are targeted
  to families with a greater number of risk factors.} Finally,
vulnerable families are referred to the ChCC Municipal Network,
which includes meetings with municipal workers offering information
related to education and labour market programs where relevant,
information regarding other government programs and community
services, and eventually access to free child care. \emph{In situ}
interviews conducted with midwives and social workers involved in
the program highlighted that the implemention
of ChCC resulted in a considerable increase in the quality of
pre-natal care offered, and the ability to easily refer families
between institutions.  We provide additional information regarding
the scope and design of the program in Appendix \ref{app:program},
and a comprehensive list of program benefits is available in
\citet{MDS2014}, and summarised in Appendix Table \ref{ascn:benefits}.
%\textcolor{red}{Add discussion about prenatal care coverage in Chile.}

\subsection{Existing Evidence on The Impact of Early Life Programs on Infant Health}
A well-established body of work---much in the economic
literature---has documented the importance of public policies
on indicators of health at birth and during gestation.  These
can be broadly split into two types of programs: those explicitly
targeting infant health, and those with indirect impacts on
infant health.

There is relatively less evidence on programs explicitly
targeting infant health. Nevertheless, convincing evidence
from the United States shows that publically provided food and
nutritional advice to pregnant mothers has considerable effects
on birth outcomes.  The Special Supplemental Nutrition Program
for Women, Infants, and Children (WIC), has been shown to have
appreciable impacts on health at birth (refer to
\citet{BitlerKaroly2015} for a clear overview).  A number of
policies directly designed to targeted health at birth exist
in Latin America, though often rigourous evaluations have not
been implemented. These include programs such as Plan Nacer
(Argentina) and Qali Warma (Peru).  One notable exception is
a CCT from Bolivia.  \citet{Celhayetal2016} identify a significant
reduction in rates of still birth following receipt of a relatively
small CCT.  In section \ref{scn:efficiency} of this paper we benchmark
the impacts of a range of early-life health programs such as WIC.

Evidence also exists on the impacts of non-targeted welfare policies
on health at birth.  Analysis from the United States suggests that the
Supplementary Nutrition Assistance Program (Food Stamps) may increase
birth weight by as much as 20 grams \citep{Almondetal2011}, and
unintended impacts on child health have also been identified from the
Earned Income Tax Credit \citep{Hoynesetal2015}.  Another series of
papers documents the impact of receipt of conditional cash transfers
on infant health, even when these transfers were not directly targeting
these outcomes.  This includes the PROGRESA/Oportunidades program
in Mexico \citep{Barham2011}, and the PANES program in Uruguay
\citep{Amaranteetal2016}

%Lim, S. S., Dandona, L., Hoisington, J. A., James, S. L., Hogan, M. C., and Gakidou, E.
%(2010). India’s janani suraksha yojana, a conditional cash transfer programme to increase
%births in health facilities: an impact evaluation. The Lancet, 375(9730):2009–2023.

%Barham, T. (2011). A healthier start: the effect of conditional cash transfers on neonatal
%and infant mortality in rural mexico. Journal of Development Economics, 94(1):74–85.

\subsection{Other Social Safety Net Programs in Chile}
Chile Crece Contigo joined a number of other targeted social
security programs in Chile.  However, unlike other programs
offered by the Ministry of Social Development, Chile Crece
Contigo focuses exclusively on the early life stages, and
covers a large proportion of the population of Chile.

The Chile Solidario program is focused on poverty reduction, and
is targeted to the most vulnerable 10\% of the population.  This
program includes a cash transfer (which fades out over time) and
a series of home visits.  This program has been demonstrated to
increase the take up of employment programs, as well as participation
in other public policies \citep{Carneiroetal2014}.  Other
programs targeted to families with children include the Subsidio
\'Unico Familiar, a subsidy for families with children, as well
as a series of targeted scholarships and school meal programs.
In each case, these policies are targeted to a more restricted
group than ChCC recipients \citep{Herreraetal2010}.  One
component of the (targeted) component of ChCC is ensuring that
vulrenable families are adequately enrolled in additional
social policies for which they are eligible.  We examine the
potential link between ChCC usage and connection to the social
welfare network more generally in section \ref{scn:Mechanisms} of
this paper.

\section{Data}
\label{scn:data}
\paragraph{Birth Outcomes} Vital statistics covering all births
occurring in Chile are publicly available from 1990 until 2015 from
the Ministry of Health. Additonally, data on fetal deaths occurring
after 22 weeks of gestation are available from 2002 onwards. Vital
Statistics data in Chile covers greater than 99\% of all births,
and coverage is stable over time.  In this paper we use the full
universe of births and fetal deaths occurring between 2003 and
2010, and match this with administrative data
on Chile Crece Contigo usage in the gestational period provided
by the Ministry of Social Development (MDS).  This data allows us
to calculate usage by month for each of the 346 municipalities of
Chile.\footnote{Municipalities in Chile are the third level
  administrative district, and the lowest level of local
  governance, after provinces and regions.  In Chile there are
  346 municipalities, 54 provinces, and 15 regions.}  The precise
date of program roll-out by municipality is also provided by the
MDS.

This birth data allows us to observe a range of human capital measures
at birth.  These are the weight of the baby, the baby's length
in centimetres, and the gestational length as recorded at birth. These
measures have been consistently shown to have large and long-lasting
effects on health and well-being \citep{AlmondCurrie2011b}.
Although Apgar and head circumference are measured at birth in
Chile, they are not currently available in administrative data.
Along with measures of health immediately at birth, we are able
to calculate rates of fetal death per live birth by combining
fetal death registers with live birth registers.  While fetal death
data only records deaths occurring in hospitals, recording is
consistent throughout the country (see for example
\citet[p.\ 22]{Rauetal2017} for additional details).  

Administrative (micro-) data is collapsed at the municipal by
month level, and matched with data on ChCC intensity by municipality
and month.  We match all births occurring between January of 2003
and December of 2010 (inclusive), surrounding the program's roll-out.
ChCC data is available from mid-2007 (the first date of program
roll-out) until 2010, and the pre-2007 period provides coverage of
the pre-reform dates.  This results in a sample of 1,917,085 births
occurring to 1,241,514 mothers. When collapsed to the municipal
level, this results in 31,842 municipal$\times$month observations.
The theoretical maximum number of observations is 346 municipalities
$\times$ 8 years $\times$ 12 months (33,216 municipalities), but a
number of smaller municipalities do not have births in each month.

In Table \ref{tab:sumstats} we provide summary statistics of
principal health indicators at birth, as well as rates of participation
in Chile Crece Contigo by municipality and month.  Municipal-level
averages are largely in agreement with values observed in Vital
Statistics data observed elsewhere (we also provide summary statistics
at the level of births in Appendix Table \ref{tab:sumstatsMother}.  The
average birth weight in municipal averages is approximately 3,350 grams,
gestation is on average 38.7 weeks, and 5 and 6\% of births are low birth
weight or premature (respectively).  In administrative data from 2003 to
2010, 25\% of mothers are observed to participate in Chile Crece Contigo,
though this value is considerably lower than actual participation rates
once the program was implemented, as the program only began running from
June of 2007 onwards.  Rates of usage of the program (only the gestational
component) by time are displayed in Figure \ref{fig:coverage}. In Appendix
Figure \ref{fig:usage} we present the proportion of ChCC usage by
municipalities once the program was implemented.  We observe considerable
variation in program intensity by municipality, in line with the proportion
of the population attending births and pre-natal check-ups in the public
health system.  In examining the number of births occurring in each
municipality in Table \ref{tab:sumstats} (``Number of Births'') we
observe considerable variation in the size of municipalities. Depending on
the municipality, the number of births per month ranges from as low as 1
birth (conditional on there not being 0 births) to as high as 787 births.
As we discuss below, regression estimates are consistently weighted by the
number of births per cell.

\begin{table}[htpb!]
  \begin{center}
    \caption{Summary Statistics: Birth and Chile Crece Contigo Data}
    \label{tab:sumstats}
    \begin{tabular}{lccccc} \toprule
      & N& Mean & Std. Dev. & Min & Max \\ \midrule
      \input{../results/SummaryMunicipal-update.tex} \midrule
      \multicolumn{6}{p{14.2cm}}{{\footnotesize \textsc{Notes to Table
            \ref{tab:sumstats}}: Summary Statistics are displayed for
          municipality by month averages for
          each month from January 2003 to December 2010.  Averages are
          displayed for each municipality in which there is at least one
          birth in the given month.  The average number of births by
          comuna and month is displayed above.  There are 346 municipalities
          in Chile, and hence a maximum number of observations of 346
          municipalities $\times$ 8 years $\times$ 12 months, or 33,216
          municipality$\times$ month observations.  The difference between
          this maximum and the observed number of observations are cases
          where no births occurred.  Uncollapsed micro-data on births
          consists of 1,917,086 observations between 2003 and 2010.
          Additional details on this birth data is provided in
          Appendix \ref{app:context}.  Proportion enrolled in ChCC
          refers to the average proportion of births in each municipality
          which were covered by ChCC \emph{in utero} during the entire
          period of 2003-2010, and so is always zero prior to the implementation
          of ChCC in 2007/2008.
      }} \\ \bottomrule
      \end{tabular}
  \end{center}
\end{table}

For a sub-set of births, we are able to match all siblings with
mothers, as well as with the mother's participation in social
programs.  For these mothers we thus observe her full fertility
history, as well as whether she participated in Chile Crece Contigo,
and her social vulnerability score, defining the degree of usage
of ChCC for which she will be eligible.  Approximately 40\% of
births are correctly matched to their mother.  We thus use this
matched micro-data sample as an auxiliary test of the main result.
While this does not include the full universe of births used in
the municipal level analysis, the resulting data set is a unique
source of information on births in Chile matched to the mother's
take-up of social safety net programs.  In appendix \ref{MFE}
we discuss the match rates, as well as the characteristics of
the matched and unmatched sample.  The 60\% of unmatched children
were overwhelmingly matched with their father rather than their
mother in the social registry, and so are excluded from micro-level
analyses given the lack of information on the \emph{mother's} usage
of public programs.

\paragraph{Chile Crece Contigo Data} Administrative data on ChCC
usage as well as the exact date of rollout is provided by the
Ministry of Social Development of Chile.  As discussed in section
\ref{sscn:ChCC}, program rollout occurred gradually, based on
infrastructure availability, and is documented geographically
in Appendix Figure \ref{fig:map}.  Administrative figures for
intensity of program use are also provided by the Ministry of
Social Development which record the proportion of births in each
month and municipality which used at least some ChCC components at
some point of their gestation.  The trend in this measure over
time was plotted in Figure \ref{fig:coverage} of this paper.
We also collect month-by-month figures describing the usage of a
number of key program components from the Department of Health
Information of the Minsitry of Health.  We return to discuss
this data when examining the program's mechanism of impact.

\section{Methodology}
\label{scn:methods}
\paragraph{Estimating the Impact of ChCC}
We leverage the time-varying roll-out and intensity of ChCC by
municipality to estimate the following flexible
difference-in-differences (DD) model:
\begin{equation}
  \label{eqn:DD}
  Infant Health_{ct} = \alpha_0 + \alpha_1 ChCC_{ct} + \bm{W_{ct}\alpha}_{w} + \mu_t + \lambda_c + \eta_{ct}
\end{equation}
where $InfantHealth$ measures average birth outcomes for each
municipality $c$ in period $t$.  In principal specifications,
the unit of time is month by year. The variable $ChCC_{ct}$ is
a treatment measure indicating the proportion of all births in
each municipality and month which received coverage from the
\emph{Chile Crece Contigo} program during gestation.  This measure is
always 0 prior to the program implementation, and increases
to reach approximately 75\% of the population following the
program's implementation.  Given that fact that the program
was implemented in different municipalities at different times,
we include full municipality and time (month$\times$year) fixed
effects as $\lambda_c$ and $\mu_t$ respectively.  As we discuss
above, the measure of $ChCC$ depends on program roll-out as
well as the proportion of a municipality which is enrolled in
the public health system.  This share is largely fixed by
municipality, and is higher in municipalities with a larger
proportion of low income households.  In Appendix Figure
\ref{ChCCenrol} we present scatter plots of the level of
municipal enrollment, and various municipal characteristics,
where, unsurprisingly, higher ChCC usage is associated with
greater poverty shares and vulnerability.  While we could use
a simple binary measure for $ChCC$ availability in specification
\ref{eqn:DD}, this is practically challenging, given that
there is considerable variation in actual usage of ChCC for
different time periods and municipalities, and replacing the
continuous intensity variable with a binary availability variable
results in much less identifying variation.  Nonetheless, we
present this specification as an Appendix model.  Similarly,
we present an appendix specification where we instrument
$ChCC_{ct}$ with lagged usage in the same municipality.

If implementation of the policy were completely random,
$\alpha_1$ will give the unbiased effect of ChCC participation on
infant health measures.  However, as we may be concerned that
early adopting municipalities with better infrastructure were
following different trends over time, we propose to include
a series of time-varying controls for health infrastructure
and municipal development $\bm{W_{ct}}$, and in supplementary
regressions also examine the robustness of results to regional
and municipal time trends, and separate regional and municipal
fixed effects for each year. %\footnote{Despite these considerations,
  %we note that there is no particularly notable geographic
  %clustering of early- and late-adopted municipalities,
  %even within metropolitan areas such as Santiago (refer to
  %Appendix Figure \ref{fig:map} to observe the variation in
  %roll-out by area).}
As is typical, we will cluster standard errors by municipality
(346 municipalites) to account for the well-known time-dependence
in unobserved stochastic errors by geographic area
\citep{Bertrandetal2004,CameronMiller2015}.  We discuss a number
of placebo checks below.

Our principal outcome measures of $InfantHealth$ are based on
the available measurements recorded in vital statistics data,
and consist of birth weight in grams, low birth weight
($<$2,500 grams), birth length in centimetres, gestation time
in weeks, prematurity ($<$ 37 weeks gestation), and the frequency
of fetal deaths.  Given that we propose to use various outcome
measures and a single indepedent treatment variable, we will
correct for multiple hypothesis testing in a number of ways.  We
briefly return to this point at the end of this section.

%\subsection{Inference, Robustness Strategies and Extensions}
\paragraph{Alternative Identification Strategies}
While our main identification strategy takes advantage of
the time-varying roll-out of Chile Crece Contigo by municipality,
we also estimate a child-level regression controlling for mother
fixed effects leveraging within mother variation in policy exposure.
For each mother in matched administrative data we observe all
births occurring between 2003 and 2010, both before and after policy
implementation.  The inclusion of mother fixed effects thus allows
us to capture all time-invariant unobservables of mothers correlated
with program participation.  We also include a number of time-varying
controls, including maternal age and birth order fixed effects.

We estimate mother fixed effect models only as a robustness check
rather than our main specification given that the match between
children and mothers was not universal (while municipal-level
regressions are based on complete vital statistics data).  As
discussed in section \ref{scn:data}, approximately 50\% of births
were correctly merged with data on their mother's use of public
programs, while the remaining births were merged with the father's
social program participation. We provide additional details regarding
the precise mother FE specification to be estimated, as well as match
rates and characteristics of matched and unmatched children in
Appendix \ref{MFE}.

\paragraph{Placebo Tests} We observe monthly usage rates of ChCC
during gestation for each
municipality following the reform's implementation.  This measure
of usage by municipality and time is our independent variable of
interest in specification \ref{eqn:DD}.  In order to ensure that
our estimates for $\alpha_1$ are not simply capturing systematic
differences between municipalities with varying time and intensity
of ChCC, we propose to conduct a series of placebo tests using
lagged measures of the independent variable of interest.%
\footnote{Frequently, identifying assumptions in DD-style models
  are tested by event study analysis, where treatment status is
  interacted with a full set of lags and leads.  In the setting of
  this paper, where program usage is a continuous rather than
  binary measure, an event study is unsuitable given the lack of
  binary treatment, and the fact that all municipalities are
  eventually treated.  We thus proceed with the lagged placebo
  tests as described in this section.}
Specifically, we estimate the following model:
\begin{equation}
  \label{eqn:placebo}
  Infant Health_{ct} = \gamma_0 + \gamma_1^k ChCC_{c,t-k} + \bm{W_{ct}\alpha}_{w} + \mu_t + \lambda_c + \eta_{ct} \qquad \forall k \in {1,\ldots,40}.
\end{equation}
Here, rather than regressing ChCC usage in the month of birth on
concurrent outcomes, we regress usage in earlier months.  Provided
that improvements in birth outcomes are truly flowing from the
program, rather than systematic differences between municipalities,
we should see that lags of $ChCC_{ct}$ do not impact birth outcomes
in future periods conditional upon municipal and time fixed effects.
%Thus, we are interested in testing estimated coefficients
%$\widehat{\gamma}_1^k$ to determine 

%In order to run a consistency check on DD results we propose
%to estimate a full event study.  This event study is a test in the style
%of \citet{Granger1969}.  We will examine precisely when indicators
%diverge between early and late treatment areas, estimating the following
%specification:
%Here we interact a series of indicator variables indicating policy
%implementation leads ($-k$) and lags ($+l$).  If the difference between
%early and late implementing states only emerge when the policy is
%implemented, there should be no differential impact in any of the
%lead terms, suggesting an individual and joint test that each of the
%$\gamma^{lead}$ terms are equal to zero.

\paragraph{Correcting for Multiple Hypothesis Testing}
We examine the impact of ChCC on a number of distinct outcome
variables.  These variables were pre-defined, and are the only
measures of neonatal health avaialble in published vital statistics
data in Chile.  Given that we exame the impact of the policy on a
number of separate outcomes, we correct for multiple hypothesis
testing.  We do this in two ways.  Firstly, in order to ensure
adequate size we apply \citet{RomanoWolf2005}'s stepdown hypothesis
testing algorithm which fixes the Family Wise Error Rate at a set
level $\alpha$.  This hypothesis correction technique is considerably
more powerful than older FWER techniques such as Bonferroni or Holm,
and is increasingly used in the economic literature (see for example
\citet{Gertler2014}).  This is also a more demanding correction than
corrections which fix the False Discovery Rate of findings.
Secondly, we construct a single index based on the full set of
outcome variables which gives more weight to variables which
provide the most independent variation.  To construct this index
we follow the procedure described in \citet{Anderson2008}, allowing
us to examine the estimated effect of ChCC on a single outcome
variable, where variables which provide more independent
information are given larger weights in the index.

\paragraph{Distributional Effects of the Policy}
Finally, along with regressions examining birth weight, and
gestational length, we are able to observe the effects of the
policy over the entire range of these health distributions, to
examine precisely where any average effects are observed.  While
our principal regressions consider two particular points of the
weight and gestational length distributions (low birth weight and
prematurity), we can similarly consider outcomes across the entire
support of the health measures. We follow \citet{RossinSlater2013}
in definining a range of binary variables which take the value of
1 if birth weight exceeds a certain weight, and zero otherwise,
for points from 1000 to 5000 grams.  Similarly, we create binary
measures for gestational length greater than $k$ weeks, where
$k$ is set at 30-41 weeks.  This allows us to determine if mean
impacts vary throughout the distribution of health at birth,
where we simply replicate equation \ref{eqn:DD}, however now with
the range of distributional variables.  Once again in these
specifications we report results both uncorrected for multiple
hypothesis testing, and results accounting for the fact that with
multiple outcomes, we are likely to over-reject the null hypothesis
of a zero-reform impact.

\section{Results}
\subsection{Program Impacts}
\subsubsection{Headline Effects}
\label{scn:headline}
Baseline estimates based on municipality and time-varying exposure
to the Chile Crece Contigo program are presented in table \ref{mDD}.
Estimates in this table are all produced by an archeypical DD model
including ChCC coverage as the independent variable of interest, and
municipality and month$\times$year fixed effects.  Standard errors
are clustered by municipality.

Results from Table \ref{mDD} suggest large and significant effects,
particularly for birth weight and gestational length.  As the
independent variable is measured as the proportion of ChCC coverage
in a municipality, an increase in 1 unit of this variable is equivalent
to moving from 0 to universal ChCC coverage, or the mean impact of
ChCC if the full population were treated. The mean impact of Chile
Crece Contigo is estimated as an 10 gram increase in birth weight.  When
examining the proportion of low birth weight babies, results suggest
that ChCC brought about a reduction in these births by 0.2 percentage
points, however this is not distinguishable from 0 at the 10 percent
level.  When comparing the point estimate to the absolute value of low
birth weight births, this is approximately a 3.7\% reduction. We find no
impact of ChCC on size at birth, but do observe a small increase in
gestational length of 0.24 weeks. No statistically significant effect
is observed when considering the proportion of premature births, though
once again impacts are weakly negative (ie a reduction in premature
births).  Finally, in turning to fetal deaths, we also observe a
significant reduction, of 1.6 per 1,000 live births following the
program's implementation and expansion.

\input{../results/comunaDD_edit.tex}

We examine alternative specifications and controls in Table
\ref{tab:AltSpecs}.  Here rather than simply estimating a baseline
DD model with time and geographic fixed effects, we add additional
time varying controls, region and municipal specific linear time
trends, and region and municipality by year fixed effects.  Even in
the most demanding specification which includes both time-varying
controls as well as separate fixed effect for each municipality in
each year (346$\times$8 fixed effects), estimates largely agree
with those in the baseline DD model.  The estimated effect of ChCC
on birth weight falls slightly (to 9.78 grams), while the remaining
effects are quite stable, with the exception of the estimated effect
of ChCC on the rate of fetal deaths which no longer remain significant
with year by municipality fixed effects.  In some models,
significant positive impacts are observed on birth size and
significant reductions are observed in the proportion of low birth
weight babies, but these are not consistently observed. 
If we estimate using trimester$\times$municipality averages rather
than month by municipality outcomes, estimates remain quite stable
(refer to Appendix Table \ref{mDDt}).  While the precision of
estimates falls slightly, rendering a number of coefficients no longer
statistically significant at typical levels, the effect sizes agree
very closely with those in Table \ref{mDD}. %\footnote{We generally
  %prefer to report month by municipal estimates, as it allows for
  %greater variation in the independent variable of interest.
  %}
Finally, we correct for multiple hypothesis testing in Appendix
Table \ref{tab:MultHyp}.  Panel A presents uncorrected and
corrected $p$-values where we account for the fact that we are
likely to over-reject the null when testing the impact of ChCC
on multiple outcome variables.  Original $p$-values come from
estimates presented in Table \ref{mDD}, while corrected values
follow \citet{RomanoWolf2005}.  This is a demanding correction,
ensuring that no null hypotheses will be incorrectly rejected
at a given size.  In this case, we still observe a statistically
significant effect on birth weight.  When considering an index
capturing infant health (where a positive value implies greater
health), we observe that regression the single index on rates
of participation in ChCC results in statistically significant
impacts.

We examine the plausibility of identifying assumptions using a
series of placebo tests.  These placebo tests use the ChCC
participation rates for each municipality, however assigning the
placebo reform treatment to a period entirely \emph{before} the
corresponding births had occurred.  Thus, if there is no general
prevailing difference in trends between municipalities with different
roll-out timing or intensity of ChCC usage, we should observe
that all placebo tests based on pre-reform dates lead to
insignificant estimates of the effect of the placebo treatment on
birth outcomes.

\begin{landscape}
\begin{table}
  \begin{center}
    \caption{Alternative Specifications: Diff-in-diff Estimates of Program Impacts}
    \label{tab:AltSpecs}
    \scalebox{0.95}{
    \begin{tabular}{lcccccccc} \toprule
      &(1)&(2)&(3)&(4)&(5)&(6)&(7)&(8)\\ \midrule
      \multicolumn{9}{l}{\textbf{Panel A: Birth Weight}} \\
      \input{../results/Alt_peso.tex}
      \multicolumn{9}{l}{\textbf{Panel B: LBW}} \\
      \input{../results/Alt_lbw.tex}
      \multicolumn{9}{l}{\textbf{Panel C: Size}} \\
      \input{../results/Alt_talla.tex}
      \multicolumn{9}{l}{\textbf{Panel D: Gestation}} \\
      \input{../results/Alt_gestation.tex}
      \multicolumn{9}{l}{\textbf{Panel E: Premature}} \\
      \input{../results/Alt_premature.tex}
      \multicolumn{9}{l}{\textbf{Panel F: Infant Mortality}} \\
      \input{../results/Alt_fDeathRate.tex}
      \midrule
      Municipal and Year FEs      & Y & Y & Y & Y & Y & Y & Y & Y \\
      Time-Varying Controls       &   & Y &   &   & Y &   &   & Y \\  
      Region Time Trends          &   &   & Y &   &   &   &   &   \\
      Region $\times$ Year FEs    &   &   &   & Y & Y &   &   &   \\
      Municipal Time Trends       &   &   &   &   &   & Y &   &   \\
      Municipal $\times$ Year FEs &   &   &   &   &   &   & Y & Y \\
      \bottomrule
      \multicolumn{9}{p{19.8cm}}{{\footnotesize \textsc{Notes to Table \ref{tab:AltSpecs}}:
          Each specification is estimated by differences-in-differences using
          municipal-level averages by month, and weights for the number of observations
          in each cell.  Column 1 replicates results from Table \ref{mDD}, and then
          columns 2-8 include additional controls, linear trends, or fixed effects.
          Regions in Chile are the top-level administrative district, of there are
          15.  Municipalities are within districts (analogous to states and counties
          in other countries), and there are 346 municipalities in Chile.  The
          most demanding specification allows for a separate fixed effect for each
          municipality in each year under study, given that there are twelve
          observations for each municipality in each year.  Time-varying controls are
          collected from the Government of Chile's National System for Municipal Information,
          and are available for each municipality in each year.  These controls consist
          of total transfers for education and health, the proportion of each municipality
          enrolled in the public health system (FONASA), the proportion enrolled in school,
          a pre-determined poverty index calculated by the Government, and the coverage
          of drinking water.  Standard errors are always clustered by Municipality. Refer
          to Table \ref{mDD} for additional notes.}} \\
    \end{tabular}}
  \end{center}
\end{table}
\end{landscape}


These results are displayed in Figure \ref{placebo}.  Each point
estimate and confidence interval corresponds to a placebo reform
lagged by the number of periods indicated on the $x$-axis.  In
general, the large majority of placebo tests indicate results
which are not statistically distinguishable from zero.  At times
certain lags result in estimates which are significant at 95\%,
however these generally occur with large time lags, when more
observations are lost given lags in the placebo variable.

%short lags (such as 1 or 2 months) result in significant estimates,
%however this owes to the fact that births occurring in closely
%spaced months were exposed to similar levels of program coverage.
%Moving further away from the true reform dates, few estimates
%are statistically distinguishable from zero, suggesting that
%identifying assumptions underlying DD estimates are reasonable.
%%later (as occurred in the true reform).  We use all time periods
%%for which full coverage is available, until arriving to the date
%%of the true reform (indicated by the red dotted line in the figures).
%%We observe that in nearly all cases placebo reforms lead to smaller
%%and statistically insignificant estimates.  In the case of tests
%%using birth weight, we observe one statistically significant result
%%in the placebo tests (of approximately 40), and none when examining
%%low birth weight.  The effects of the reform only begin to be
%%observed when approaching the true reform date, reaching their
%%maximum estimates when the true reform dates are used.
\begin{figure}[htpb!]
  \begin{center}
    \caption{Placebo Tests}
    \label{placebo}
    \begin{subfigure}{.5\textwidth}
      \centering
      \includegraphics[scale=0.6]{../results/placebolag_peso.eps}
      \caption{Birth Weight}
      \label{placebo-peso}
    \end{subfigure}%
    \begin{subfigure}{.5\textwidth}
      \centering
      \includegraphics[scale=0.6]{../results/placebolag_lbw.eps}
      \caption{LBW}
      \label{placebo-lbw}
    \end{subfigure}
    \begin{subfigure}{.5\textwidth}
      \centering
      \includegraphics[scale=0.6]{../results/placebolag_gestation.eps}
      \caption{Gestation}
      \label{placebo-gest}
    \end{subfigure}%
    \begin{subfigure}{.5\textwidth}
      \centering
      \includegraphics[scale=0.6]{../results/placebolag_fDeathRate.eps}
      \caption{Fetal Deaths}
      \label{placebo-fdeaths}
    \end{subfigure}
  \end{center}
  \floatfoot{\textsc{Notes to Figure \ref{placebo}}: Each point estimate
    and resulting confidence interval display the impact of a placebo test
    where the share of Chile Crece Contigo enrollees is lagged
    $j\in \{1,\ldots,40\}$ months, where $j$ is displayed on the horizontal
    axis. Each placebo test is estimated following the principal
    specification displayed in Table \ref{mDD}.  Additional notes relating
    to each model can be found in Table \ref{mDD}.}
\end{figure}

As discussed in section \ref{scn:methods}, our measure of treatment
intensity is usage of ChCC, which increases following the date of
reform implementation.  If we estimate using a binary measure of
ChCC program availability, results are of the same direction, though
frequently much less precisely implemented (Appendix Table
\ref{mDDavailable}).  For example, in the case of birth weight,
we observe that for those individuals born when the program was
available in utero (but for less than the full 9 months) that ChCC
availability increases birth weight by 1.4 grams, while for those
individuals for whom ChCC was available during the entire pre-natal
period, birth weight is 3.3 grams higher.  These lower impacts are
perhaps not surprising given that there is massive variation in usage
of ChCC even when the program is avaialble.  This is observed
temporaily in Figure \ref{fig:coverage}, where usage expands
considerably during 2007 and 2008, and also by municipality in Appendix
Figure \ref{fig:usage}.  While the rate of use of ChCC when the
program was available was 56.5\% (when weighted by municipal population,
or 60.6\% without weights), certain municipalities have rates of
usage lower than 20\%, while others have rates of usage of nearly
100\%.  Despite the lower precision of results when simply using
a binary available/non-available distinction, if these results are
scaled up based on usage rates of ChCC (following \citet{Almondetal2011}),
results are closer in magnitude to those reported in our main
specification. For example, inflating the ChCC Availability estimate
in Table \ref{mDDavailable} to account for the fact that usage rates
of ChCC where ChCC was available for less than the full pregnancy
was 35.7\% results in an inflated estimate of approximately 4 grams,
while inflating the full availability estimate of 3.25 grams with
usage rates of 56.6\% results in an estimate of approximately 6
grams.  To address concerns that our estimates may reflect the
decision to use ChCC rather than participation itself, in Appendix
Table \ref{tab:ChCC_IV} we present IV estimates, where participation
in each municipality is instrumented by lagged participation rates.
The logic behind these estimates is that while actual participation
may reflect the decisions of the women who gave birth in a particular
month, the participation rates of mothers in prior periods in the
same municipality will be highly correlated with those of mothers
in future periods, however will not reflect that actual characteristics
of the precise group of mothers giving birth.  In this case we observe
that the first stage results presented in Appendix Table \ref{tab:ChCC_IV}
are strong, suggesting reasonably stable rates of usage of public care
within municipalities over time, and second-stage IV estimates agree
in magnitude with those reported in Table \ref{mDD}, however with
slightly inflated standard errors.
%NOTE: Both of these are calculated as:
% sum chcc if ChCCTime>1              [fw=poblacionfem ]
% sum chcc if ChCCTime>1&ChCCTime <=9 [aw=poblacionfem ]

Estimates based on mother fixed effects for the matched sample
are presented in Table \ref{mFE}. We present fixed effects
estimates in each case also controlling for mother's age and birth
parity fixed effects which may vary around the reform date.
Identification is thus driven by changes in birth outcomes between
siblings born before and after their
mothers began participating in Chile Crece Contigo, compared with
similar timed siblings occurring to never-participating mothers.

Once again, we observe that the effect of Chile Crece Contigo
participation is large and statistically significant.  In this
case we \emph{do} observe an impact on the size of the baby at
birth, and the impacts on all other variables remain consistent
with those estimated from municipal-level DD models.  The effect
sizes observed for birth weight and gestational weeks are
considerable.  We estimate an effect of 19 grams in mother FE
models, equivalent to approximately 0.5\% of the mean birth
weight in Chile over the time period examined, and similar to the
reported effects of large successful programs world wide.  For
example, recent evidence suggests that participation in the Food
Stamp Program in the United States, one of the largest and most
costly social security programs, increases birth weight by
approximately 20 grams \citep{Almondetal2011}. Similarly,
participation in the supplementary nutrition program for Women,
Infants and Children is estimated to increase birth weight
by around 17-30 grams \citep{Hoynesetal2011,RossinSlater2013}.
Additional discussion related to the mother FE models, as well
as data match rates is provided in Appendix \ref{MFE}.


\subsubsection{Program Targeting and Equity}
While ChCC is universally accessible for any mother or family
participating in the public health system, the degree of benefits
of the program is means tested, and targeted more heavily to
families identified as most vulnerable.  In Table \ref{tab:FPS}
we estimate the impact of ChCC usage in each of three groups on
average birth outcomes.  In Panel A we examine the impact ChCC
use among the 60\% most vulnerable (the targeted group), in panel
B we focus on the 40\% most vulnerable (in early years, the targeted
group was only the 40\% most vulnerable), and in panel C we
examine the impact of ChCC usage in the non-targeted group (those
with a Social Protection Score in the top 40\%, or those without a
Social Protection Score).\footnote{In practice, the means
  tested benefits also include a considerable discretionary
  component, beyond the simple cut-off score.  For example, the
  home visit component of the program while only available for
  the 60\% most vulnerable, was not available to the full
  vulnerable group given program demands, but rather was
  discretionarily offered by each local health centre based
on perceived need and vulnerability \citep{MDS2014}.}

We observe that the impacts of the program are largest among
those in the targeted group, and smallest among those in the
non-targeted group.  In general, these estimates become somewhat
less precise, however, a clear gradient in estimated impacts is
observed in moving from more to less vulnerable groups, particularly
when considering the impact on average birth weight. The impact of
ChCC for the most vulnerable 60\% is estimated at 9.1 grams, while
it is estimated as -3.1 grams among the non-targeted group.%
\footnote{These estimates \emph{are} statistically distinguishable
  from each other at the 10\% level.  However it is worth noting
  that the estimated value of 9.09 among the 60\% most vulnerable
  is not distinguishable from the estimated average value of 10.09
  reported in Table \ref{mDD}.}
Similar gradients in point estimates are observed in the
probability of being low birth weight, size at birth, gestational
length, and the likelihood of being premature, however none of
these estimates are statistically distinguishable from zero.

\begin{table}[htpb!]
  \begin{center}
    \caption{Impacts by Vulnerability Quintile}
    \label{tab:FPS}
    \begin{tabular}{lccccc} \toprule
      &(1)&(2)&(3)&(4)&(5)\\
      & Weight &\ \ LBW \ \ &\ \  Size \  \ & Gestation & Premature \\ \midrule
      \multicolumn{6}{l}{\textbf{Panel A: 60\% Most Vulnerable}} \\
      \input{tables/FPS_1.tex}
      \\
      \multicolumn{6}{l}{\textbf{Panel B: 40\% Most Vulnerable}} \\
      \input{tables/FPS_2.tex}
      \\
      \multicolumn{6}{l}{\textbf{Panel C: Non-Targeted Group}} \\
      \input{tables/FPS_3.tex}
      %\multicolumn{6}{l}{\textbf{Panel D: No Poverty Score Requested}} \\
      %\input{../results/FPS_4.tex}
      %\multicolumn{6}{l}{\textbf{Panel E: Quintile 5}} \\
      %\input{../results/FPS_5.tex}
      \bottomrule
      \multicolumn{6}{p{14.7cm}}{{\footnotesize \textsc{Notes to Table \ref{tab:FPS}}:
          Identical specifications are estimated as in table \ref{mDD}, however now each
          model is estimated using \emph{only} observations which meet the criteria
          defined in panel headings. Classification of the 60\% and 40\% most vulnerable
          is based on the Government of Chile's offical scoring based on the
          \emph{Ficha de Protecci\'on Social} (FPS, or Social Protection Score in English),
          which is used to classify the degree of benefits received by families in ChCC.
          The official cut-off for the 40\% most vulnerable is a score of 11,734 points or
          lower on the social protection score, and for the 60\% most vulnerable is a score
          of 13,484 points or lower.  Any mother with a score above 13,484 (or who has not
          applied for a score) is not in the targeted group. Additional discussion of the
          FPS is available in \citet{Herreraetal2010}.}} \\
    \end{tabular}
  \end{center}
\end{table}

These results are in line with ChCC's stated aim of closing early-life
health gaps. Equity-promoting early-life health policies are
particularly important in the context of Latin America. Many Latin
American countries are characterised by irregular, rather than universally
poor, infant health outcomes \citep{Belizanetal2007}.  Indicators
are particularly sub-standard among socially isolated groups, including
low-income households, rural communities, and indigenous people.  These
early-life health differentials are only magnified over the life course of
individuals, partially explaining the emergence of significant gaps in
adulthood in education, salary, and morbidity and mortality. In the
Chilean context this has been documented, where divergence of outcomes at
a very young age (birth weight) have important effects on adacemic
achievement up to 18 years later \citep{Bharadwajetal2017}.  We return to
this point in the section which follows.

\subsubsection{Distributional Effects}
Mean impacts suggest that Chile Crece Contigo participation increases
average birth weight by approximately 10 grams and increases average
gestational length by 0.024 weeks.  However, in Table \ref{mDD}, we
found relatively little evidence to suggest that these impacts reduce the
probablity of being born with low birth weight ($<$ 2,500 grams) or
premature ($<$ 37 weeks).  To examine further \emph{where} the mean
impacts are produced, in figure \ref{quintiles}, we present estimates
of the impact of ChCC at various points of the health distribution.
In figure \ref{quintiles-level} we examine ChCC's impact on the
likelihood that birth weight exceeds $x$ grams, where
$x\in\{1000,1500,\ldots,4750,5000\}$, and in figure \ref{quintiles-log}
we examine the likelihood that gestation exceeds a $x$ weeks, where
$x\in\{30,31,\ldots,40,41\}$.  In these figures we present a series
of point estimates and confidence intervals which correspond to the
estimates on $ChCC_{ct}$ from equation \ref{eqn:DD} where the outcome
variable is infant health exceeding the indicated cut-off.

%particular weeks of gestation, the natural logarithm of birth weight, given that it
%is likely that the impact of the policy will be larger when
%baseline birthweights are higher.  The log specification allows 
%us to estimate the constant impact of the policy across the
%birth weight distribution.  We examine the impact of the
%policy starting at the 5th percentile of the birth weight
%distribution, and ranging to the 95th percentile of the
%distribution.

\begin{figure}[htpb!]
  \begin{center}
    \caption{Policy Impact Across the Health Distribution}
    \label{quintiles}
    \begin{subfigure}{.5\textwidth}
      \centering
      \includegraphics[scale=0.6]{../results/Birthweight_Cutoffs.eps}
      \caption{Birth Weight}
      \label{quintiles-level}
    \end{subfigure}%
    \begin{subfigure}{.5\textwidth}
      \centering
      \includegraphics[scale=0.6]{../results/Gestation_Cutoffs.eps}
      \caption{Gestation}
      \label{quintiles-log}
    \end{subfigure}
  \end{center}
  \floatfoot{\textsc{Notes to Figure \ref{quintiles}}: Point estimates and 95\%
    confidence intervals are presented of the impact of Chile Crece Contigo on
    birth weight and gestational length at different points of the distribution.
    Each specification follows equation \ref{eqn:DD}, however instead of using
    mean birth weight or gestational length in each municipality, uses the
    proportion of births exceeding determined cut-points of the distribution
    (displayed on the horizontal axis) as the dependent variable of interest.
    Panel \ref{quintiles-level} displays the estimates when considering birth
    weight, while panel \ref{quintiles-log} presents estimate for gestational
    length.  For additional notes, refer to notes to Table \ref{mDD}.
  }
\end{figure}

In Figure \ref{quintiles-level} we observe that, although point estimates
of the policy on birth weight are universally positive, estimated impacts
are larger, and statistically less likely to be type I errors, at higher
points in the birth weight distribution.\footnote{Here once again we are
  testing many dependent variables on a single treatment variable, and so
  may expect that we will be prone to over-reject null hypotheses of a
  zero effect.  When we correct each graph for multiple hypothesis
  testing using the Romano Wolf step-down procedure, inferential results
  are qualititatively similar (refer to Appendix Table \ref{tab:RWdist}).
  While this may seem surprising given that we test many outcome variables,
  this is a result of the more efficient Romano Wolf procedure, which
  controls for the very high correlation between outcome variables (which
  are based on the same underlying variable) in this case.}  Estimates
first become statistically
significant at 2000 grams, suggesting that ChCC has a small impact on
increasing weight of quite low birth weight babies, before once again
becoming statistically significant from about 3000-3500 grams, which is
quite close to the mean of the distribution (3346 grams).  The impact
peaks at 3500 grams, with the point estimate suggesting that participation
in ChCC increases the likelihood of exceeding this barrier by as much as
1 percentage point.  Quite a similar pattern is observed when considering
the impact of gestational length in Figure \ref{quintiles-log}. While
consistently positive impacts are observed across the gestational length
distribution, these become largest at approximately the mean of the
distribution (39 weeks) and remain considerable even at 40 weeks.  It is
worth noting that Chile Crece Contigo has targeted reductions in the
rates of C-sections, which are extremely high in Chile, at approximately
50\%, or 43\% in the public health system, potentially partially
explaining the increase in gestational length of full-term births. 

Taken together with the findings from section \ref{scn:headline}, these
impacts point to the difficulty in shifting outcomes towards the very
bottom of the health distribution at birth.\footnote{Investments in low
  birth weight babies, which are difficult to determine ex-ante, are also
  significant once the baby is born, and observed to be of low or very
  low birth weight.  See \citet{Bharadwajetal2013} for a discussion of
  public investments in very low birth weight babies in Chile.}  While we
do find a small impact on some low birth weight categories, we observe
here that impacts are higher among larger babies.  Work examining the
impact of the WIC program from \citet{RossinSlater2013} notes a similar
pattern, with the largest impacts occurring 3000-3500 grams.\footnote{
  \citet{RossinSlater2013} uses slightly broader distributional points,
  with estimates at each 500 grams, however the general pattern is very
  similar. It is important to note that such a finding is not universal
  in early life public programs.  Notably, \citet{Orazioetal2013}
  find that the impact of a community nursery program in Colombia
  impacted child height much more at quintiles 10, 25 and 50
  of the height distribution than at quintiles 75 and 90.}  While
this points to the challenge of improving birth outcome at bottom of the
health distribution, especially in large public programs such as ChCC,
these improvements in birth weight even from the median of the birth weight
distribution are certainly not trivial.  Indeed, evidence from
\citet{Royer2009} suggests that returns to birth weight may actually be
highest \emph{above} the low birth weight cut-off. We turn to considerations
relating to these returns, and returns of ChCC in particular, in the
following sub-sections.


\subsection{Program Efficiency}
\label{scn:efficiency}
\subsubsection{External Efficiency}
Chile Crece Contigo is the flagship early life health program
in the Chile, and one of the largest social safety net programs
of any type in the country.  It is also one of the most
important early life health programs in a middle or lower-middle
income country setting worldwide \citep{Richteretal2017}.  As
such, considerations of efficiency in public health care spending
are of considerable importance.  As we describe in Appendix Table
\ref{tab:spending}, spending on ChCC is approaching 1\% of the
fiscal budget per year, documenting the importance of this policy
nation-wide.  Using the current exchange rate, spending on ChCC
in 2010 was approximately USD 422 million, or 600 million in
PPP-adjusted terms.

To provide a broader consideration of the program's impacts and
efficiency given public investment, we calculate the inferred
cost of producing one gram of birth weight through this policy.
In order to do so we compare the total cost of the pre-natal
portion of Chile Crece Contigo with the total grams of birth
weight produced by the policy.  Information on the total costs
of the program in each year included in this paper are compiled
from government reports \citep{Arrietetal2013}.  This breaks
costs down by component, and we display all costs in Chilean
pesos and in US dollars (PPP adjusted and un-adjusted) in Appendix
Table \ref{tab:spending}.  Based on this, we estimate that it
costs USD \$111 for a single participant in the pre-natal
period of ChCC, based on the average PPP-adjusted cost in each
of the four years laid out in Table \ref{tab:spending}.\footnote{
  We note that this refers to the marginal
  costs of the program.  This will thus not include the costs of
  historical infrastructure investment, costs of non-program
  medical care during pregnancy, and so forth.  These costs are
  compared with the benefits from project participation, which
  also are marginal benefits.}
This value can then be compared with the average birth weight gain
per birth to program participant of approximately 10 grams (from
Table \ref{mDD}).  All told, this suggests an average cost of
per gram of birth of \$11 in PPP-adjusted terms (in non-PPP
adjusted terms this is even lower, at around \$7).

%In order to calculate the total birth weight gained due to the
%program, we use our headline estimate (table \ref{mDD}) of
%approximately 10 grams.  Using these values, as well as the total
%number of pregnant women covered per year, the inferred cost of a
%gram of birth weight is approximately 3,750 Chilean Pesos\footnote{
%  This value is calculated using the costs, the total estimated
%  impact, and the number of program recipients as:
%  \[
%  \text{Inferred Cost}=\frac{\frac{1}{5}\times 126,446,362}{11.1 \text{ grams}\times161,834}=14,078 \text{ pesos/gram}
%  \]} (or based on the
%current exchange rate (1 USD is equivalent to 650 CLP),
%approximately 21 US dollars.  Interestingly, this value is similar
%in magnitude to that calculated from the US Food Stamp Program and
%The WIC program \citep{Clarkeetal2017}.  If we compare this to
%the estimated impacts and costs from a successful (non-targeted)
%CCT in Uruguay, we also find that the cost per gram of birth weight
%is of a similar magnitude.  The estimates presented in
%\citet{Amaranteetal2016} suggest that the cost of producing a
%gram of birth weight was approximately 17-30 USD.\footnote{These
%  values are calculated using the monthly cost of the program
%  (56-102 USD) multiplied by the number of months of pregnancy,
%  and divided by the program's estimated impact on birth weight.
%  We thus estimate the impact as:
%  \[
%  \text{Inferred Cost}=\frac{\$102\times 9}{30.8 \text{ grams}}=29.8 \text{ \$/gram}.
%  \]
%}

In order to put these estimates in context, we can compare them to a
series of successful early-life programs in other countries.  In Table
\ref{tab:impactsOther} we collect all estimates of the impact of
early-life public programs on outcomes at birth where birth weight is
available as an outcome, and where administrative data on birth outcomes
are available.  This results in a series of comparison programs.  These
are largely from the US (WIC, the Food Stamp Program and the Early
Income Tax Credit), however one estimate is also available for a CCT
program from Uruguay \citep{Amaranteetal2016}.  It is important to note
that not all of these programs actually target health at birth (in the same
way that ChCC explicitly targets early-life health).  Thus we can split
the programs listed above into those which explicitly target health at
birth (WIC and ChCC), and those which do not (PANES, FSP, EITC) but
which have nonetheless been documented to have unintended impacts on
early-life outcomes.

\begin{table}
  \caption{Costs and Estimated Impacts of Selected Early-Life Programs}
  \label{tab:impactsOther}
  \begin{tabular}{lcccc} \toprule
    Reference & Estimated & Cost per    & Estimated      \\
    & Impact    & Participant & Cost per gram  \\ \midrule
    \multicolumn{5}{l}{\textbf{Supplemental Nutrition Program for Women, Infants and Children (WIC, US)}} \\
    \citet{RossinSlater2013} & 27.30 (7.98)               & \$405 USD & \$14.8 \\
    \citet{Hoynesetal2011}   & 28.75 (15.13)              & \$405 USD & \$14.1 \\
    %\citet{BitlerCurrie2005}   & 63.65 (6.19) & \$321 USD & \\
    &&&\\
    \multicolumn{5}{l}{\textbf{PANES (Uruguay)}} \\
    %\citet{Amaranteetal2016} & 30.83 (18.44)  & \$504 USD & \$16.3 \\
    \citet{Amaranteetal2016} & 30.83 (18.44)  & \$918 USD & \$29.8 \\
    %&&[\$918 PPP USD]&[\$29.8 PPP]\\
    &&&\\
    \multicolumn{5}{l}{\textbf{Supplemental Nutrition Assistance Program (FSP, US)}} \\
    \citet{Almondetal2011}   & 8.96 (5.05) & \$1125 USD & \$125.6 \\
                             & 20.27 (6.89) & \$1125 USD & \$55.5 \\
    &&&\\
%    \multicolumn{5}{l}{\textbf{PROGRESA/Oportunidades (Mexico)}} \\
%    \citet{BarberGertler2010}& 81.9 (54.2)   & \$279 USD & \$3.40  \\
    \multicolumn{5}{l}{\textbf{Earned Income Tax Credit (EITC, US)}} \\
    \citet{Strullyetal2010}& 15.70 (1.211)& \$1558 USD &  \$99.2\\
    \citet{Hoynesetal2015} & 9.95  (2.05)  & \$1558 USD & \$156.6  \\
    \multicolumn{5}{l}{\textbf{Chile Crece Contigo (Chile)}} \\
    Our estimates            & 10.09 (3.37)   & \$111 USD & \$11.0\\ \bottomrule
    \multicolumn{5}{p{15.0cm}}{{\footnotesize\textsc{Notes}: Estimates from
        \citet{Hoynesetal2015} refer to single women with no more than a
        high-school education (the ``high impact'' group, with highest
        eligibility for policy use). Two estimates are presented for
        \citet{Almondetal2011}, given that their results are presented by
        race.  The top line refers only to black mothers, while the bottom
        line refers only to white mothers.  Estimates for black mothers are
        based on the most recent estimates presented by the authors' in their
        Erratum.  All US program costs are expressed in US dollars, and
        non-US program costs (Chile and Uruguay) are denoted in PPP adjusted
        US dollars.  PPP adjusted costs are higher than non-PPP adjusted
        costs, so this results in a conservative estimates of costs per gram.
        Similar estimates and additional calculation details are presented in
        \citet{Clarkeetal2017} for the WIC and FSP.}}\\% Add citation for \citet{CurrieRajani2015}.}} \\
  \end{tabular}
\end{table}



The estimated impact of each alternative program is drawn from the articles
cited in the first column of Table \ref{tab:impactsOther}.  In most cases,
these are presented as a single estimate, although in the case of
\citet{Almondetal2011} estimates are presented separately for black and
white mothers, so we report each estimate.  Details on the cost per user
are also generally drawn from various sources.  In the case of the PANES
program in Uruguay, the cost per user is reported by
\citet{Amaranteetal2016} as \$102 per month in PPP adjusted terms.  In each
case we infer the cost of the program by assuming 9 months of coverage, as
this allows for consistent comparisons across programs.  In the case of the
WIC program, recent figures suggest that the cost of the program is quite
stable at around \$45 per month per participant \citep{WIC2017}, suggesting
a 9 month cost estimate of \$405 per participant.  Similar estimates for
the FSP suggest costs of approximately \$125 per month per participant, or
\$1125 over the course of 9 months \citep{FSP2017}. Finally, costs from the
EITC program are reported in \citet[Appendix Table A1]{Hoynesetal2015}.

These comparisons lead to a number of conclusions regarding the cost of
producing birth weight in public programs, and the relative efficiency
of different programs.  Firstly, perhaps unsurprisingly, programs which
explicitly target health at birth produce birth weight much more cheaply
than non-targeted programs.  The targeted programs (WIC and ChCC) range
from anywhere between 2--15 times cheaper per gram of birth weight
produced than non-targeted programs such as SNAP/FSP, the EITC or PANES
in Uruguay.  In general, it is likely reasonable to demand more of a
program which aims to increase child health, so the increased costs
among non-targeted programs should not be seen as a progam inefficiency.
Secondly, we note that ChCC produces birth weight in a comparatively
efficient way, even when compared to WIC in the US. Our back-of-the-%
envelope calculation of the cost of birth weight in Chile is
\$11 USD per gram (PPP-adjusted) in Chile, compared to estimates of
around \$14 USD per gram from the WIC program.  As discussed above,
this is then additionally more efficient than comparison non-targeted
programs both in US, and in Latin America.

\subsubsection{Internal Efficiency}
Finally, while this value benchmarks the efficiency of the ChCC program
compared to other early-life health programs, it provides less context on
the implications of these costs for social spending and development
outcomes within the country. In order to put these estimates in context,
we can ask how investments in birth weight can be expected to map to
\emph{returns} to birth weight in the country.  In Chile there are a number
of well-identified estimates of the value of birth weight to later-life
education, with significant and long-standing observed impacts
\citet{Bharadwajetal2013,Bharadwajetal2017}.  Using a within family
estimation (similar to the strategy proposed as a specification check in
\ref{eqn:panel}), \citet{Bharadwajetal2017} estimate that a 10\% increase
in weight at birth increases child test scores by approximately 0.05
standard deviations (for language and math), and that these returns are
quite stable between primary, secondary, and university entry exams. Using
our estimates discussed above, as well as data on birth weights in Chile,
we can thus back out the approximate amount required to be invested in ChCC
to produce an additional 0.05 standard deviations on educational outcomes.

From table \ref{tab:sumstats}, a 10\% increase in average birth weight is
334 grams.  Our calculation above suggests that the cost per gram of birth
weight produced by ChCC is \$11, implying that the cost of 334 grams is
approximately \$3700.  Thus, this rough calculation suggests that for every
\$3700 invested in the pre-natal components of the ChCC program, performance
on tests (compared to a static population) would increase by 5\% of a
standard deviation for the recipient child.  Stated in another way, given
that the cost per participant is estimated at \$111, the follow-on impact of
this investment during the child's life is an increase in 0.15\% of a standard
deviation.\footnote{This is calculated as $\$111/\$3700\times5\%=0.15\%$.}
What's more, these costs are clearly an upper bound,
as we ignore all other impacts of improvements in early-life health.  While
birth weight is a well known determiniant of educational attainment, birth
weight is also known to impact labour market outcomes
\citep{JohnsonSchoeni2011,CookFletcher2015,BehrmanRosenzweig2004,
  RosenzweigZhang2013,Caseetal2005}, the prevalence of chronic morbidities
\citep{Barker1995,AlmondMazumder2005,JohnsonSchoeni2011b}, mortality
\citep{vandenBergetal2006}, and a range of psychological outcomes
\citep{Fletcher2011}.

\subsection{Mechanisms}
\label{scn:Mechanisms}
Currently, our headline estimate of an average impact of 10 grams treats
ChCC receipt as a black box.  However, as discussed in section
\ref{sscn:ChCC}, ChCC includes a range of provisions and services, which
have been shown to work in other contexts.  For example, provision of food
to mothers during pregnancy has been shown to have large short- and
long-term effects in the US using data from the 1960s and `70s
\citep{Almondetal2011,Hoynesetal2016}. \citet{Doyle2017} documents
medium-term improvements in cognitive and socio-emotional development
of children in response to home visits to families and group education
classes.  In this section we consider four potential mechanisms of action
to explain the impacts of ChCC.  These are (i) A maternal nutrition component,
(ii) a prenatal care component, (iii) a home visit component, and (iv) a social
connection component capturing links to the wider social-safety net.  These
potential mechanisms envelope the majority of ChCC components, with the
exception of the pre-natal educational component for parents, which as
we discuss below is not included given problems with data availability.

In order to assess the importance of different components we require data
describing the usage of each component with variation ideally by month and
municipality.  Administrative data from The Ministry of Health of Chile
describe usage of various health services by month and by Health Service
for each month from 2001 onwards as part of their Summarised Monthly
Statistics (REM).  We thus collect in a consistent way all available indicators
related to prenatal use of health services for the period under study.  However,
it is important to note that this data not currently available at the municipal
level, but rather by Health Service, which generally encompass various
municipalities.  In Appendix Figure \ref{fig:healthServices} we show how
municipalities are
classified into Health Services, where each municipality is contained in only
one Health Service.  Using this data we are able to collect consistent reports
of the number of prenatal controls, the number of home visits to pregnant
mothers, the number of packages (kilos) of fortified milk dispursed to expectant
mothers, as well as the number of visits to Social Assistants at local health
clinics.   We thus cross our municipal level data with \emph{health service}
level controls, where each mechanism is consistently measured as the average
use of each component per pregnancy in the 9 months prior to each birth.
In Appendix Figure \ref{mech-plots} we display usage of components over
time, and in Appendix Table UVW provide summary statistics of avearge
use per pregnancy.

To examine the importance of different potential mechanisms, we augment
equation \ref{eqn:DD}, adding the vector of program usage variables to
the specification in the following way:
\begin{equation}
  \label{eqn:DDmech}
  birthweight_{cst} = \alpha^m_0 + \alpha^m_1 ChCC_{cst} + \bm{Mech_{st}\gamma} + \bm{W_{ct}\alpha}_{w} + \mu_t + \lambda_c + \eta_{cst}.
\end{equation}
Here we add a subscript $s$ to indicate health service given that mechanism
data is available at this level.\footnote{When a variable is collapsed at
  the level of municipality and health service, this results in identical
  levels and number of observations as when only collapsed at the level of
  municipality, given that each municipality is only found in one health
  service.  In 2008, a single health service split into two, meaning that
  for a small number of observations, we are unable to calculate lags for
  the mechanism variables.  The number of month$\times$municipal observations
  in the original regression are 31,805, however when including municipal
controls this health service split results in 31,760 observations.}
The vector of $\bm{Mech_{st}}$ controls are clearly ``bad controls''
\citep{AngristPischke2009} given that they are themselves outcomes of the
ChCC program. However, we include these controls as a mechanism test as
it allows us to examine the impact of ChCC on birth weight,
\emph{conditional} on a particular program component.  We include
different mechanism variables in a step-wise manner, and examine, conditional
on each mechanism, how $\widehat\alpha_1^m$ compares to the original
$\widehat\alpha_1$ estimate, allowing us to infer the proportion of the
ChCC effect explained by each particular mechanism.  As the order in
which we add the mechanisms in this process is arbitrary, %as a consistency check
we also calculate the \citet{Gelbach2016} decomposition (for each
outcome variable considered).  This decomposition allows us to consistently
apportion changes in the estimate of ChCC impact between $\widehat\alpha_1$
and $\widehat\alpha_1^m$ to each mechanism, by considering the impact of ChCC
on each mechanism, and the impact of each mechanism on a the outcome
variable of interest.

Table \ref{Mechs} displays estimates of unconditional ChCC impacts, and
the impact of ChCC conditional on the various proposed mechanisms.  The
baseline impact of 9.851 grams is marginally different (not statistically
distinguishable) from the 10.092 grams reported in Table \ref{mDD} given
the small number of observations without mechanism controls.  We
consistently compare conditional impacts with the 9.851 unconditional
estimate to maintain fixed the estimation sample.  Subsequent columns
introduce particular mechanisms one-by-one.  In column 2 we observe that
an additional pre-natal control is associated with a $\sim$ 6 gram increase
in birth weight.  Column 3 includes controls for the original and updated
formulation of fortified milk distributed to mothers (we provide full
details related to fortified milk, and full mechanism data, in Appendix
\ref{app:program}). In general we find quite inexact estimates of their
impacts on birth weight, potentially also reflecting the lack of data
availability at the finer municipal level.  Additional columns of home
vistis and social safety net components are similar imprecise, with the 
exception of enrolment of mothers in the Chile Solidario program, which
is associated with a large positive impact on birth weight (comprehensive
details and analysis of the Chile Solidario program is provided in
\citet{Carneiroetal2014}).

Most interesting for the present analysis are the changes in the estimates
of the impact of ChCC moving across columns.  The estimated impact of
9.85 grams in column 1 is reduced to 6.88 grams in the final column,
suggesting that the proposed mechanisms, even though measured noisily,
can explain 30\% of the full impact.  At the foot of the table we provide
two decompositions of this movements.  The first row (``Explained Effect'')
calculates the percent of the movement in the effect from one column to
another attributable to the particular mechanism.  Here we observe that
the mechanism which explains the largest proportion of the impact in this
setting is food suplementation (18.9\%), followed by increased link to the
social safety net (9.6\%), and then prenatal care and home visits.
The second row displaying the cumulative explained effect provides a
cumulative sum of the ability of proposed mechanisms to explain Chile
Crece Contigo's impact on birthweight, which reaches 30.1\% of the full
effect.

\input{tables/ChCC_Mechanism.tex}

It is important to note that this calculation is at best partial, as
there are components which are hard to measure or not observed in
publicly available data.  Indeed, even when controlling for the full
set of mechanisms, there is still 69.9\% of the impact which we are
unable to explain.  For example, as discussed above, we do not
observe group education usage over time, and measures like prenatal
controls are potentially significantly under-reporting the true changes
due to ChCC, given that prenatal controls approximately doubled in
length and included a number of new and standardised components.
Thus, measures of prenatal check-up coverage, while capturing ChCC's
impact on extensive margin impacts, does not capture intensive margin
impacts of additional time and additional components in a given check-up.
Finally, in Appendix Table \ref{GelbachMech} we present the alternative
decomposition suggested by \citet{Gelbach2016} which is based on the
regression in column 5 of Table \ref{Mechs}.  Here we present the
decomposition for each outcome measure in Table \ref{mDD}, and generally
find that food supplements and increased linkages to the social safety
net explain the largest proportion of (explainable) ChCC impacts on
health outcomes at birth.

\section{Conclusion}
We estimate the impact of a large early-life health and social inclusion
policy, \emph{Chile Crece Contigo}.  This policy, explicitly designed to
target differences in psychological, behavioural, and cognitive
development of children in vulnerable houesholds, is found to have
significant impacts on health at birth over a range of dimensions.  Using
municipal roll-out and variation in intensity of use of ChCC in a
difference-in-difference specification, we estimate that participation in
ChCC increased weight at birth by 10 grams on average.  We also find an
increase in gestational length, and some evidence to suggest that the
program increased the likelihood of fetal survival.  These results are
validated by a large (but not universal) sample of micro-data where within
mother variation in program exposure is used to estimate the policy's impact
on infants.

We find that this policy is both equity enhancing, as well as quite
efficient when compared with other policies world-wide, and as a manner
to invest in human capital accumulation.  The impacts are observed to
be largest amongst the most vulnerable groups, which are specifically
targeted to receive preferential transfers in the program. Combined with
the cost of running ChCC, our estimates suggest that the government of
Chile spends approximately \$11 per gram of birth weight---a figure that
is comparable and slightly less to other large neonatal health programs,
even when controlling for purchasing power.  What's more, given the well
known positive effects of birth weight on later life outcomes, we are able
to estimate that as an \emph{upper bound} cost, each \$3700 spent on ChCC
results in an additional 0.05 standard deviation of educational attainment
on later life test scores.

However, our estimates suggest that the program impacts are highest for
babies with health stocks at birth \emph{above} the median outcome.  We
observe that the mean program effect of 10 grams largely come from
shifting children who were born weighing between 3,500-4,000 grams, and
for increasing gestational length at full term (weeks 39 and 40). 
All told this paper suggests that despite challenges of targeting and
improving the health at birth of conceptions towards the bottom of the
health distribution, public investments in early life health in developing
and emerging economies can have considerable mean impacts when well targeted
and well designed, and that these impacts may propogate through the economy
long after birth and program implementation.  %In the case of ChCC, a
%stand-out and replicated policy in the Latin American region, 



%We examine the importance of a large early life social safety net program
%in a middle income country.  This program---Chile Crece Contigo---is one
%of the largest social programs in Chile, reaching more than 150,000 pregnant
%women each year, and accounting for nearly 1\% of the national budget.
%Using newly generated administrative data matching all births with a program
%participation indicator, as well as time and geographical variation in program
%roll-out, we are able to combine a number of estimation strategies leading
%to plausibly causal effects under varying assumptions.
%
%We document, firstly, that this program has considerable effects on neonatal
%health in Chile.  Depending on the specification examined, we estimate that
%the program participation increases birth weight between 12 and 13 grams,
%reduces the probability of being low birth weight by up to 10\% and reduces
%premature births by as much as 7\%.  What's more, it appears to eliminate
%the birth weight differential between the poorer program participants and
%the less-poor non-participants.  Results appear to agree quite well whether
%working with between-mother micro-level estimates, or difference-in-difference
%estimates based on program roll-out nation-wide.



\newpage
\bibliography{references}
\newpage

%\section*{Figures and Tables}

\clearpage
\setcounter{table}{0}
\renewcommand{\thetable}{A\arabic{table}}
\setcounter{figure}{0}
\renewcommand{\thefigure}{A\arabic{figure}}
\pagenumbering{arabic}
\renewcommand{\thepage}{A\arabic{page}}
\appendix
\section*{Appendices}

\section{Appendix Figures and Tables}
%\begin{figure}[htpb!]
%  \begin{center}
%    \centering
%    \caption{Timing of Program Rollout and Coverage}
%    \includegraphics[scale=1]{./figures/ChCCcover.eps}
%    \label{fig:coverage}
%  \end{center}
%  \vspace{-5mm}
%  \floatfoot{\textsc{Notes to figure \ref{fig:coverage}}: The first
%    program rollout occurred in June of 2007, with the remainder of
%    municipalities joining in the second rollout in January of 2008.
%    All coverage figures are based on ChCC reports \citep{Arrietetal2013}.
%    %The program continued after 2010 and is due to do so indefinitely,
%    %however we only consider the period up until 2010 given that our matched
%    %birth data end in this year.
%  }
%\end{figure}



\begin{figure}[htpb!]
  \begin{center}
    \caption{ChCC Usage in Post-Implementation Period}
    \label{fig:usage}
    \includegraphics[scale=1]{./figures/chccUsage.eps}
  \end{center}
  \floatfoot{\textsc{Notes to figure \ref{fig:usage}}: The density of ChCC usage
    by municipality over the entire post-treatment period is displayed.  Usage
    refers to the average proportion of all births in which ChCC components 
    were accessed by the mother during the gestational period.  Usage data comes
    from The Ministry of Social Development's administrative data on public program
    use, and is averaged at the level of each municipality.  Refer to Figure
    \ref{ChCCenrol} for additional details regarding municipal level usage of
    ChCC components.
  }
\end{figure}


\begin{figure}[htpb!]
  \begin{center}
    \centering
    \caption{Program Roll-out (early and late adopters)}
    \label{fig:map}
    \begin{subfigure}{.3\textwidth}
      \centering
      \includegraphics[scale=0.4]{../results/Rollout_Time.eps}
      \caption{Full Country}
      \label{Chile}
    \end{subfigure}%
    \begin{subfigure}{.7\textwidth}
      \centering
      \includegraphics[scale=0.84]{../results/Rollout_Time_RM.eps}
      \caption{Santiago Metropolitan Region}
      \label{RM}
    \end{subfigure}
  \end{center}
  \vspace{-5mm}
  \floatfoot{\textsc{Notes to figure \ref{fig:map}}: Chile consists
    of 346 municipalities (``\emph{comunas}'') which are the lowest
    geographic administrative level with their own political administration.
    ChCC roll-out started in June 2007, and reached 159 of the 346
    municipalities in 2007 (chosen due to the availability of
    infrastructure) and then was rolled out to the remaining municipalities
    during 2008.  Precise rollout dates are provided by the Ministry of Social
    Development of Chile.  The full country is displayed in the left-hand
    panel, and only the Metropolitan Region of Santiago (from the centre of
  the country) is displayed in the right-hand panel.}
\end{figure}

\begin{landscape}
\begin{figure}[htpb!]
  \begin{center}
    \caption{Municipal Characteristics and ChCC Enrolment}
    \label{ChCCenrol}
    \begin{subfigure}{.33\textwidth}
      \centering
      \includegraphics[scale=0.44]{../results/comunaLevel/chcc_aguapotable.eps}
      \caption{Treated Piped Drinking Water}
      \label{agua}
    \end{subfigure}%
    \begin{subfigure}{.33\textwidth}
      \centering
      \includegraphics[scale=0.44]{../results/comunaLevel/chcc_fonasaPC2064.eps}
      \caption{FONASA enrolments}
      \label{fonasa}
    \end{subfigure}%
    \begin{subfigure}{.33\textwidth}
      \centering
      \includegraphics[scale=0.44]{../results/comunaLevel/chcc_FPSanioPC.eps}
      \caption{Proportion of FPS per Year}
      \label{FPS}
    \end{subfigure}
    \begin{subfigure}{.33\textwidth}
      \centering
      \includegraphics[scale=0.44]{../results/comunaLevel/chcc_pobrezacasen.eps}
      \caption{Poverty}
      \label{pobreza}
    \end{subfigure}%
    \begin{subfigure}{.33\textwidth}
      \centering
      \includegraphics[scale=0.44]{../results/comunaLevel/chcc_subEscPC.eps}
      \caption{Education Subvention}
      \label{subEsc}
    \end{subfigure}%
    \begin{subfigure}{.33\textwidth}
      \centering
      \includegraphics[scale=0.44]{../results/comunaLevel/chcc_teen.eps}
      \caption{Proportion of Teen Births}
      \label{teen}
    \end{subfigure}
    \begin{subfigure}{.33\textwidth}
      \centering
      \includegraphics[scale=0.44]{../results/comunaLevel/chcc_votop.eps}
      \caption{Vote Share (Mayor)}
      \label{votop}
    \end{subfigure}%
    \begin{subfigure}{.33\textwidth}
      \centering
      \includegraphics[scale=0.44]{../results/comunaLevel/chcc_LCR.eps}
      \caption{Political Association}
      \label{LCR}
    \end{subfigure}%
    \begin{subfigure}{.33\textwidth}
      \centering
      \includegraphics[scale=0.44]{../results/comunaLevel/chcc_meduc.eps}
      \caption{Maternal Education}
      \label{teen}
    \end{subfigure}
  \end{center}
  \floatfoot{\textsc{Notes to figure \ref{ChCCenrol}}: Each panel presents the proportion
    of Chile Crece Contigo enrollees in each municipality after the introduction of the
    program along with municipal level averages in a range of other social or political
    variables.  In each case, ChCC enrolment is displayed on the horizontal axis, and
    alternative outcomes on the vertical axis.
  }
\end{figure}
\end{landscape}



\begin{landscape}
\input{../results/comunaDD-Available.tex}
\end{landscape}


\clearpage
\begin{table}[htpb!]
  \begin{center}
    \caption{Summary Statistics by Trimester: Birth and Chile Crece Contigo Data}
    \label{tab:sumstatsTri}
    \begin{tabular}{lccccc} \toprule
      & N& Mean & Std. Dev. & Min & Max \\ \midrule
      \input{../results/SummaryMunicipal-trimester.tex} \bottomrule
      \multicolumn{6}{p{15.2cm}}{{\footnotesize \textsc{Notes to Table
            \ref{tab:sumstatsTri}}: Summary Statistics are displayed for
          municipality by trimesterly averages for
          each trimester from January 2003 to December 2010.  Trimesters
          refer to January-March, April-June, July-September, and
          October-December.  For additional notes, refer to Table
          \ref{tab:sumstats}.
      }}
      \end{tabular}
  \end{center}
\end{table}

\begin{landscape}
\input{../results/comunaDD-trimester.tex}
\end{landscape}

\begin{table}[htpb!]
  \caption{Adjusting For Multiple Hypothesis Testing}
  \label{tab:MultHyp}
  \begin{center}
      \begin{tabular}{lcccccc} \toprule
        &  Index & \multicolumn{5}{c}{Original Variables} \\ \cmidrule(r){2-7}
        &Anderson & Birth & LBW & Birth & Weeks & Premature \\
        &Index    & Weight&     & Size  & Gestation & \\ \midrule
        \multicolumn{7}{l}{\textsc{Panel A: Municipal-Level Analysis}} \\
        $p$-value  (Original)    & \input{../results/MC_DD_porig.tex}
        $p$-value  (Corrected) & \input{../results/MC_DD_pRW.tex}
        &&&&&&\\
        \multicolumn{7}{l}{\textsc{Panel B: Individual-Level Analysis}} \\
        $p$-value  (Original)    & \input{../results/MC_FE_porig.tex}
        $p$-value  (Corrected) & \textbf{0.0479} & 0.0392 & 0.2549 & 0.0588 & 0.0196 & 0.7451\\
        \midrule
        \multicolumn{7}{p{14.8cm}}{{\footnotesize \textsc{Notes}: Corrected $p$-values based
            on original variables are calculated using the \citet{RomanoWolf2005} technique to
            control the Family Wise Error Rate of hypotesis tests. The \citet{Anderson2008}
            index converts the multiple dependent variables into a single dependent variable
            (index) giving more weight to variables which provide more independent variation.
            The specification of each regression follows Table \ref{mDD} (panel A), and
            \ref{mFE} (panel B).}}
        \\ \bottomrule
    \end{tabular}
  \end{center}
\end{table}

\begin{figure}[htpb!]
  \begin{center}
    \caption{Proportion of Births Attended in the Public Health System}
    \label{publicBirth}
    \includegraphics[scale=0.84]{../results/birthsPublic.eps}
  \end{center}
  \vspace{-6mm}
  \floatfoot{\textsc{Notes to figure \ref{publicBirth}}: Figures on
    the proportion of births in the public health system and all births
    nation-wide are provided monthly by the Department of Statistics and
    Health Information (DEIS) of the Ministry of Health of Chile. Monthly
    proportions are displayed for each month from January 2002 until
    December 2010.  The first vertical dotted line is the beginning of
    ChCC rollout, while the second vertical dotted line is when ChCC
    reached the full country.
  }
\end{figure}

\begin{table}
  \begin{center}
    \caption{IV Estimates Using Lagged ChCC Enrolment}
    \label{tab:ChCC_IV}
    \begin{tabular}{lcccccc} \toprule 
      & (1)    & (2) & (3)  & (4)       & (5)       & (6)         \\
      & Weight & LBW & Size & Gestation & Premature & Fetal Death \\ \midrule
      \multicolumn{7}{l}{\textbf{Second Stage Estimates}} \\
      \input{tables/comunaDD-IV.tex} \\
      \multicolumn{7}{l}{\textbf{First Stage Estimates}} \\
      \input{tables/comunaDD-first.tex}
      \midrule
      \multicolumn{7}{p{17.4cm}}{{\footnotesize \textsc{Notes}:
          Difference-in-difference estimates are presented following the
          results of Table \ref{mDD}.  However, here the Proportion of ChCC
          Coverage among births in a given month is instrumented with
          lagged ChCC coverage from the same municipality.  The
          IV results along with standard errors clustered by municipality
          are displayed in the top panel of the Table.  The second panel
          documents the first stage results of regression ChCC coverage
          on its lagged value.  The associated first stage F-statstic and its
          p-value are documented at the foot of the table.
      }} \\ \bottomrule
    \end{tabular}
  \end{center}
\end{table}

\clearpage
\begin{table}[htpb!]
  \caption{Correction for Multiple Hypothesis Testing in Distributional Estimates}
  \label{tab:RWdist}
\begin{tabular}{lcclcc} \toprule
  \multicolumn{3}{c}{Birth Weight}&\multicolumn{3}{c}{Gestation} \\ \cmidrule(r){1-3} \cmidrule(r){4-6}
  Cut-off & Original & Romano Wolf &   Cut-off & Original & Romano Wolf \\
  & $p$-value & $p$-value  &           & $p$-value& $p$-value \\ \midrule
  1000 &0.4592 & 0.6707 & 30 & 0.6905 & 0.6587 \\
  1250 &0.5786 & 0.7206 & 31 & 0.6245 & 0.6587 \\
  1500 &0.7191 & 0.8383 &           32 & 0.3666 & 0.4850 \\
  1750 &0.0632 & 0.0619 &           33 & 0.0464 & 0.0439 \\
  2000 &0.0014 & 0.0000 &           34 & 0.1695 & 0.2535 \\
  2250 &0.0135 & 0.0020 &           35 & 0.0804 & 0.0818 \\
  2500 &0.0737 & 0.0838 &           36 & 0.0539 & 0.0559 \\
  2750 &0.2736 & 0.4371 &           37 & 0.2337 & 0.3413 \\
  3000 &0.1169 & 0.1397 &           38 & 0.2651 & 0.3513 \\
  3250 &0.2212 & 0.3373 &           39 & 0.0477 & 0.0439 \\
  3500 &0.0056 & 0.0000 &           40 & 0.0005 & 0.0000 \\
  3750 &0.0030 & 0.0000 &           41 & 0.5312 & 0.6587 \\
  4000 &0.0221 & 0.0100 &    && \\
  4250 &0.0167 & 0.0040 &    && \\
  4500 &0.0144 & 0.0020 &    && \\
  4750 &0.9501 & 0.9421 &    && \\
  5000 &0.4313. & 0.6707&    && \\ \midrule
  \multicolumn{6}{p{12cm}}{{\footnotesize \textsc{Notes to Table \ref{tab:RWdist}}: Un-adjusted and
      multiple-hypothesis test adjusted $p$-values are displayed corresponding to the estimates and
      standard errors displayed in Figure \ref{quintiles}.  Unadjusted $p$-values refer to the
      $p$-value on ChCC in each regression where the outcome variable is birth weight or gestation
      exceeding the listed cut-off.  Romano Wolf adjusted $p$-values are based on a null
      resampled distribution as described in \citet{RomanoWolf2005}.  We re-sample using 500
      bootstrap samples.}} \\ \bottomrule
\end{tabular}
\end{table}


%828.000& 1.365.541& 1.450.205& 1.979.458
%103.500&539.235& 572.668& 581.258
%70.310& 625.200& 1.498.812&1.636.891

\begin{table}[htpb!]
  \caption{Costs of ChCC Per Participant in Gestational Program}
  \label{tab:spending}
  \begin{tabular}{lcccc} \toprule
    &2007&2008&2009&2010 \\ \midrule
    \multicolumn{5}{l}{\textsc{Panel A: All Amounts in 1000s of Chilean Pesos}} \\
    Costs Associated with PADBP&1,969,162&6,116,663&14,231,107&14,444,574\\
    Costs Ministry of Planning &1,001,810&2,529,976&2,604,131&4,197,607\\
    Massive Education Program &20,000&195,640&261,462&196,624\\
    \textbf{Total Prenatal Development Components} &2,990,972&8,842,279&17,096,700&18,838,805\\
    Total Budget (ChCC)&67,903,331&126,446,362&159,660,473&214,505,550\\
    Total Budget/1000 (All Chile)&17,883,154& 20,650,579 & 23,406,879& 25,651,970\\
    Total Women Participating during Gestation &47,683&166,900&171,811&171,799\\
    Proportion of all Participants in Pre-natal Care &1&0.449&0.307&0.303\\
    Cost per Pre-Natal Participant &62,726 & 24,714 & 30,549 & 33,116\\
    %Cost per Pre-Natal Participant &62,726 & 52,980 & 99,509 & 109,656\\
    %\midrule
    &&&&\\
    \multicolumn{5}{l}{\textsc{Panel B: All Amounts in US Dollars}} \\
    Costs Associated with PADBP                    &3,702,025  &12,288,376 &22,257,451&28,470,255\\
    Costs Ministry of Planning                     &1,883,403  &5,082,722  &4,072,861 &8,273,483\\
    Massive Education Program                      &37,600     &393,041    &408,917   &387,546\\
    \textbf{Total Prenatal Development Components} &5,623,027  &17,764,139 &26,739,239&37,131,285\\
    Total Budget (ChCC)                            &127,658,262&254,030,741&249,708,980&422,790,439\\
    Total Budget/1000 (All Chile)                  &33,620,330 &41,487,013 & 36,608,359& 50,560,033\\
    Total Women Participating during Gestation     &47,683     &166,900&171,811&171,799\\
    Proportion of all Participants in Pre-natal Care &1&0.449&0.307&0.303\\
    Cost per Pre-Natal Participant &\$118 & \$50 & \$48 & \$65\\
    Cost per Pre-Natal Participant (PPP Adjusted) &\$192 & \$72 & \$87 & \$93\\
    \midrule
    \multicolumn{5}{p{18.1cm}}{{\footnotesize \textsc{Notes to Table \ref{tab:spending}:}
        Costs per pre-natal participant are calculated by dividing the pro-rata total costs
        of prenatal development components by the total number of participants in the pre-natal
        period.  Total prenatal development components are calculated as the sum of the costs
        of the PADBP program, fixed costs assigned to the Ministry of Planning, and the costs
        of the Massive Education program.  Costs are assigned pro-rata to pre-natal versus non
        pre-natal components using the proportion of all participants which are in the pre-natal
        period, rather than during years 1-5.  In the first year, the program only began in
        utero, so all costs are assigned to pre-natal development.  Budget details are all
        compiled from the ChCC final reports \citep{Arrietetal2013}, and historic budget
        laws (for example \citet{Presupuesto2007}).  Total participants during gestation as
        well as in the post-natal period are compiled from the Department of Health Statistics
        and Information from the Ministry of Health. PPP-adjusted costs are based on the
        World Bank's PPP conversion factor.}} \\ \bottomrule
  \end{tabular}
\end{table}
%166.900/(166.900+204.375)
%171.811/(171.811+388.336)
%171.799/(171.799+395.431)
%%Double costs
%166.900*2/(166.900*2+204.375)=0.620
%171.811*2/(171.811*2+388.336)=0.469
%171.799*2/(171.799*2+395.431)=0.465

%Exchange rate:
% 2007 - 1CLP = 0.001880 USD
% 2008 -        0.002009
% 2009 -        0.001564
% 2010 -        0.001971
% PPP conversion factor
% 2006 - 322.174
% 2007 - 326.154
% 2008 - 342.988
% 2009 - 353.163
% 2010 - 357.464

%\begin{table}[htpb!]
%  \caption{Spending on ChCC as a Portion of National Spending}
%  
%  \begin{tabular}{llcc} \toprule
%    Year & Spending & Spending & Percent \\
%    & (ChCC)        & (National) & \\ \midrule
%    2007 & 67,903,331  & & 0.380 \\
%    2008 & 126,446,362  & 0.612 \\
%    2009 & 159,660,473  & 0.689 \\
%    2010 & 214,505,550  & 0.836 \\
%    \midrule
%
%  \end{tabular}
%\end{table}
%

\begin{landscape}
  \input{../results/ChCC_Inputs.tex}
\end{landscape}


\begin{figure}[htpb!]
  \begin{center}
    \caption{ChCC rollout and Pregnancy Inputs Dispursed}
    \label{mech-plots}
    \begin{subfigure}{.5\textwidth}
      \centering
      \includegraphics[scale=0.6]{../results/ChCCmechanism_controlesPrenatales.eps}
      \caption{Prenatal Controls}
      \label{mech-prenatal}
    \end{subfigure}%
    \begin{subfigure}{.5\textwidth}
      \centering
      \includegraphics[scale=0.6]{../results/ChCCmechanism_homeVisits.eps}
      \caption{Home Visits}
      \label{mech-visits}
    \end{subfigure}
    \begin{subfigure}{.5\textwidth}
      \centering
      \includegraphics[scale=0.6]{../results/ChCCmechanism_puritaFortificada.eps}
      \caption{Fortified Milk (Original Formula)}
      \label{mech-purita}
    \end{subfigure}%
    \begin{subfigure}{.5\textwidth}
      \centering
      \includegraphics[scale=0.6]{../results/ChCCmechanism_puritaMama.eps}
      \caption{Fortified Milk (Updated Formula)}
      \label{mech-mama}
    \end{subfigure}
    \begin{subfigure}{.5\textwidth}
      \centering
      \includegraphics[scale=0.6]{../results/ChCCmechanism_asistenciaSocial.eps}
      \caption{Social Assistance Appointments}
      \label{mech-social}
    \end{subfigure}%
    \begin{subfigure}{.5\textwidth}
      \centering
      \includegraphics[scale=0.6]{../results/ChCCmechanism_ChileSolidario.eps}
      \caption{Chile Solidario}
      \label{mech-CS}
    \end{subfigure}
  \end{center}
  \floatfoot{\textsc{Notes to figure \ref{mech-plots}}: Solid blue line
    displays the rollout of ChCC and proportion of coverage of births
    as in Figure \ref{fig:coverage}.  Dotted red lines display the total
    units of various components of the program dispursed over time in
    whole of Chile.  Each panel with the exception of Chile Solidario
    coverage in panel \ref{mech-CS} presents the number of units divided by
    1,000.  Additional discussion of variables and their measurement is
    provided in section \ref{scn:Mechanisms}.
  }
\end{figure}

\begin{figure}[htpb!]
  \begin{center}
    \caption{Health Services and Municipalities}
    \label{fig:healthServices}
    \includegraphics[scale=0.76]{figures/Servicio_Salud.eps}
  \end{center}
  \vspace{-6mm}
  \floatfoot{\textsc{Notes to figure \ref{fig:healthServices}}: Municipalities
    are indicated by municipal boundaries, and health services are indicated
    by colours.  Each of Chile's 346 municipalities belongs to one of 29 Health
    Services.  The entire country is displayed at right, and the densely populated
    Metropolitan Region of Santiago is displayed at left.
  }
\end{figure}

\input{../results/GelbachMechanism.tex}

\clearpage
\setcounter{table}{0}
\renewcommand{\thetable}{B\arabic{table}}
\setcounter{figure}{0}
\renewcommand{\thefigure}{B\arabic{figure}}
\section{Additional Program Details and Component Data}
\label{app:program}
\paragraph{Additional Program Details}
The full Chile Crece Contigo program covers children from before
birth (officially from the first planned gestational control
at week 14 of pregnancy) until early childhood.  Initially, with
the design and rollout of the program in 2007, the program
ended at age 4, once children enter the first transition level to
primary school.\footnote{In Chile pre-primary education ends with
  the first and second levels of transition (or pre-kinder and
  kindergarten), which begin at ages 4 and 5 respectively.  At age
  6, children begin grade 1 of primary school.}  More recent
extensions mean the program now follows children up until the
age of 8, with mental health treatment for children with mental
health disorders aged between 5 and 8.

The original program designed for children aged up to 4 years
consisted of 5 components and various sub-components.  We lay
these out below in Table \ref{ascn:benefits}.  Component
1, which is targeted to pregnant mothers, is the only component
which can potentially impact birth outcomes as the remainder of
the components are entirely delivered in the birth to 4 year
period of life.  The components below are universal, with the
exception of component 1B and component 5, which are preferential
components received by families flagged as being among the
60\% most vulnerable based on a social protection score.

Each particular program item described in table \ref{ascn:benefits}
consists of one or a series of check-ups, goods or other
services.  Each item also comes with a clear definition of
how to deliver the item to the objective population, and key
targets for public service workers.  For example, item 1A, part
i (pre-natal check-ups) specifies that 7 prenatal check-ups
should be targeted in low risk cases, and that the duration of
these check-ups is 40 minutes.  Particular check-ups also have
their own requirements, such as specific diagnostic tests, such
as the abbreviated psycho-social evaluation during the first and
third trimester.

In this appendix we provide only a short summary of each
component in Table \ref{ascn:benefits}. Full details regarding
each component are available in the ChCC
guide to services \citep{MDS2014}.  Specific components
targeted to vulnerable families consist of the generation of
a personalised plan identifying availability of differential
services, home visits lasting 1 hour (which are 
targeted to families with specific risk-factors), information
related to other subsidies and local programs, and 
contact with local healthcare and social profesionals.  Additionally,
all children in vulnerable families are guaranteed access to
extended nursery and pre-school programs at no cost.

\definecolor{dccol}{rgb}{0.97, 0.9, 1.0}

\begin{landscape}
\begin{table}[htpb!]
  \caption{List of ChCC Policy Components and Phases}
  \label{ascn:benefits}
  \begin{tabular}{p{4.5cm}p{5cm}p{10cm}p{2.5cm}}
    \toprule
    Component Name & Subcomponent Name & Program Item & Time-Period\\ \midrule
\cellcolor{dccol}&  \cellcolor{dccol} &\cellcolor{dccol}i) Prenatal check-ups, establishment of link and detection of psychosocial risk factors  &  \cellcolor{dccol} \\
\cellcolor{dccol} & \multirow{-3}{5cm}{\cellcolor{dccol}A. Strengthening of Prenatal Care}        &ii) Receipt of gestational reading guides & \cellcolor{dccol}\\
\cellcolor{dccol} &&\cellcolor{dccol}i) Design of individual health plan for pregnant mothers and families in psycho-social vulnerability&\cellcolor{dccol}\\
\cellcolor{dccol} &\multirow{-2}{5cm}{\textbf{*}B. Integral Support for families in Psycho-Social Vulnerability} &ii) Integral home visits for pregnant mothers in vulnerable situations &\cellcolor{dccol} \\
\cellcolor{dccol} & &\cellcolor{dccol}iii) Links with municipal ChCC Network in cases of vulnerability &\cellcolor{dccol}\\
\multirow{-6}{4.5cm}{\cellcolor{dccol}1. Strengthening of Prenatal Development}                                                        &\cellcolor{dccol} C. Education for the Pregnant women and her partner or companion&i) Group or indiviudal education for pregnant women and partner/companion.  Cognitive and emotional support for birth and child-rearing&\multirow{-6}{2.5cm}{\cellcolor{dccol}Weeks 14-40 Gestation}\\ \midrule %%%%%%%%%%%%%%%%%%%%%%%%%%%%%%%%%%%%%%%%%%%%%%%%%%%%%%%%%%%%%%%%%%%%%%%%%%%%%%%%%
&\multirow{1}{5cm}{A. Personalised care during childbirth}&\cellcolor{dccol}i) Integral care prior and during childbirth  & \\
&\cellcolor{dccol}    &i) Personalised integral support for the postpartum mother and infant & \\
&\cellcolor{dccol}&\cellcolor{dccol}ii) Personalised cross-check of families bio-psycho social development&\\
&\multirow{-5}{5cm}{\cellcolor{dccol}B. Integral Care during the Postpartum period} &iii) Timely coordination with the primary health team & \\
& &i) \cellcolor{dccol}Education regarding the use of the PARN implements and early-life child-rearing  &\\
\multirow{-6}{4.5cm}{2. Personalised Care During the Birth Process}              &\multirow{-2}{5cm}{C. Newborn Support Program (PARN)}&ii) Handout of basic implement set and educational material &\multirow{-6}{2.5cm}{At Birth}\\ \midrule
\cellcolor{dccol}&\vspace{2mm}\cellcolor{dccol}A. Integral support for newborns in neonatal care &\cellcolor{dccol} i) Integral evaluation; Developmental care plan; integration with families; hospitals open to families; prevention of neuro-developmental deficit; education and psycho-social interventions&\cellcolor{dccol} \\
\multirow{-2}{4.5cm}{\cellcolor{dccol}3. Integral Developmental Support for hospitalized children}&\vspace{2mm}B. Integral support for children in pediatric care&ii) Integral evaluation; Developmental care plan; Provision of space for education and play; Use of stimulation protocol; Helpful relationships built between  health team and father/mother/carer &\multirow{-2}{2.5cm}{\cellcolor{dccol} 0-4 Years}\\ \midrule
&&& Continued...
  \end{tabular}
\end{table}
\end{landscape}

\begin{landscape}
\begin{table}[htpb!]
  %\caption{List of ChCC Policy Components and Phases}
  %\label{ascn:benefits}
  \begin{tabular}{p{4.5cm}p{5cm}p{10cm}p{2.5cm}}
    \toprule
    Component Name & Subcomponent Name & Contents & Time-Period\\ \midrule
&\cellcolor{dccol}&\cellcolor{dccol}i) Prenatal controls, establishment of link and detection of psychosocial risk factors  & \\
&   \cellcolor{dccol}         &ii) Participation in Child Health checkups (``Ni\~no/a sano'') & \\
&  \multirow{-3.8}{5cm}{\cellcolor{dccol}A. Strengthening Child Health Checkups for Integral Development} &\cellcolor{dccol}iii) Check-ups with evaluation and follow-ups & \\
    \multirow{-4}{4.5cm}{4. Strengthening Integral Development of Children} &B. Educational Interventions to support child-rearing&i) Group or individual education for development of parenting tools, ``Nobody is Perfect'' workshops&\multirow{-4}{2.5cm}{0-4 Years}\\ \midrule %%%%%%%%%%%%%%%%%%%%%%%%%%%%%%%%%%%%%%%%%%%%%%%%%%%%%%%%%%%%%%%%%%%%%%%%%%%%%%%%%
\cellcolor{dccol}&\cellcolor{dccol} &\cellcolor{dccol} i) Health support for children who are vulnerable, or developmentally delayed in integral components&\cellcolor{dccol} \\
\cellcolor{dccol}&\cellcolor{dccol}&ii) Health support for children with developmental deficit in integral components&\cellcolor{dccol}\\
\cellcolor{dccol}&\cellcolor{dccol}&\cellcolor{dccol} iii)Integral home visits for families of children under 4 in vulnerable situations for their bio-psycho-social development&\cellcolor{dccol}\\
\multirow{-6}{4.5cm}{\cellcolor{dccol}\textbf{*}5. Support for Children in Vulnerable Situations}&\multirow{-6}{5cm}{\cellcolor{dccol}A. Strengthening of interventions for children in vulnerable situations, or developmentally delayed}&iv) Support module for infant development in health centres &\multirow{-2}{2.5cm}{\cellcolor{dccol} 0-4 Years}\\ \midrule
\multicolumn{4}{p{22cm}}{Notes: Components and sub-components are based on official Chile Crece Contigo guide to services \citep{MDS2014}. Components or sub-components indicated with ``\textbf{*}'' are targeted compoents received only by means-tested groups.} \\
\bottomrule

  \end{tabular}
\end{table}
\end{landscape}

\paragraph{Data on Program Component Coverage}
The examination of program mechanisms of action in section
\ref{scn:Mechanisms} relies on data recording program components,
and their coverage over time.  As laid out in the paper, we
collect this data from public monthly administrative health
statistics data.  In each case we calculate the average
level of component use for each birth in the 9 months prior to
birth.  Averages are always calculated at the health service
and monthly level.  In a number of cases, we linearly
extrapolate coverage by month \emph{prior to 2005} only, 
given that data is not always available in 2003 and 2004.  This
period is entirely in the pre-program period, and time fixed
effects also capture periods in which linear extrapolation
is performed.

Fortified milk dispursed to pregnant women as part of the program
was originally called ``Leche Purita Fortificada'' (\emph{Purita}
Fortified Milk).  In 2008 this underwent a modification to better
meet the dietary requirements of pregnant women, and was renamed to
``Purita Mam\'a''.  Purita Mam\'a thus replaced Leche Purita
Foritificada, although a very small number of batches of the original
formula was still dispursed post 2008.  In Table \ref{lechepurita}
we show the change in composition between the two types of dietary
supplementations.  The guidelines issued by the Ministry of Health
provide a clear description of how this milk should be dispursed
to pregnant women.  For those who begin pregnancy with normal weight,
are overweight, or are obese, 1 kilogram of milk powder is given
per month.  For those women who begin pregnancy with an underweight
diagnostic, 3kg of milk powder is delivered per month \citep{GobChile2008}.

Measures of home visits refer to ``Integral Home Visits'' to
pregnant women.  Government reports highlight that Chile Crece
Contigo has increased the frequency of home visits to pregnant mothers
by around 500\% .  These home visits are targeted particularly
to families identified as in ``psycho-social risk'', which implies
meeting the vulnerability cut-off, and also presenting a number
of additional risk factors.  Given that the demand for home visits
varies considerably by income level of municipalities, the precise
decision of which families to visit is made by municipal health
centres, where visits should be targeted to families with the largest
number of risk factors.  A complete discussion of the goals and
recommendations for social workers completing home visits is provided
in \citet{GobChile2009}.

Remaining components such as prenatal check-ups and appointments
with social assistants in local health centres are also reported
in monthly health usage data.  In this case the number of
appointments completed are reported, and in Section \ref{scn:Mechanisms}
we calculate the average number of appointments per health service
for a pregnancy in the 9 months prior to the birth.

\begin{table}[htpb]
  \caption{Changes in Comoposition of Complementary Nutrition Component}
  \label{lechepurita}
  \begin{tabular}{lcll} \toprule
    Micronutrient & Units/Portion & Purita Mam\'a&Purita Fortificada \\ \midrule
    Vitamin A & $\mu$g &120&50\\
    Vitamin C & mg     &35&14\\
    Vitamin D & $\mu$g &1&0.6\\
    Vitamin E & mg     &7.5&0.1\\
    Vitamin B$_1$ & mg &0.4&0.06\\
    Vitamin B$_2$ & mg &0.4&0.24\\
    Niacin & mg        &4&0.12\\
    Vitamin B$_6$ & mg &0.5&0.06\\
    Folate  &  $\mu$g  &130 &7.34\\
    Vitamin B$_{12}$   & $\mu$g&1.3&0.64\\
    Vitamin B$_{5}$    & mg& --&0.46\\
    Calcium  &mg       &325&182.4\\
    Iron  &mg          &--&2.0\\
    Phosphourous  &mg  &291.5&155.2\\
    Magnesium &mg      &62.5&15.0\\
    Zinc &mg           &1.9&1.0\\ 
    Copper &mg         &--&0.08\\ \bottomrule
    \multicolumn{4}{p{11cm}}{{\footnotesize Notes: All values come from Technical
        Guidelines for Leche Purita Fortificada (old formula) and Leche Purita Mam\'a
        (new formula).  Each are described in terms of quantity of nutrients per
        recommendad portion.}}
  \end{tabular}
\end{table}


\setcounter{table}{0}
\renewcommand{\thetable}{C\arabic{table}}
\setcounter{figure}{0}
\renewcommand{\thefigure}{C\arabic{figure}}
\section{Maternal Fixed Effects}
\label{MFE}
As a consistency check of the difference-in-difference results  
reported in the paper, we also undertake an analysis using the
full matched micro-data observing each mother's participation
status in ChCC.  Identification is driven by variation within
mother's exposure to the program over time.  We estimate the
following mother FE specification:
\begin{equation}
  \label{eqn:panel}
  Infant Health_{ijt} = \beta_0 + \beta_1 ChCC_{jt} + \bm{X_{ijt}\beta}_{x} + \phi_t + \mu_j + \varepsilon_{ijt}
\end{equation}
where $InfantHealth$ refers to the same measures of health at
birth as discussed in the body of the paper of child $i$ born
to mother $j$ at time $t$.

The matched administrative data allows us to construct a panel
of mothers and their children, and the independent variable of
interest in \ref{eqn:panel} is $ChCC_{jt}$.  This measures for
each mother at time $t$ whether she participated in Chile Crece
Contigo, and under typical (fixed effect) panel assumptions,
$\beta_1$ identifies the effect of participation on infant health.
We include maternal fixed effects $\mu_j$ and year fixed effects
$\phi_t$, as well as a series of time-varying controls for
mothers including birth order dummies, mother's age at birth
dummies, and child year of birth dummies\footnote{We are also
  able to control for a number of other individual-level
  covariates including maternal education, however in our main
  specification do not propose include this control given that
  ChCC explicitly aims to ensure that young mothers who are still
  enrolled in education finish their studies, and hence education
  is likely a bad control.  In supplementary analyses we augment
  the controls in \ref{eqn:panel} to examine the robustness of
  findings to alternative specifications.}.  Identification takes
advantage of the fact that there are mothers who (a) participated
in ChCC and had births both before and after the introduction of
the policy, and (b) never participated in the policy and also had
births both before and after the policy's introduction.

The matched mother and child data does not include the entire
universe of births (we do use the entire universe of births in
municipal-level regressions presented in the paper).  As such,
any estimated program impacts in the micro-level mother FE
specification are at best suggestive of the average effects in
the population.  When matching vital statistics data with
parental social program use data, approximately 50\% of births
were matched with fathers, rather than mothers, and in these
cases we do not observe the mother's ChCC participation status.
We thus restrict the analysis with mother FE only to the
population of children matched with mothers, noting that it is
not a representative sample, and as such not directly comparable
to the municipal-level difference-in-difference regressions
presented in the paper based on the entire universe of births.
Nevertheless, it acts as a useful robustness check of the impact
of ChCC based on different identifying assumptions.\footnote{The
  two proposed strategies (the DD estimates in the body of the
  paper and the mother FE estimates in Appendices) are based on
  different identifying assumptions: strict (conditional)
  exogeneity for the family panel specification in equation
  \ref{eqn:panel} and parallel trends for the DD specification
  in equation \ref{eqn:DD}.}
%and indeed, estimated effects should be diluted
%considerably at the municipal-level given that only a subset of
%a municipality enrolls in ChCC. However we can provide a
%rough comparison and consistency check of the effects if we inflate
%$\alpha_1$ to account for partial enrollment at the municipal level.
%We propose to conduct this comparison, following the methodology
%described in \citet{Almondetal2011} who conduct a similar adjustment
%in examining the roll-out of the Food Stamp Program at the municipal
%level in the United States.
%
%\begin{figure}[htpb!]
%  \begin{center}
%    \centering
%    \caption{Program Roll-out (early and late adopters)}
%    \includegraphics[scale=1]{./../results/ChCCTrend.eps}
%    \label{fig:trends}
%  \end{center}
%  \vspace{-5mm}
%  \floatfoot{\textsc{Notes to figure \ref{fig:trends}}: Yearly averages are taken
%    for all mothers who have ever participated in Chile Crece Contigo (although
%    pre-2007 the program was not yet implemented) and all mothers who have never
%    participated.  The vertical solid line indicates the beginning of the ChCC
%    program.}
%\end{figure}

\begin{table}[htpb!]
  \begin{center}
    \caption{Summary Statistics: Matched Mother, Child and Social Security Data}
    \label{tab:sumstatsMother}
    \begin{tabular}{lccccc} \toprule
      & N& Mean & Std. Dev. & Min & Max \\ \midrule
      \multicolumn{6}{l}{\textbf{Panel A: All Mothers}} \\
      \input{../results/SummaryMotherAll-update.tex}
      \multicolumn{6}{l}{\textbf{Panel B: Matched Mothers and Children}} \\
      \input{../results/SummaryMother-update.tex}
      \multicolumn{6}{l}{\textbf{Panel C: Unmatched Mothers and Children}} \\
      \input{../results/SummaryMotherNoChCC-update.tex}
      \bottomrule
      \multicolumn{6}{p{15.4cm}}{{\footnotesize \textsc{Notes to Table \ref{tab:sumstatsMother}}:
          Summary statistics are presented for all births matched with the mother's participation
          in social programs. Summary statistics are presented for all years from 2003-2010.
          \emph{Chile Crece Contigo} began in June of 2007, and so any mothers having all births
          prior to this date never participated in ChCC.  For additional notes on variable definitions
          and comparison with the full universe of births (collapsed by municipality) refer to Table
      \ref{tab:sumstats}.}}
      \end{tabular}
  \end{center}
\end{table}


In Table \ref{tab:sumstatsMother} we present summary statistics of
births to all mothers, births to mothers who were matched with their
social program usage, and births to mothers who were not matched
their own social program usage data.  While their observable measures
are largely similar, matched mothers appear to be slightly younger
(26.91 versus 27.19 years), and have births with slightly better health
indicators (3,333 grams of birthweight versus 3,324 on average).

We present regression results using maternal fixed effects in Table
\ref{mFE}.  In this case identification is driven by mothers who have
had more than one birth, and hence variation in program coverage.
Despite the alternative methodology (and estimation sample) we observe
results that are qualitatively similar to those reported using the
municipal roll-out to estimate program impacts. In this case we observe
a larger impact on birth weight (19 grams, versus 10 grams), and
significant impacts also when considering size at birth of each child.
One result does not agree across specifications, and this is the
estimate on the impact of ChCC on low birth weight children.
In this specification we observe a weakly positive impact, while
in the specification reported in Table \ref{mDD} we observed a
weakly negative impact.  However, in Table \ref{mFEc} when we additionally
include full time and municipal fixed effects, we observe that the
result is no longer statistically distinguishable from zero, while
remaining effects are largely unchanged.  In panel B of Appendix Table
\ref{tab:RWdist} we present $p$-values on the impact of ChCC when
correcting for multiple hypothesis testing.  For birth weight, birth
size, and gestational length we observe that results remain statistically
distinguishable from zero when controlling for the family wise error
rate using \citeauthor{RomanoWolf2005}'s step-down correction.

\input{../results/motherFE.tex}

\input{../results/motherFE-controls.tex}

Finally, we briefly examine distributional impacts of the program
on health at birth, as examined in Figure \ref{quintiles}.  In this
case we simply examine descriptive evidence, considering the
distribution of birth weight between program recipients and
non-program recipients prior and posterior to the program's
implementation.  These are presented in Figure \ref{dists}, and
we observe that in the pre-program period, the distribution of birth
weight for recipient mothers is slightly below the corresponding
distribution for non-recipient mothers, while post-program the
reverse pattern is observed (both differences are observed in
the rejection Kolmogorov-Smirnov of tests of the equality of
distributions).  Interestingly, the distribution appears to be
most shifted from around 2500-4500 grams, providing some descriptive
support of the distributional results documented in Figure
\ref{quintiles}.


\begin{figure}[htpb!]
  \begin{center}
    \caption{Birthweight Distributions Pre- and Post-Program Implementation}
    \label{dists}
    \begin{subfigure}{.5\textwidth}
      \centering
      \includegraphics[scale=0.54]{../results/Density_weightPre.eps}
      \caption{Birthweights Pre-ChCC}
      \label{predists}
    \end{subfigure}%
    \begin{subfigure}{.5\textwidth}
      \centering
      \includegraphics[scale=0.54]{../results/Density_weightPost.eps}
      \caption{Birthweights Post-ChCC}
      \label{postdists}
    \end{subfigure} \vspace{-9mm}
  \end{center}
  \floatfoot{\textsc{Notes to figure \ref{dists}}: Densities are plotted using
    an Epanechnikov kernel with a bandwidth of 5 grams.  Each panel separates
    distributions by whether the mother \emph{ever} participates in Chile Crece
    Contigo.  Panel (a) displays only pre-ChCC time periods, while panel (b)
    displays only post-ChCC time periods.  In both cases, Kolmogorov-Smirnov
    tests reject equality of distributions (in different directions).
  }
  \end{figure}


\clearpage


%\section{Data Agreement}
%\label{app:agreement}
%We provide the full confidentiality agreement between the principal investigator
%of this project (Clarke) and the Ministry of Social Development (previously known
%as Ministry of Planification) and the Ministry of Health.  This agreement is
%displayed in full overleaf.
%\\
%%
%\includepdf[pages=-,pagecommand={},width=\textwidth]{confidencialidad.pdf}

\clearpage
\setcounter{table}{0}
\renewcommand{\thetable}{D\arabic{table}}
\setcounter{figure}{0}
\renewcommand{\thefigure}{D\arabic{figure}}
\section{Broader Context on Birth Outcomes and Maternal Characteristics in Chile}
\label{app:context}
Following the return to democratic rule in 1990, full microdata on
all births in Chile has been available from the Ministry of Health's
Department of Statistics and Health Information (DEIS).  These
vital statistics include each child's birth weight, weeks of gestation,
and a number of characteristics of the mother and father (when the
father is present).  This data is recognised to be of high quality
and very close to universal (see for example \citet{Mikkelsenetal2015}).

The average age of mothers in Chile has risen from slightly over 26 in
1990, to slightly under 28 in 2015 (Figure \ref{trendlongMother}).  The
average age of mothers increased constantly from 1990 until approximately
2004, before falling slightly, and ascending once again from 2009 onwards.
This reduction in maternal age occurred during a considerable slow-down
in growth, and an uptick in the number of births each year (Figure
\ref{NBirths}), in line with results suggesting countercyclicality in
fertility.  Panel b of Figure \ref{trendlongMother} displays the proportion
of teenage births (among all births), which rose until the early 2000s,
began to fall until the growth slowdown in the mid-2000s, and has fallen
sharply from 2007.
\begin{figure}[htpb!]
  \begin{center}
    \caption{Trends in Maternal Characteristics in Chile}
    \label{trendlongMother}
    \begin{subfigure}{.5\textwidth}
      \centering
      \includegraphics[scale=0.6]{../results/trends/edad_m_year.eps}
      \caption{Mother's Age}
      \label{trendMAge}
    \end{subfigure}%
    \begin{subfigure}{.5\textwidth}
      \centering
      \includegraphics[scale=0.6]{../results/trends/teenbirth_year.eps}
      \caption{Proportion of Teenage Births}
      \label{placebo-lbw}
    \end{subfigure}
  \end{center}
  \floatfoot{\textsc{Notes to figure \ref{trendlongMother}}: Yearly averages of
    age and the proportion of all mothers aged under 20 years of age based on
    Ministry of Health (DEIS) microdata covering all births in Chile between
    1990 and 2015.
  }
\end{figure}

\begin{figure}[htpb!]
  \begin{center}
    \caption{Number of Births per Year}
    \label{NBirths}
    \includegraphics[scale=1]{../results/trends/N_year.eps}
  \end{center}
\end{figure}

We display descriptive plots of average birth outcomes across time in
figure \ref{trendlong}.  These indicators, particularly birth weight,
improved sharply following the transition to democracy in the early
1990s, and the implementation of a considerable public health reform.
Average birth weight increased by more than 60 grams, and the proportion
of low birth weight babies fell by a full percentage point (refer to
panels \ref{trendBW} and \ref{trendLBW}).  From the year 2000 onwards,
average outcomes have gradually worsened, in line with increases in
maternal age.

\begin{figure}[htpb!]
  \begin{center}
    \caption{Longer Term Trends in Birth Outcomes in Chile}
    \label{trendlong}
    \begin{subfigure}{.5\textwidth}
      \centering
      \includegraphics[scale=0.6]{../results/trends/peso_year.eps}
      \caption{Birth Weight}
      \label{trendBW}
    \end{subfigure}%
    \begin{subfigure}{.5\textwidth}
      \centering
      \includegraphics[scale=0.6]{../results/trends/lbw_year.eps}
      \caption{LBW}
      \label{trendLBW}
    \end{subfigure}
    \begin{subfigure}{.5\textwidth}
      \centering
      \includegraphics[scale=0.6]{../results/trends/semanas_year.eps}
      \caption{Gestation}
      \label{trendGest}
    \end{subfigure}%
    \begin{subfigure}{.5\textwidth}
      \centering
      \includegraphics[scale=0.6]{../results/trends/premature_year.eps}
      \caption{Prematurity}
      \label{trendPrem}
    \end{subfigure}
  \end{center}
  \floatfoot{\textsc{Notes to figure \ref{trendlong}}: Yearly averages of
    birth weight, the proportion of low birth weight births ($<$ 2500 grams),
    weeks of gestation, and the proportion of premature births ($<37$ weeks)
    from Ministry of Health (DEIS) microdata covering all births in Chile
    between 1990 and 2015.
  }
\end{figure}

Primary care in the public health system in Chile is provided by municipal
health centres which, among other things, provide pre-natal appointments
for pregnant mothers and families.  These municipal health centres exist
in each municipality in Chile (refer to Figure \ref{fig:ests} for geographic
distribution).  These health centres are distributed much more sparsely
in less populated northern and southern regions of the country.  Secondary
and tertiary care are provided in hospitals which are located in each
region of the country.  Births attended in the public health centre are
delivered in these hospitals.  The geographical distribution of hospitals
is displayed in Figure \ref{fig:hosps}, where once again these are
concentrated in the central region of the country where the largest population
resides.

\begin{figure}[htpb!]
  \begin{center}
    \caption{Geographic Distribution of Health Centres and Hospitals}
    \label{fig:healthgeo}
    \begin{subfigure}{.5\textwidth}
      \centering
      \includegraphics[trim={8cm 2cm 9cm 2cm},clip,scale=0.86]{figures/Mapa_Estab_punto3_final.eps}
      \caption{Health Clinics}
      \label{fig:ests}
    \end{subfigure}%
    \begin{subfigure}{.5\textwidth}
      \centering
      \includegraphics[trim={8cm 2cm 9cm 2cm},clip,scale=0.86]{figures/Mapa_Hosp_punto3_final.eps}
      \caption{Hospitals}
      \label{fig:hosps}
    \end{subfigure}
  \end{center}
  \floatfoot{\textsc{Notes to figure \ref{fig:healthgeo}}: Geo-referenced hospital
    and Health Clinic information from the Ministry of Health of Chile.  All points
    represent public hospitals and health clinics.
  }
\end{figure}



\end{spacing}
\end{document}
