\documentclass[12pt]{article}
%===============================================================================
%=== (1) Packages
%===============================================================================
\usepackage[english]{babel}
\selectlanguage{english}
%\usepackage[utf8]{inputenc}

\usepackage{amsfonts}
\usepackage{amsmath}
\usepackage{amsthm}
\usepackage{appendix}
\usepackage{bm}
\usepackage{booktabs}
\usepackage[usenames, dvipsnames]{color}
\usepackage{graphicx}
\usepackage[margin=0.7in]{geometry}
\usepackage{epstopdf}
\epstopdfsetup{update}
\usepackage{helvet}
%\usepackage{hyperref}
\usepackage{indentfirst}
\usepackage{pdflscape}
\usepackage{morefloats}
\usepackage{natbib} \bibliographystyle{aea}
%\bibliographystyle{abbrvnat}\bibpunct{(}{)}{;}{a}{,}{,}
\usepackage{setspace}
\usepackage{subcaption}
\usepackage[capposition=top]{floatrow}
\usepackage{subfloat}
\usepackage[latin1]{inputenc}
%\usepackage[pdf]{pstricks}
\usepackage{pdfpages}
\usepackage{multirow}
\usepackage[bottom]{footmisc}

\usepackage{pgfplotstable}
%\usepackage{filecontents}
\usepackage[hypertexnames=false]{hyperref}
% \renewcommand{\familydefault}{\sfdefault}
%===============================================================================
%=== (2) Specific Instructions
%===============================================================================
\hypersetup{
    colorlinks=true,      
    linkcolor=BlueViolet, 
    citecolor=BlueViolet, 
    filecolor=BlueViolet, 
    urlcolor=BlueViolet   
}

%\setlength\topmargin{-0.375in}
%\setlength\textheight{8.9in}
%\setlength\textwidth{5.8in}
%\setlength\oddsidemargin{0.4in}
%\setlength\evensidemargin{-0.5in}
\setlength\parindent{0.25in}
\setlength\parskip{0.15in}

\definecolor{blue}{HTML}{84CECC}
\definecolor{gr}{HTML}{375D81}
\DeclareMathOperator{\plim}{plim}
\newcommand{\Var}{\mathrm{Var}}
\newcommand{\Cov}{\mathrm{Cov}}
\newcommand{\Bias}[2]{\frac{\Cov[#1,#2]}{\Var[#1]}}

%The Importance of a Large Early-Life Social
% Inclusion Program on Neonatal Health Outcomes in Latin America

%  the proportion of low birth weight babies, and the proportion
%  of premature births. What's more, we validate micro-level between-mother
%  estimates with difference-in-difference estimates based on time-varying program
%  roll-out at the sub-national level.  Taken together our estimates suggest
%  that program participation increased weight at birth by 12 or 13 grams, at
%  an estimated public cost of \$18 per gram.  These estimates are comparable to
%  those observed in a developed country setting and have important efficiency and
%  equity implications for a developing economy. We show that program participation
%  closed the prevailing early life health gap between targeted program
%  participants and richer non-participants, and that they imply considerable
%  changes in cognitive achievement in the long-run.

%  [ACTUAL NEW SUBTITLE]:
%  Assessing Equity and Efficiency in an Early-Life Health Program in Chile
%===============================================================================
%=== (4) Title, authors, data
%===============================================================================
\title{\textbf{Growing Together} \\
  Estimating the Effects of a Large Early-Life Social Inclusion Program on
  Neonatal Health Outcomes in Chile\thanks{We
         are grateful to Rodrigo Alarc\'on, Jeanette Leguas and Felipe Arriet of
         the Ministry of Social Development of Chile and Andr\'es Alvarez of the
         Ministry of Health for providing invaluable data linkages and other
         guidance.  We thank Serafima Chirkova, Rudi Rocha, Gabriel Romero, and
         seminar audiences at UNU-WIDER Mozambique, %Universidad de Concepci\'on, and
         %Universidad de la Rep\'ublica
         for very useful comments.  We are
         particularly grateful to Fresia Jara and team at Hospital San Juan de
         Dios for providing interviews and discussion regarding program
         functionality.  We gratefully acknowledge financial
         support from the CAF Development Bank's Research Program on Health and
         Social Inclusion in Latin America.  Damian Clarke additionally
         acknowledges the generous support from the Comisi\'on Nacional de
         Investigaci\'on Cient\'ifica y Tecnol\'ogica of the Government of Chile
         (grant reference 11160200).  The results and views in this paper are
         our own. Any errors are our own.}}
\author{Damian Clarke\thanks{Universidad de Santiago
    de Chile, and CSAE Oxford. Contact: Department of Economics, USACH, Avenida Libertador
    Bernardo O'Higgins, 3363, Estaci\'on Central, Chile. E-mail: damian.clarke@usach.cl.}
  \and Gustavo Cort\'es Mend\'ez\thanks{Department of Economics, Universidad de Santiago
    de Chile. Email: gustavo.cortes@usach.cl.}
\and Diego Vergara Sep\'ulveda\thanks{Department of Economics, Universidad de Santiago
    de Chile Email: diego.vergarase@usach.cl.}}
\date{\today}

\begin{document}
\maketitle
\vspace{-1.5cm}
\begin{center}
%\textcolor{red}{\textbf{Preliminary Draft}}
\end{center}
%-------------------------------------------------------------------------------
\begin{spacing}{1.4}
\begin{abstract}
  We estimate the impact of participation in \emph{Chile Crece Contigo} (ChCC),
  Chile's flagship early-life health and social welfare program, on neonatal
  health outcomes. We use administrative birth data matched to social benefits
  usage, and the staggered program roll-out, to identify the impact of
  participation.  We validate results using within mother-variation in program
  access across time.  We find that this targeted social program had significant
  effects on birth weight and other early life human capital measures.  These
  benefits are largest among the most vulnerable groups, and shift outcomes
  toward the bottom, but not at the very bottom, of the distribution of health
  at birth.  We show that program is efficient when compared to other neonatal
  health programs, and in terms of the implied cost of cognitive achievements
  in the long-run.
\end{abstract}
\noindent \textbf{JEL codes}: H23, O15, I14, H43, O38, H51. \\
\noindent \hspace{1mm} \textbf{Keywords}: Public health; neonatal health; social
security; efficiency; early life investments. \\

\clearpage
\noindent \textbf{Authors' Note} \\
This is a preliminary draft submitted to CAF Development Bank as part of the
Program on Health and Social Inclusion in Latin America.  All code producing
empirical results is finalised and publicly available
\href{https://github.com/damiancclarke/ChCC-neonate/tree/master/source}{here},
however we are still awaiting receipt of one auxiliary database. This database
provides the precise roll-out date of the program of interest in each
municipality.  Current results are based on a database indicating if each
municipality is an early- or late-enrollee, giving the approximate date of
entry of each municipality into the program.  Upon receipt of the updated
database, estimated regression results are likely to vary slightly from those
reported with the current iteration of analysis code, and these updated
results will be reported in future versions of the paper.


\clearpage
\section{Introduction}
The importance of early life health over the entire life course of an
individual has been extensively recognised in the economic (and
non-economic) literature \citep{Almondetal2017,AlmondCurrie2011,Barker1990}.
This has lead to considerable investments in fetal and infant health in
a wide range of contexts.  Public spending on infant and maternal health
during the pre-natal period often makes up a central pillar of the social
safety net (see for example discussion in \citep{BitlerKaroly2015}), while
considerable public spending is focused on technological innovation and
remedial investments to improve neonatal health outcomes
\citep{Almondetal2010,Bharadwajetal2013}.

An important motivation of these early-life health policies owes
to the dynamic complementary between the efficiency of investments in health
early in life and investments later in life.  In an influential series of
papers, \citet{HeckmanCunha2007,CunhaHeckman2009,Cunhaetal2010} argue that
early-life remedial investments are not only efficient, but equity promoting.
This has lead, at times explicitly, to policies targeting early-life health
outcomes as a basic column of the social safety net across the developed and
the developing world.\footnote{For example,
  in the context of Chile Crece Contigo---the policy we propose to examine in
  this paper---the policy's design explicitly references
  \citet{HeckmanCarneiro2003} as support for the implementation of a
  large-scale early-life social program (see the official materials in
  \citet{Arrietetal2013}).}

In this paper we study the equity and efficiency implications of a large
targeted public health program.  We examine the program \emph{Chile Crece
  Contigo} (hereafter ChCC): a national-level health program 
explicitly designed to target early-life health in vulnerable groups.
ChCC was implemented in Chile in 2007, offering a basket of medical
and social services, information and supplies to all children enrolled
in the public health system.\footnote{The Chilean health system consists
  of a private and public stream and users nominally choose between public or
  private care. An associated monthly payment is automatically deducted
  from all formal salaries as a previsional payment.  This payment is
  either made to the public health insurance (FONASA) or a private health
  insurer known as an ISAPRE.  Any individual unable to pay contributions
  is covered by the public FONASA system.  The private system is
  consderably more costly in terms of out of pocket costs.  Recent
  administrative data suggests that 76\% of the population is covered
  by public care.  Nationally, 67\% of beds are in the public system and
  the remaining 33\% are in the private system \citep{DEIS2016}.}  As well
as a transversal series of benefits, an additional series of means-tested
benefits were provided to families classified as part of the 60\% of most
vulnerable families in the country.  ChCC also has a stated
aim of addressing divergent health outcomes in socially excluded groups,
releasing materials in both Spanish and native indigenous languages, given
the well-documented health disparities among indigenous people across the
world, and in Chile \citep{Andersonetal2016}.\footnote{Chile's population
  is 4.58\% indigenous, the majority of whom are Mapuche, and this group
  has been documented as having poorer birth, neonatal and child health
  outcomes \citep{Andersonetal2016}.}\textsuperscript{,}\footnote{In order
  to provide an idea of the program's scope, we provide a brief list of
  program benefits in appendix \ref{ascn:benefits} of this paper.}

In terms of total cost, ChCC is one of the largest health or welfare
programs in Chile.  Recent figures suggest that ChCC spending currently
accounts for almost 1\% of the national budget. And in terms of coverage,
this program is very large, currently reaches between 75 to 80\% of all
newborns in the country. Despite the
size and scope of ChCC, few rigourous or well identified studies have been
conducted on the program's effectiveness, and none, as far as we are aware,
have examined the policy's effect on birth outcomes or survival
during gestation.%\footnote{As well as searching the economics
  %literature, we conducted a search on PubMed using the keywords
  %``Chile Crece Contigo'' AND ``Child Health'' OR ``Chile Crece Contigo''
  %AND ``Neonatal Health'' OR ``Chile Crece Contigo'' AND ``Birth weight''
  %OR ``Chile'' AND ``Neonatal Health'' over the span 2006 to 2014
%(ie after the design of the program).}
In this study we take advantage of newly matched administrative data of
vital stastics and enrollment in public programs to conduct the first
such study, drawing identification from two (different) sources: the
first, based on time- and geographic variation in program roll-out and
intensity, and the second using within-mother variation in exposure
produced across siblings around the date of the policy's introduction.
The first strategy allows us to estimate a difference-in-difference
specification at the municipal level using the universe of live
births and fetal deaths, and the second allows us to estimate a
mother fixed effects model (at the individual level), for a subset of
matched data.

Our results suggest that this program has considerable effects on
neonatal health in Chile.  Depending on the specification examined,
we find that the effect of program participation on birth weight
is between a 10 and 12 gram increase.  We also observe that the
program's rollout reduces the proportion of low birth weight babies.
Similarly, we observe an increase in gestational length.  Notably,
these impacts are focused on the most vulnerable families who receive
a richer set of program benefits, including home visits in some cases.
We find that the program has positive impacts across the birth weight
distribution, but that these are most substantial after the 60th
percentile of birth weights.
%reduction in the frequency
%of premature births.  Given the large coverage of the ChCC program,
%these results are noteable in national level averages, and appear to
%\emph{eliminate} the birth weight differential between the poorer
%program participants and the less-poor non-participants.

To put the program's effects in context, we calculate the inferred cost
of producing a gram of birth weight, and the implications of this to
educational attainment later in life.  When combined with the cost of
running Chile Crece Contigo, our estimates suggest that the government
spends around \$21 per gram of birth weight---a figure that is comparable
to other large successful neonatal health programs, even in developed
countries, 
(such as the Special Supplemental Nutrition Program for Women, Infants,
and Children, or WIC, in the US).  What's more, given the well-known
positive effects of birth weight on later life outcomes, we estimate that
as an \emph{upper bound} cost, each \$7200 spent on Chile Crece Contigo
results in an additional 0.05 standard deviations of educational attainment
on later life test scores.  These results suggest that targeted public
health and social welfare programs can have large impacts in developing
and emerging economies, and that these impacts may last much longer than
the period in which an individual is enrolled in the program.

The importance of investment in health---and early-life health in
particular---as a driver of individual and national outcomes in the
developed world has been flagged in various dimensions.  Influential
work points to the importance of health as a determinant of equality
within countries \citep{Deaton2003}, and document the long-shadow of
early life insults to health in the developing world
\citep{CurrieVogl2012}.   The social determinants of health starting
\emph{in utero} have lead to the recent design and implementation of
many large, targeted early life social safety-net programs throughout
Latin America and the developing world \citep{Monteiroetal2015}.
The lessons from ChCC suggest that targeted health policies can
have a substantial impact on birth outcomes of their intended
recipients, but point to challenges in shifting very poor outcomes
even with quite intensive investments.

In what remains of this paper we briefly describe the ChCC program,
as well as the matched administrative data that allows us to link
birth outcomes with ChCC usage and intensity.  We discuss the proposed
estimation strategies to determine the impact of ChCC on neonatal health,
discuss estimated results, and in closing estimate the efficiency of
public spending on this program, benchmarking against other public
neonatal health programs, as well as the estimated value of improvements
in health at birth in Chile.

\section{Background}
\subsection{Chile Crece Contigo}
Chile Crece Contigo is a multidimensional early life health program,
targeting children from the first pre-natal control during gestation, and
following them through the first
five years of their life.\footnote{From 2018 onwards, this will be extended
  to the first seven years of life.}  It is the Government of Chile's
flagship social security program for
children, reaching in some form approximately 75-80\% of children in
the country.  The most comprehensive set of benefits are targeted to
children from the 60\% of most vulnerable families.\footnote{
  ``Vulnerability'' has historically been measured using a deterministic
  score assigned by Government social workers, known as the \emph{Ficha
    de Protecci\'on Social} (FPS), or Social Protection Score.  Families
  with a FPS inferior to 13,484 points are classified as belonging to
  the 60\% of most vulnerable households.  Additional details of the FPS
  can be found in \citet{Herreraetal2010}.}  ChCC is jointly implemented
by the Minsitry of Social Development, the Ministry of Health, the
Ministry of Education, and the Ministry of Labour, and is delivered by
a local network of public providers in each municipality (known as the
Chile Crece Contigo Municipal Network).

The program was implemented gradually throughout the country, starting
in June of 2007.  The yearly expansion in program size, both in terms of
number of municipalities and number of pregnancies covered is displayed
in figure \ref{fig:coverage}. In
the first year the program covered 159 of Chile's 346 municipalities,
before being extended to all municipalities in early-2008.  Program
participation among pregnant women also increased in line with geographic
coverage.  In figure \ref{fig:coverage} we plot administrative figures of
program usage over the life of the program.  In 2010 the program covered
greater than 200,000 pregnancies nationwide.\footnote{According to vital
  statistics data, in 2010 there were 250,643 live births.  We note
  however that a pregnancy will be contained in the ChCC administrative
  data plotted in figure \ref{fig:coverage} even if the mother miscarries
  or a fetal death is recorded.}
The program was institutionalised as a basic
pillar of the Social Security system in 2009, with the approval of a
law\footnote{The Law 20.379 was passed unanimously by parliament on
  April 2nd, 2009 to ``institutionalise the subsystem of integral
  protection of infancy, Chile Crece Contigo''.} guaranteeing its
ongoing existence.

The program consists of two main pillars.  The first is the Program
Supporting Bio/Psycho/Social Development (PADBP), and the second
is the Program Supporting New-Borns (PARN).  The PADBP pillar begins
at the first pre-natal medical check-up, with the main goal of
supporting fetal and child development by providing information and
ongoing support in periodic check-ups, and in certain circumstances,
home visits.  We outline the principal components of the PADBP pillar
in more detail below.  The second program arm, the PARN, begins at the
birth of the child.  Among other things, this pillar provides a
comprehensive kit of materials to all newborns born in the public
health system including a crib, blankets, baby carrier, toys and
didactic materials, clothing and sanitary products.  In what remains
of this section we provide a description of the components of the PADBP
program, focusing only on the pre-natal components.  Additional
discussion of the program, including post-natal components is provided
in Appendix \ref{app:program}.

\paragraph{Pre-Natal Components of ChCC} The design of ChCC called for
an increase in the amount of time spent on pre-natal controls (with
midwives in public health clinics) from 20 minutes per appointment to
40 minutes per appointment.  The increased time was used on newly
incorporated components, such as the application of standardised
tests for pre-partum depression, social support programs, and information
for fathers or partners.  ChCC targets 7 pre-natal controls in public
health centres.  At the date of the first pre-natal control, families are
supplied with an information kit (in Spanish or one of five indigenous
languages or regional dialects), as well as a (music) CD for pre-natal
stimulation.  Any person meeting a set of pre-defined risk factors%
\footnote{These factors are: a first pre-natal control at 20 weeks or
  later, the pregnant women being aged under 18 years, having 6 or
  fewer years of primary education, insufficient family support,
  ``rejection of the pregnancy'', symptoms of depression, substance
abuse, or suffering from intra-family violence.} also
receives an additional psycho-social evaluation to determine whether
they are referred for immediate additional support.

Along with these universal benefits, families flagged as vulnerable
pertaining to the 60\% most vulnerable of the populuation receive a
series of preferential benefits.  These benefits begin at the
first pre-natal check-up with the definition of a personalised
plan created between municipal health workers and families, as
well as hour long home visits from social workers paramedical
technicians.\footnote{These home visits are not universally
  offered among the preferential group.  Home visits are targeted
  to families with a greater number of risk factors.} Finally,
vulnerable families are referred to the ChCC Municipal Network,
which includes meetings with municipal workers offering information
related to education and labour market programs where relevant,
information regarding other government programs and community
services, and eventually access to free child care. \emph{In situ}
interviews conducted by the authors with midwives and social
workers involved in the program highlighted that the implemention
of ChCC resulted in a considerable increase in the quality of
pre-natal care offered, and the ability to easily refer families
between institutions.  We provide additional information regarding
the scope and design of the program in Appendix \ref{app:program},
and a comprehensive list of program benefits is available in
\citet{MDS2014}.
%\textcolor{red}{Add discussion about prenatal care coverage in Chile.}

\subsection{Existing Evidence on The Impact of Early Life Programs on Infant Health}
A well-established body of work---much in the economic
literature---has documented the importance of public policies
on indicators of health at birth and during gestation.  These
can be broadly split into two types of programs: those explicitly
targeting infant health, and those with indirect impacts on
infant health.

There is relatively less evidence on programs explicitly
targeting infant health. Never the less, convincing evidence
from the United States shows that publically provided food and
nutritional advice to pregnant mothers has considerable effects
on birth outcomes.  The Special Supplemental Nutrition Program
for Women, Infants, and Children (WIC), has been shown to have
appreciable impacts on health at birth (refer to
\citet{BitlerKaroly2015} for a clear overview).  A number of
policies directly designed to targeted health at birth exist
in Latin America, though often rigourous evaluations have not
been implemented. These include programs such as Plan Nacer
(Argentina) and Qali Warma (Peru).  One notable exception is
a CCT from Bolivia.  \citet{Celhayetal2016} identify a significant
reduction in rates of still birth following receipt of a relatively
small CCT.

Considerable evidence also exists on the impacts of non-targeted
welfare policies on health at birth.  Evidence from the United
States suggests that the Supplementary Nutrition Assistance Program
(Food Stamps) may increase birth weight by as much as 20 grams
\citep{Almondetal2011}, and unintended impacts on child health
have also been identified from the Earned Income Tax Credit
\citep{Hoynesetal2015}.  Evidence also exists documenting the
impact of receipt of conditional cash transfers on infant health,
even when these transfers were not directly targeting these
outcomes.  This includes the PROGRESA/Oportunidades program
in Mexico \citep{Barham2011}, and the PANES program in Uruguay
\citep{Amaranteetal2016}

%Lim, S. S., Dandona, L., Hoisington, J. A., James, S. L., Hogan, M. C., and Gakidou, E.
%(2010). India’s janani suraksha yojana, a conditional cash transfer programme to increase
%births in health facilities: an impact evaluation. The Lancet, 375(9730):2009–2023.

%Barham, T. (2011). A healthier start: the effect of conditional cash transfers on neonatal
%and infant mortality in rural mexico. Journal of Development Economics, 94(1):74–85.

\subsection{Other Social Safety Net Programs in Chile}
Chile Crece Contigo joined a number of other targeted social
security programs in Chile.  However, unlike other programs
offered by the Ministry of Social Development, Chile Crece
Contigo focuses exclusively on the early life stages, and has
considerably extended coverage.

The Chile Solidario program is focused on poverty reduction, and
is targeted to the most vulnerable 10\% of the population.  This
program includes a cash transfer (which fades out over time) and
a series of home visits.  This program has been demonstrated to
increase the take up of employment programs, as well as participation
in other public policies \citep{Carneiroetal2014}.  Other
programs targeted to families with children include the Subsidio
\'Unico Familiar, a subsidy for families with children, as well
as a series of targeted scholarships and school meal programs.
In each case, these policies are targeted to a more restricted
group than ChCC recipients \citep{Herreraetal2010}.  One
component of the (targeted) component of ChCC is ensuring that
vulernable families are adequately enrolled in additional
social policies for which they are eligible.

\section{Data}
\label{scn:data}
Vital statistics covering all births occurring in Chile are
publicly available from 1990 until 2015 from the Ministry of
Health. Additonally, data on fetal deaths occurring after 22
weeks of gestation are available from 2002 onwards. Vital
Statistics data in Chile covers greater than 99\% of all births,
and coverage is stable over time.  We use the full universe of
births and fetal deaths, and match this with administrative data
on Chile Crece Contigo usage in the gestational period provided
by the Ministry of Social Development (MDS).  This data allows us
to calculate usage by month for each of the 346 municipalities of
Chile.\footnote{Municipalities in Chile are the third level
  administrative district, and the lowest level of local
  governance, after provinces and regions.  In Chile there are
  346 municipalities, 54 provinces, and 15 regions.}  The precise
date of program roll-out by municipality is also provided by the
MDS.

This birth data allows us to observe a range of human capital measures
at birth.  These are the weight of the baby, the baby's length
in centimetres, and the gestational length as recorded at birth. These
measures have been consistently shown to have large and long-lasting
effects on health and well-being \citep{AlmondCurrie2011b}.
Although Apgar and head circumference are measured at birth in
Chile, they are not currently available in administrative data.
Along with measures of health immediately at birth, we are able
to calculate rates of fetal death per live birth by combining
fetal death registers with live birth registers.  While fetal death
data only records deaths occurring in hospitals, recording is
consistent throughout the country.

Administrative (micro-) data is collapsed at the municipal by
month level, and matched with data on ChCC intensity by municipality
and month.  We match use all births occurring between January of 2003
and December of 2010 (inclusive).  ChCC data is available from
mid-2007 (the first date of program roll-out) until 2010, and the
pre-2007 provides coverage of the pre-reform period.  This results in
a sample of 1,917,085 births occurring to 1,241,514 mothers. When
collapsed to the municipal level, this results in 31,842 observations.
The theoretical maximum number of observations is 346 municipalities
$\times$ 8 years $\times$ 12 months (33,216 municipalities), but a
number of smaller municipalities do not have births in each month.

In Table \ref{tab:sumstats} we provide summary statistics of
principal health indicators at birth, as well as rates of participation
in Chile Crece Contigo by municipality and month.  Municipal-level
averages are largely in agreement with values observed in Vital
Statistics data observed else where (we also provide summary statistics
at the level of births in Appendix Table \ref{tab:sumstatsMother}.  The
average birth weight in municipal averages is approximately 3,350 grams,
gestation is on average 38.7 weeks, and 5 and 6\% of births are low birth
weight or premature (respectively).  In administrative data from 2003 to
2010 25\% of mothers are observed to participate in Chile Crece Contigo,
though this value is considerably lower than actual participation rates
once the program was implemented, as the program only began running from
June of 2007 onwards.  Rates of usage of the program (only the gestational
component) by time are displayed in Figure \ref{fig:usageTime}. In
examining the number of births occurring in each municipality in
Table \ref{tab:sumstats} (``Number of Births'') we observe considerable
variation in the size of municipalities. Depending on the municipality,
the number of births per month ranges from as low as 1 birth (conditional
on there not being 0 births) to as high as 787 births.  As we discuss
below, regression estimates are consistently weighted by the number of
births per cell.

For a sub-set of births, we are able to match all siblings with
mothers, as well as with the mother's participation in social
programs.  For these mothers we thus observe her full fertility
history, as well as whether she participated in Chile Crece Contigo,
and her social vulnerability score, defining the degree of usage
of ChCC for which she will be eligible.  Approximately 40\% of
births are correctly matched to their mother.  We thus use this
matched micro-data sample as an auxiliary test of the main result.
While this does not include the full universe of births used in
the municipal level analysis, the resulting data set is a unique
source of information on births in Chile matched to the mother's
take-up of social safety net programs.  In appendix \ref{MFE}
we discuss the match rates, as well as the characteristics of
the matched and unmatched sample.  The 60\% of unmatched children
were overwhelmingly matched with their father rather than their
mother in the social registry, and so are excluded from micro-level
analyses given the lack of information on the \emph{mother's} usage
of public programs.

\section{Methodology}
\paragraph{Estimating the Impact of ChCC}
We leverage the time-varying roll-out of ChCC by municipality to estimate
the following flexible difference-in-differences (DD) model:
\begin{equation}
  \label{eqn:DD}
  Infant Health_{ct} = \alpha_0 + \alpha_1 ChCC_{ct} + \bm{W_{ct}\alpha}_{w} + \mu_t + \lambda_c + \eta_{ct}
\end{equation}
where $InfantHealth$ measures average birth outcomes for each
municipality $c$ in period $t$.  In principal specifications,
the unit of time is month by year. The variable $CHCC_{ct}$ is
a treatment measure indicating the proportion of all births in
each municipality and month which received coverage from the
\emph{Chile Crece Contigo} program during gestation.  This measure is
always 0 prior to the program implementation, and increases
to reach approximately 75\% of the population following the
program's implementation.  Given that fact that the program
was implemented in different municipalities at different times,
we include full municipality and time (month$\times$year) fixed
effects as $\lambda_c$ and $\mu_t$ respectively.

If implementation of the policy were completely random,
$\alpha_1$ will give the unbiased effect of ChCC participation on
infant health measures.  However, as we may be concerned that
early adopting municipalities with better infrastructure were
following different trends over time, we propose to include
a series of time-varying controls for health infrastructure
and municipal development $\bm{W_{ct}}$, and in supplementary
regressions also examine the robustness of results to regional
and municipal time trends, and separate regional and municipal
fixed effects for each year.\footnote{Despite these considerations,
  we note that there is no particularly notable geographic
  clustering of early- and late-adopted municipalities,
  even within metropolitan areas such as Santiago (refer to
  Appendix Figure \ref{fig:map} to observe the variation in
  roll-out by area).}
As is typical, we will cluster standard errors by municipality
(346 municipalites) to account for the well-known time-dependence
in unobserved stochastic errors by geographic area
\citep{Bertrandetal2004,CameronMiller2015}.

Our principal outcome measures of $InfantHealth$ are based on
the available measurements recorded in vital statistics data,
and consist of birth weight in grams, low birth weight
($<$2,500 grams), birth length in centimetres, gestation time
in weeks, prematurity ($<$ 37 weeks gestation), and the frequency
of fetal deaths.  Given that we propose to use various outcome
measures and a single indepedent treatment variable, we will
correct for multiple hypothesis testing in a number of ways.  We
briefly return to this point in the following subsection.

%\subsection{Inference, Robustness Strategies and Extensions}
\paragraph{Alternative Identification Strategies}
While our main identification strategy takes advantage of
the time-varying roll-out of Chile Crece Contigo by municipality,
we also estimate a child-level regression controlling for mother
fixed effects leveraging within mother variation in policy exposure.
For each mother in matched administrative data we observe all
births occurring between 2003 and 2010, both before and after policy
implementation.  The inclusion of mother fixed effects thus allows
us to capture all time-invariant unobservables of mothers correlated
with program participation.  We also include a number of time-varying
controls, including maternal age and birth order fixed effects.

We estimate mother fixed effect models only as a robustness check
rather than our main specification given that the match between
children and mothers was not universal (while municipal-level regressions
are based on complete vital statistics data).  As discussed above,
approximately 50\% of births were correctly merged with data on
their mother's use of public programs, while the remaining births
were merged with the father's social program participation.
We provide additional details regarding the precise mother FE
specification to be estimated, as well as match rates and
characteristics of matched and unmatched children in Appendix
\ref{MFE}.

\paragraph{Placebo Tests}
We observe monthly usage rates of ChCC during gestation for each
municipality following the reform's implementation.  This measure
of usage by municipality and time is our independent variable of
interest in specification \ref{eqn:DD}.  In order to ensure that
our estimates for $\alpha_1$ are not simply capturing systematic
differences between municipalities with varying time and intensity
of ChCC, we propose to conduct a series of placebo tests using
lagged measures of the independent variable of interest.%
\footnote{Frequently, identifying assumptions in DD-style models
  are tested by event study analysis, where treatment status is
  interacted with a full set of lags and leads.  In the setting of
  this paper, where program usage is a continuous rather than
  binary measure, an event study is unsuitable given the lack of
  binary treatment, and the fact that all municipalities are
  eventually treated.  We thus proceed with the lagged placebo
  tests as described in this section.}
Specifically, we estimate the following model:
\begin{equation}
  \label{eqn:placebo}
  Infant Health_{ct} = \gamma_0 + \gamma_1^k ChCC_{c,t-k} + \bm{W_{ct}\alpha}_{w} + \mu_t + \lambda_c + \eta_{ct} \qquad \forall k \in {1,\ldots,40}.
\end{equation}
Here, rather than regressing ChCC usage in the month of birth on
concurrent outcomes, we regress usage in earlier months.  Provided
that improvements in birth outcomes are truly flowing from the
program, rather than systematic differences between municipalities,
we should see that lags of $ChCC_{ct}$ do not impact birth outcomes
in future periods conditional upon municipal and time fixed effects.
%Thus, we are interested in testing estimated coefficients
%$\widehat{\gamma}_1^k$ to determine 

%In order to run a consistency check on DD results we propose
%to estimate a full event study.  This event study is a test in the style
%of \citet{Granger1969}.  We will examine precisely when indicators
%diverge between early and late treatment areas, estimating the following
%specification:
%Here we interact a series of indicator variables indicating policy
%implementation leads ($-k$) and lags ($+l$).  If the difference between
%early and late implementing states only emerge when the policy is
%implemented, there should be no differential impact in any of the
%lead terms, suggesting an individual and joint test that each of the
%$\gamma^{lead}$ terms are equal to zero.

\paragraph{Correcting for Multiple Hypothesis Testing}
We examine the impact of ChCC on a number of distinct outcome
variables.  These variables were pre-defined, and are the only
measures of neonata health avaialble in published vital statistics
data in Chile.  Given that we exame the impact of the policy on a
number of separate outcomes, we correct for multiple hypothesis
testing.  We do this in two ways.  Firstly, in order to ensure
adequate size we apply \citet{RomanoWolf2005}'s stepdown hypothesis
testing algorithm which fixes the Family Wise Error Rate at a set
level $\alpha$.  This hypothesis correction technique is considerably
more powerful than older FWER techniques such as Bonferroni or Holm,
and is increasingly used in the economic literature (see for example
\citet{Gertler2014}).  This is also a more demanding correction than
corrections which fix the False Discovery Rate of findings.
Secondly, we construct a single index based on the full set of
outcome variables which gives more weight to variables which
provide the most independent variation.  To construct this index
we follow the procedure described in \citet{Anderson2008}, allowing
us to examine the estimated effect of ChCC on a single outcome
variable.

\paragraph{Distributional Effects of the Policy}
Finally, along with regressions examining birth weight, and low
birth weight we are able to observe the effects of the policy over
the entire range of the birth weight distribution, to examine
precisely where effects are observed.  While our principal model in
\ref{eqn:DD} will allow us to examine the impact of ChCC on mean
outcomes in each municipality, it is likely that mean impacts vary
considerably throughout the distribution of health at birth.
We thus replicate equation \ref{eqn:DD}, however the dependent
variable of mean birth weight by municipality is replaced with
the birth weight at a range of percentiles of the birth weight
distribution.  We estimate these distributional effects only
for birth weight (both in levels and in logarithms) given that
this is the only non-categorical variable available measuring
early life health.


\section{Results}
\subsection{Program Impacts}
\subsubsection{Headline Effects}
\label{scn:headline}
Baseline estimates based on municipality and time-varying exposure
to the Chile Crece Contigo program are presented in table \ref{mDD}.
Estimates in this table are all produced by an archeypical DD model
including ChCC coverage as the independent variable of interest, and
municipality and month$\times$year fixed effects.  Standard errors
are clustered by municipality.

Results from table \ref{mDD} suggest large and significant effects,
particularly for birth weight and gestational length.  As the
independent variable is measured as the proportion of ChCC coverage
in a municipality, an increase in 1 unit of this variable is equivalent
to moving from 0 to universal ChCC coverage, or the mean impact of
ChCC in the population. The mean impact of Chile Crece Contigo is
estimated as an 11 gram increase in birth weight.  When examining
the proportion of low birth weight babies, results suggest that
ChCC brought about a reduction in these births by 0.3 percentage
points.  When compared to the absolute value of low birth weight
births, this is approximately a 5.5\% reduction. We find no impact
of ChCC on size at birth, but do observe a small increase in gestational
length. No statistically significant effect is observed when considering
the proportion of premature births.  Finally, in turning to fetal
deaths, we also observe a significant reduction, in 1.6 per 1,000
live births following the program's implementation and expansion.

We examine alternative specifications and controls in Appendix Table
\ref{tab:AltSpecs}.  Here rather than simply estimating a baseline
DD model with time and geographic fixed effects, we add additional
time varying controls, region and municipal specific linear time
trends, and region and municipality by year fixed effects.  Even in
the most demanding specification which includes both time-varying
controls as well as separate fixed effect for each municipality in
each year (346$\times$8 fixed effects), estimates largely agree
with those in the baseline DD model.  The estimated effect of ChCC
on birth weight falls slightly (to 10 grams), while the remaining
effects are quite stable, with the exception of the estimated effect
of ChCC on the rate of fetal deaths.  When year by municipality
fixed effects are included this effect no longer remains.  Similarly,
if we estimate using trimester$\times$municipality averages rather
than month by municipality outcomes, estimates remain quite stable
(refer to Appendix Table \ref{mDDt}).  While the precision of
estimates falls slightly, rendering a number of coefficients no longer
statistically significant at typical levels, the effect sizes agree
very closely with those in Table \ref{mDD}. %\footnote{We generally
  %prefer to report month by municipal estimates, as it allows for
  %greater variation in the independent variable of interest.
  %}
Finally, we correct for multiple hypothesis testing in Appendix
Table \ref{tab:MultHyp}.  Panel A presents uncorrected and
corrected $p$-values where we account for the fact that we are
likely to over-reject the null when testing the impact of ChCC
on multiple outcome variables.  Original $p$-values come from
estimates presented in Table \ref{mDD}, while corrected values
follow \citet{RomanoWolf2005}.  This is a demand correction,
ensuring that no null hypotheses will be incorrectly rejected
at a given size.  In this case, we still observe a statistically
significant effect on birth weight.  When considering an index
capturing infant health (where a positive value implies greater
health), we observe that regression the single index on rates
of participation in ChCC results in statistically significant
impacts.

We examine the plausibility of identifying assumptions using a
series of placebo tests.  These placebo tests use the ChCC
participation rates for each municipality, however assigning the
placebo reform treatment to a period entirely \emph{before} the
corresponding births had occurred.  Thus, if there is no general
prevailing difference in trends between municipalities different
roll-out timing or intensity of ChCC usage, we should observe
that all placebo tests based on pre-reform dates lead to
insignificant estimates of the effect of the placebo treatment on
birth outcomes.

These results are displayed in figure \ref{placebo}.  Each point
estimate and confidence interval corresponds to a placebo reform
lagged by the number of periods indicated on the $x$-axis.  In
general, the large majority of placebo tests indicate results
which are not statistically distinguishable from zero.  At times
short lags (such as 1 or 2 months) result in significant estimates,
however this owes to the fact that births occurring in closely
spaced months were exposed to similar levels of program coverage.
Moving further away from the true reform dates, few estimates
are statistically distinguishable from zero, suggesting that
identifying assumptions underlying DD estimates are reasonable.
%later (as occurred in the true reform).  We use all time periods
%for which full coverage is available, until arriving to the date
%of the true reform (indicated by the red dotted line in the figures).
%We observe that in nearly all cases placebo reforms lead to smaller
%and statistically insignificant estimates.  In the case of tests
%using birth weight, we observe one statistically significant result
%in the placebo tests (of approximately 40), and none when examining
%low birth weight.  The effects of the reform only begin to be
%observed when approaching the true reform date, reaching their
%maximum estimates when the true reform dates are used.

Estimates based on mother fixed effects for the matched sample
are presented in Table \ref{mFE}. We present fixed effects
estimates in each case also controlling for mother's age and birth
parity fixed effects which may vary
around the reform date.  Identification is thus driven by changes
in birth outcomes between siblings born before and after their
mothers began participating in Chile Crece Contigo, compared with
similar time siblings occurring to never-participating mothers.


Once again, we observe that the effect of Chile Crece Contigo
participation is large and statistically significant.  In this
case we \emph{do} observe an impact on the size of the baby at
birth, and the impacts on all other variables remain consistent
with those estimated from municipal-level DD models.  The effect
sizes observed for birth weight and gestational weeks are
considerable.  We estimate an effect of 19 grams in mother FE
models, equivalent to approximately 0.5\% of the mean birth
weight in Chile over the time period examined, and similar to the
reported effects of large successful programs world wide.  For
example, recent evidence suggests that participation in the Food
Stamp Program in the United States, one of the largest and most
costly social security programs, increases birth weight by
approximately 20 grams \citep{Almondetal2011}. Similarly,
participation in the supplementary nutrition program for Women,
Infants and Children is estimated to increase birth weight
by around 17-30 grams \citep{Hoynesetal2011,RossinSlater2013}.
Additional discussion related to the mother FE models, as well
as data match rates is provided in Appendix \ref{MFE}.


\subsubsection{Program Targeting and Equity}
While ChCC is universally accessible for any mother or family
participating in the public health system, the degree of benefits
of the program is means tested, and targeted more heavily to
families identified as most vulnerable.  In Table \ref{tab:FPS}
we estimate the impact of ChCC separately for three groups.
In Panel A we examine the impact of birth outcomes on the
60\% most vulnerable (the targeted group), in panel B we
focus on the 40\% most vulnerable (in early years, the targeted
group was only the 40\% most vulnerable), and in panel C we
examine the impact on the non-targeted group (those with a
Social Protection Score in the top 40\%, or those without a
Social Protection Score).

We observe that the impacts of the program are largest among
those in the targeted group, and smallest among those in the
non-targeted group.  However, in these specifications estimated
impacts become considerably less precise, and lose statistical
significance at typical levels.  For example, when focusing on
estimated effects, the impact of ChCC for the most vulnerable
60\% is estimated at 8 grams, while it is estimated as -1.2 grams
among the non-targeted group.

The importance of equity-promoting early-life health policies
are particularly important in the context of Latin America. Many
Latin American countries are characterised by irregular, rather
than universally poor, infant health outcomes
\citep{Belizanetal2007}.  Indicators are particularly sub-standard
among socially isolated groups, including low-income households,
rural communities, and indigenous people.  These early-life health
differentials are only magnified over the life course of
individuals, partially explaining the emergence of significant
gaps in adulthood in education, salary, and morbidity and mortality.
In the Chilean context this has been documented, where divergence
of outcomes at a very young age (birth weight) have important
effects on adacemic achievement up to 18 years later
\citep{Bharadwajetal2017}.  We return to this point in the
section which follows.

\subsubsection{Distributional Effects}
\subsection{Program Efficiency}
Mean impacts suggest that Chile Crece Contigo participation
increases average birth weight by approximately 11 grams in
the population.  In figure \ref{quintiles}, we examine the impact
at difference percentiles of the birth weight distribution.
In figure \ref{quintiles-level} we examine ChCC's impact on
absolute birth weight, and in figure \ref{quintiles-log} we
examine the natural logarithm of birth weight, given that it
is likely that the impact of the policy will be larger when
baseline birthweights are higher.  The log specification allows 
us to estimate the constant impact of the policy across the
birth weight distribution.  We examine the impact of the
policy starting at the 5th percentile of the birth weight
distribution, and ranging to the 95th percentile of the
distribution.

In figure \ref{quintiles-level} we observe that, although
point estimates of the policy are universally positive,
estimated impacts become statistically significant from the
40th percentile of the distribution.  In absolute terms
policy impacts peak at the 80th percentile of the birth weight
distribution, and are above 10 grams from the 50th percentile
onwards.  These impacts are standardised by birth weight
in figure \ref{quintiles-log}, where point estimate and
confidence intervals estimate the approximate percentage
change in birth weight associated with ChCC participation.
Once again, we observe that impacts are universally positive,
we can reject a 0 impact at 95\% confidence levels from the
40th percentile onwards, and in this specification, ChCC
is estimated to increase birth weight by as much as 0.5\%.
When standardising birth weight as a logarithm, we observe
that there is an approximately constant effect on birth
weight from the 60th percentile onwards.

Taken together with the findings from section
\ref{scn:headline}, these impacts point to the difficulty
in shifting outcomes towards the very bottom of the birth
weight distribution.  While we do find an impact on low
birth weight in table \ref{mDD}, we observe here that
impacts are higher among larger babies.  While this points
to the challenge of improving birth outcome at bottom of the
health distribution, especially in large public programs such
as ChCC, these improvements in birth weight from the 40th
percentile are certainly not trivial.  Indeed, evidence
from \citet{Royer2009} suggests that returns to birth weight
may actually be highest \emph{above} the low birth weight
cut-off.  We turn to considerations relating to these returns,
and returns of ChCC in particular, in the following
sub-sections.

\subsubsection{External Efficiency}
Chile Crece Contigo is the flagship early life health program
in the Chile, and one of the largest social safety net programs
of any type in the country.  It is also one of the most
important early life health programs in a middle or lower-middle
income country setting worldwide \citep{Richteretal2017}.  As
such, considerations of efficiency in public health care spending
are of considerable importance.  As we describe in table
\ref{tab:spending}, spending on ChCC is approaching 1\% of the
fiscal budget per year, documenting the importance of this policy
nation-wide.  Using the current exchange rate, spending on ChCC
in 2010 was approximately USD 330 million.

To provide a broader consideration of the program's impacts and
efficiency given public investment, we calculate the inferred
cost of producing one gram of birth weight through this policy.
In order to do so we compare the total cost of the pre-natal
portion of Chile Crece Contigo with the total grams of birth
weight produced by the policy.  In order to estimate the total
costs, we assign pro-rata costs by years in the program.  Given
that the prenatal component is approximately 1 of 5 years of
coverage, we assume that one fifth of costs are apportioned to this
sector.\footnote{In ongoing work we are working with National
  Budgets to examine the factibility of calculating the exact
  cost of only the pre-natal portion of the program.}  In order
to calculate the total birth weight gained due to the program, we
use our headline estimate (table \ref{mDD}) of approximately 11
grams.  Using these values, as well as the total number of
pregnant women covered per year, the inferred cost of a gram of
birth weight is approximately 12,000 Chilean Pesos\footnote{This
  value is calculated using the costs, the total estimated impact,
  and the number of program recipients as:
  \[
  \text{Inferred Cost}=\frac{\frac{1}{5}\times 126,446,362}{11.1 \text{ grams}\times161,834}=14,078 \text{ pesos/gram}
  \]} (or based on the
current exchange rate (1 USD is equivalent to 650 CLP),
approximately 21 US dollars.  Interestingly, this value is similar
in magnitude to that calculated from the US Food Stamp Program and
The WIC program \citep{Clarkeetal2017}.  If we compare this to
the estimated impacts and costs from a successful (non-targeted)
CCT in Uruguay, we also find that the cost per gram of birth weight
is of a similar magnitude.  The estimates presented in
\citet{Amaranteetal2014} suggest that the cost of producing a
gram of birth weight was approximately 17-30 USD.\footnote{These
  values are calculated using the monthly cost of the program
  (56-102 USD) multiplied by the number of months of pregnancy,
  and divided by the program's estimated impact on birth weight.
  We thus estimate the impact as:
  \[
  \text{Inferred Cost}=\frac{\$102\times 9}{30.8 \text{ grams}}=29.8 \text{ \$/gram}.
  \]
}

\subsubsection{Internal Efficiency}
While this value benchmarks the efficiency of the ChCC program compared to
other early life health programs, it provides less context on the implications
of these costs for social spending and development outcomes within the country.
In order to put these estimates in context, we can ask how investments in birth
weight compare to the \emph{returns} to birth weight in the country.  In Chile
there are a number of well-identified estimates of the value of birth weight to
later-life education, with significant and long-standing observed impacts
\citet{Bharadwajetal2013,Bharadwajetal2017}.  Using a similar within family
estimation strategy as proposed in specification \ref{eqn:panel} of this
paper, \citet{Bharadwajetal2017} estimate that a 10\% increase in weight at
birth increases child test scores by 0.05 standard deviations.  Using our
estimates discussed above, as well as data on birth weights in Chile, we can
thus back out the approximate amount required to be invested in ChCC to produce
an additional 0.05 standard deviations on educational outcomes.

From table \ref{tab:sumstats}, a 10\% increase in average birth weight is
334 grams.  Our calculation above suggests that the cost per gram of birth
weight produced by ChCC is 14,078 pesos, implying that the cost of 334 grams
is 4.7 million Chilean pesos, or approximately 7,200 USD.  What's more, these
costs are clearly an upper bound, as we ignore all other impacts of
improvements in early-life health.  While birth weight is a well known
determiniant of educational attainment, birth weight is also known to impact
labour market outcomes \citep{JohnsonSchoeni2011,CookFletcher2015,
  BehrmanRosenzweig2004,RosenzweigZhang2013,Caseetal2005}, the prevalence of
chronic morbidities \citep{Barker1995,AlmondMazumder2005,JohnsonSchoeni2011b},
mortality \citep{vandenBergetal2006}, and a range of psychological outcomes
\citep{Fletcher2011}.

\section{Conclusion} 
We estimate the impact of a large early-life health and social inclusion
policy, \emph{Chile Crece Contigo}.  This policy, explicitly designed to
target efficiency and equity is found to have considerable impacts on
health at birth over a range of dimensions.  Using municipal roll-out
and variation in intensity of use of ChCC in a difference-in-difference
specification, we estimate that participation in ChCC increased weight
at birth by 11 grams on average.  We also find a reduction in the proportion
of low birth weight babies, and in increase in gestational lenght.  These
results are validated by a large (but not universal) sample of micro-data
where within mother variation in program exposure is used to estimate the
policy's impact on infants.

We find that this policy is both equity enhancing, as well as quite
efficient when compared with other policies world-wide, and as a manner
to invest in human capital accumulation.  The impacts are observed to
be largest amongst the most vulnerable groups, which are specifically
targeted to receive preferential transfers in the program. Combined with
the cost of running Chile Crece Contigo, our estimates suggest that the
government of Chile spends approximately \$21 per gram of birth weight---%
a figure that is comparable to other large neonatal health programs,
even in developed countries.  What's more, given the well known positive
effects of birth weight on later life outcomes, we are able to estimate that
as an \emph{upper bound} cost, each \$7200 spent on Chile Crece Contigo
results in an additional 0.05 standard deviation of educational attainment
on later life test scores.  All told this paper suggests that public
investments in early life health in developing and emerging economies have
considerable returns when well targeted and well designed, and that these
impacts may propogate through the economy long after birth and program
implementation.



%We examine the importance of a large early life social safety net program
%in a middle income country.  This program---Chile Crece Contigo---is one
%of the largest social programs in Chile, reaching more than 150,000 pregnant
%women each year, and accounting for nearly 1\% of the national budget.
%Using newly generated administrative data matching all births with a program
%participation indicator, as well as time and geographical variation in program
%roll-out, we are able to combine a number of estimation strategies leading
%to plausibly causal effects under varying assumptions.
%
%We document, firstly, that this program has considerable effects on neonatal
%health in Chile.  Depending on the specification examined, we estimate that
%the program participation increases birth weight between 12 and 13 grams,
%reduces the probability of being low birth weight by up to 10\% and reduces
%premature births by as much as 7\%.  What's more, it appears to eliminate
%the birth weight differential between the poorer program participants and
%the less-poor non-participants.  Results appear to agree quite well whether
%working with between-mother micro-level estimates, or difference-in-difference
%estimates based on program roll-out nation-wide.



\newpage
\bibliography{references}
\newpage

\section*{Figures and Tables}
\begin{figure}[htpb!]
  \begin{center}
    \centering
    \caption{Timing of Program Rollout and Coverage}
    \includegraphics[scale=0.8]{./figures/ChCCcover.eps}
    \label{fig:coverage}
  \end{center}
  \vspace{-5mm}
  \floatfoot{\textsc{Notes to figure \ref{fig:coverage}}: The first
    program rollout occurred in June of 2007, with the remainder of
    municipalities joining in the second rollout in January of 2008.
    All coverage figures are based on ChCC reports \citep{Arrietetal2013}.
    %The program continued after 2010 and is due to do so indefinitely,
    %however we only consider the period up until 2010 given that our matched
    %birth data end in this year.
  }
\end{figure}

\begin{figure}[htpb!]
  \begin{center}
    \centering
    \caption{Usage of Gestational Component of ChCC by Month}
    \includegraphics[scale=0.8]{../results/ChCCtime.eps}
    \label{fig:usageTime}
  \end{center}
  \vspace{-5mm}
  \floatfoot{\textsc{Notes to figure \ref{fig:usageTime}}: Program usage
    by month is calculated from administrative data from MDS.  This captures
    all mothers who were recorded as having participated in the pre-natal
    components of ChCC.  Additional details can be found in section \ref{scn:data}
    of this paper.
    %The program continued after 2010 and is due to do so indefinitely,
    %however we only consider the period up until 2010 given that our matched
    %birth data end in this year.
  }
\end{figure}

\begin{table}[htpb!]
  \begin{center}
    \caption{Summary Statistics: Birth and Chile Crece Contigo Data}
    \label{tab:sumstats}
    \begin{tabular}{lccccc} \toprule
      & N& Mean & Std. Dev. & Min & Max \\ \midrule
      \input{../results/SummaryMunicipal-update.tex} \bottomrule
      \multicolumn{6}{p{15.2cm}}{{\footnotesize \textsc{Notes to Table
            \ref{tab:sumstats}}: Summary Statistics are displayed for
          municipality by month averages for
          each month from January 2003 to December 2010.  Averages are
          displayed for each municipality in which there is at least one
          birth in the given month.  The average number of births by
          comuna and month is displayed above.  There are 346 municipalities
          in Chile, and hence a maximum number of observations of 346
          municipalities $\times$ 8 years $\times$ 12 months, or 33,216
          municipality$\times$ month observations.  The difference between
          this maximum and the observed number of observations are cases
          where no births occurred.  Uncollapsed micro-data on births
          consists of 1,917,086 observations between 2003 and 2010.
          Additional details on this birth data is provided in
          Appendix \ref{app:context}.  Proportion enrolled in ChCC
          refers to the average proportion of births in each municipality
          which were covered by ChCC \emph{in utero} during the entire
          period of 2003-2010, and so is always zero prior to the implementation
          of ChCC in 2007/2008.
      }}
      \end{tabular}
  \end{center}
\end{table}

\begin{landscape}
\input{../results/comunaDD.tex}
\end{landscape}

\begin{figure}[htpb!]
  \begin{center}
    \caption{Placebo Tests}
    \label{placebo}
    \begin{subfigure}{.5\textwidth}
      \centering
      \includegraphics[scale=0.64]{../results/placebolag_peso.eps}
      \caption{Birth Weight}
      \label{placebo-peso}
    \end{subfigure}%
    \begin{subfigure}{.5\textwidth}
      \centering
      \includegraphics[scale=0.64]{../results/placebolag_lbw.eps}
      \caption{LBW}
      \label{placebo-lbw}
    \end{subfigure}
    \begin{subfigure}{.5\textwidth}
      \centering
      \includegraphics[scale=0.64]{../results/placebolag_gestation.eps}
      \caption{Gestation}
      \label{placebo-gest}
    \end{subfigure}%
    \begin{subfigure}{.5\textwidth}
      \centering
      \includegraphics[scale=0.64]{../results/placebolag_fDeathRate.eps}
      \caption{Fetal Deaths}
      \label{placebo-fdeaths}
    \end{subfigure}
  \end{center}
  \floatfoot{\textsc{Notes to figure \ref{placebo}}: Each point estimate
    and resulting confidence interval display the impact of a placebo test
    where the share of Chile Crece Contigo enrollees is lagged
    $j\in \{1,\ldots,40\}$ months, where $j$ is displayed on the horizontal
    axis. Each placebo test is estimated following the principal
    specification displayed in Table \ref{mDD}.  Additional notes relating
    to each model can be found in Table \ref{mDD}.}
\end{figure}

%ADD HIGH/LOW EDUC HERE (table 2)


\begin{table}[htpb!]
  \begin{center}
    \caption{Impacts by Vulnerability Quintile}
    \label{tab:FPS}
    \begin{tabular}{lccccc} \toprule
      &(1)&(2)&(3)&(4)&(5)\\
      & Weight &\ \ LBW \ \ &\ \  Size \  \ & Gestation & Premature \\ \midrule
      \multicolumn{6}{l}{\textbf{Panel A: 60\% Most Vulnerable}} \\
      \input{../results/FPS_1.tex}
      \\
      \multicolumn{6}{l}{\textbf{Panel B: 40\% Most Vulnerable}} \\
      \input{../results/FPS_2.tex}
      \\
      \multicolumn{6}{l}{\textbf{Panel C: Non-Targeted Group}} \\
      \input{../results/FPS_3.tex}
      %\multicolumn{6}{l}{\textbf{Panel D: Quintile 4}} \\
      %\input{../results/FPS_4.tex}
      %\multicolumn{6}{l}{\textbf{Panel E: Quintile 5}} \\
      %\input{../results/FPS_5.tex}
      \bottomrule
      \multicolumn{6}{p{15.8cm}}{{\footnotesize \textsc{Notes to Table \ref{tab:FPS}}:
          Identical specifications are estimated as in table \ref{mDD}, however now each
          model is estimated using \emph{only} observations which meet the criteria
          defined in panel headings. Classification of the 60\% and 40\% most vulnerable
          is based on the Government of Chile's offical scoring based on the
          \emph{Ficha de Protecci\'on Social} (FPS, or Social Protection Score in English),
          which is used to classify the degree of benefits received by families in ChCC.
          The official cut-off for the 40\% most vulnerable is a score of 11,734 points or
          lower on the social protection score, and for the 60\% most vulnerable is a score
          of 13,484 points or lower.  Any mother with a score above 13,484 (or who has not
          applied for a score) is not in the targeted group. Additional discussion of the
          FPS is available in \citet{Herreraetal2010}.}} \\
    \end{tabular}
  \end{center}
\end{table}

\begin{figure}[htpb!]
  \begin{center}
    \caption{Policy Impact by Percentile}
    \label{quintiles}
    \begin{subfigure}{.5\textwidth}
      \centering
      \includegraphics[scale=0.64]{../results/Percentiles_peso_Level.eps}
      \caption{Birth Weight (grams)}
      \label{quintiles-level}
    \end{subfigure}%
    \begin{subfigure}{.5\textwidth}
      \centering
      \includegraphics[scale=0.64]{../results/Percentiles_peso_Log.eps}
      \caption{log(Birth Weight)}
      \label{quintiles-log}
    \end{subfigure}
  \end{center}
  \floatfoot{\textsc{Notes to figure \ref{quintiles}}: Point estimates and 95\%
    confidence intervals are presented of the impact of Chile Crece Contigo on
    birth weight at different points of the birth weight distribution.  Each
    specification follows equation \ref{eqn:DD}, however instead of using mean birth
    weight in each municipality, uses birth weight at each percentile displayed
    on the horizontal axis as the dependent variable of interest.  Panel
    \ref{quintiles-level} displays the coefficients when using absolute
    birth weight at each quintile, while panel \ref{quintiles-log} using the
    logarithm of birth weight, so each estimate reflects the approximate
    percentage change.  For additional notes, refer to notes to Table \ref{mDD}.
  }
\end{figure}

\input{../results/motherFE.tex}

\begin{table}[htpb!]
  \caption{Spending on ChCC as a Portion of National Spending}
  \label{tab:spending}
  \begin{tabular}{llcc} \toprule
    Year & Spending & Spending & Percent \\
    & (ChCC)        & (National) & \\ \midrule
    2007 & 67,903,331  & 17,883,154,418 & 0.380 \\
    2008 & 126,446,362 & 20,650,579,217 & 0.612 \\
    2009 & 159,660,473 & 23,406,879,324 & 0.689 \\
    2010 & 214,505,550 & 25,651,969,793 & 0.836 \\
    \midrule
    \multicolumn{4}{p{8.1cm}}{{\footnotesize All values are displayed in 1000s of Chilean pesos. All national spending values are taken from the corresponding yearly budget, and ChCC spending
        is compiled from the ChCC final reports \citep{Arrietetal2013}.}} \\ \bottomrule
  \end{tabular}
\end{table}




\clearpage
\setcounter{table}{0}
\renewcommand{\thetable}{A\arabic{table}}
\setcounter{figure}{0}
\renewcommand{\thefigure}{A\arabic{figure}}
\pagenumbering{arabic}
\renewcommand{\thepage}{A\arabic{page}}
\appendix
\section*{Appendices}

\section{Appendix Figures and Tables}
\begin{figure}[htpb!]
  \begin{center}
    \centering
    \caption{Program Roll-out (early and late adopters)}
    \includegraphics[scale=0.41]{./figures/Grafico_ChCC_Final.pdf}
    \label{fig:map}
  \end{center}
  \vspace{-5mm}
  \floatfoot{\textsc{Notes to figure \ref{fig:map}}: Chile consists
    of 346 municipalities (``\emph{comunas}'') which are the lowest
    geographic administrative level.  ChCC roll-out started with 159
    municipalities in June 2007 (chosen due to the availability of
    infrastructure) and then was rolled out to the remaining municipalities
    in April of the following year.  Early adopters are marked in green
    and late adopters are marked in red.}
\end{figure}

\begin{landscape}
\begin{table}[htpb!]
  \begin{center}
    \caption{Alternative Specifications: Diff-in-diff Estimates of Program Impacts}
    \label{tab:AltSpecs}
    \begin{tabular}{lcccccccc} \toprule
      &(1)&(2)&(3)&(4)&(5)&(6)&(7)&(8)\\ \midrule
      \multicolumn{9}{l}{\textbf{Panel A: Birth Weight}} \\
      \input{../results/Alt_peso.tex}
      \multicolumn{9}{l}{\textbf{Panel B: LBW}} \\
      \input{../results/Alt_lbw.tex}
      \multicolumn{9}{l}{\textbf{Panel C: Size}} \\
      \input{../results/Alt_talla.tex}
      \multicolumn{9}{l}{\textbf{Panel D: Gestation}} \\
      \input{../results/Alt_gestation.tex}
      \multicolumn{9}{l}{\textbf{Panel E: Premature}} \\
      \input{../results/Alt_premature.tex}
      \multicolumn{9}{l}{\textbf{Panel F: Infant Mortality}} \\
      \input{../results/Alt_fDeathRate.tex}
      \midrule
      Municipal and Year FEs      & Y & Y & Y & Y & Y & Y & Y & Y \\
      Time-Varying Controls       &   & Y &   &   & Y &   &   & Y \\  
      Region Time Trends          &   &   & Y &   &   &   &   &   \\
      Region $\times$ Year FEs    &   &   &   & Y & Y &   &   &   \\
      Municipal Time Trends       &   &   &   &   &   & Y &   &   \\
      Municipal $\times$ Year FEs &   &   &   &   &   &   & Y & Y \\
      \bottomrule
      \multicolumn{9}{p{22cm}}{{\footnotesize \textsc{Notes to Table \ref{tab:AltSpecs}}:
          Each specification is estimated by differences-in-differences using
          municipal-level averages by month, and weights for the number of observations
          in each cell.  Column 1 replicates results from Table \ref{mDD}, and then
          columns 2-8 include additional controls, linear trends, or fixed effects.
          Regions in Chile are the top-level administrative district, of there are
          15.  Municipalities are within districts (analogous to states and counties
          in other countries), and there are 346 municipalities in Chile.  The
          most demanding specification allows for a separate fixed effect for each
          municipality in each year under study, given that there are twelve
          observations for each municipality in each year.  Time-varying controls are
          collected from the Government of Chile's National System for Municipal Information,
          and are available for each municipality in each year.  These controls consist
          of total transfers for education and health, the proportion of each municipality
          enrolled in the public health system (FONASA), the proportion enrolled in school,
          a pre-determined poverty index calculated by the Government, and the coverage
          of drinking water.  Standard errors are always clustered by Municipality. Refer
          to Table \ref{mDD} for additional notes.}} \\
    \end{tabular}
  \end{center}
\end{table}
\end{landscape}

\clearpage
\begin{table}[htpb!]
  \begin{center}
    \caption{Summary Statistics by Trimester: Birth and Chile Crece Contigo Data}
    \label{tab:sumstatsTri}
    \begin{tabular}{lccccc} \toprule
      & N& Mean & Std. Dev. & Min & Max \\ \midrule
      \input{../results/SummaryMunicipal-trimester.tex} \bottomrule
      \multicolumn{6}{p{15.2cm}}{{\footnotesize \textsc{Notes to Table
            \ref{tab:sumstatsTri}}: Summary Statistics are displayed for
          municipality by trimesterly averages for
          each trimester from January 2003 to December 2010.  Trimesters
          refer to January-March, April-June, July-September, and
          October-December.  For additional notes, refer to Table
          \ref{tab:sumstats}.
      }}
      \end{tabular}
  \end{center}
\end{table}

\begin{landscape}
\input{../results/comunaDD-trimester.tex}
\end{landscape}

\begin{table}
  \caption{Adjusting For Multiple Hypothesis Testing}
  \label{tab:MultHyp}
  \begin{center}
      \begin{tabular}{lcccccc} \toprule
        &  Index & \multicolumn{5}{c}{Original Variables} \\ \cmidrule(r){2-7}
        &Anderson & Birth & LBW & Birth & Weeks & Premature \\
        &Index    & Weight&     & Size  & Gestation & \\ \midrule
        \multicolumn{7}{l}{\textsc{Panel A: Municipal-Level Analysis}} \\
        $p$-value  (Original)    & \input{../results/MC_DD_porig.tex}
        $p$-value  (Corrected) & \input{../results/MC_DD_pRW.tex}
        &&&&&&\\
        \multicolumn{7}{l}{\textsc{Panel B: Individual-Level Analysis}} \\
        $p$-value  (Original)    & \input{../results/MC_FE_porig.tex}
        $p$-value  (Corrected) & \textbf{0.0479} & 0.0392 & 0.2549 & 0.0588 & 0.0196 & 0.7451\\
        \midrule
        \multicolumn{7}{p{14.8cm}}{{\footnotesize \textsc{Notes}: Corrected $p$-values based
            on original variables are calculated using the \citet{RomanoWolf2005} technique to
            control the Family Wise Error Rate of hypotesis tests. The \citet{Anderson2008}
            index converts the multiple dependent variables into a single dependent variable
            (index) giving more weight to variables which provide more independent variation.
            The specification of each regression follows Table \ref{mDD} (panel A), and
            \ref{mFE} (panel B).}}
        \\ \bottomrule
    \end{tabular}
  \end{center}
\end{table}


\clearpage
\section{Maternal Fixed Effects}
\label{MFE}
As a consistency check of the difference-in-difference results  
reported in the paper, we also undertake an analysis using the
full matched micro-data observing each mother's participation
status in ChCC.  Identification is driven by variation within
mother's exposure to the program over time over time.  We
estimate the following mother FE specification:
\begin{equation}
  \label{eqn:panel}
  Infant Health_{ijt} = \beta_0 + \beta_1 ChCC_{jt} + \bm{X_{ijt}\beta}_{x} + \phi_t + \mu_j + \varepsilon_{ijt}
\end{equation}
where $InfantHealth$ refers to the same measures of health at
birth as discussed in the body of the paper of child $i$ born
to mother $j$ at time $t$.

The matched administrative data allows us to construct a panel
of mothers and their children, and the independent variable of
interest in \ref{eqn:panel} is $ChCC_{jt}$.  This measures for
each mother at time $t$ whether she participated in Chile Crece
Contigo, and under typical (fixed effect) panel assumptions,
$\beta_1$ identifies the effect of participation on infant health.
We include maternal fixed effects $\mu_j$ and year fixed effects
$\phi_t$, as well as a series of time-varying controls for
mothers including birth order dummies, mother's age at birth
dummies, and child year of birth dummies\footnote{We are also
  able to control for a number of other individual-level
  covariates including maternal education, however in our main
  specification do not propose include this control given that
  ChCC explicitly aims to ensure that young mothers who are still
  enrolled in education finish their studies, and hence education
  is likely a bad control.  In supplementary analyses we augment
  the controls in \ref{eqn:panel} to examine the robustness of
  findings to alternative specifications.}.  Identification takes
advantage of the fact that there are mothers who (a) participated
in ChCC and had births both before and after the introduction of
the policy, and (b) never participated in the policy and also had
births both before and after the policy's introduction.

The matched mother and child data does not include the entire
universe of births (we do use the entire universe of births in
municipal-level regressions presented in the paper).  As such,
any estimated program impacts in the micro-level mother FE
specification are at best suggestive of the average effects in
the population.  When matching administrative data, approximately
50\% of births were matched with fathers, rather than mothers,
and in these cases we do not observe the mother's ChCC
participation status.  We thus restrict the analysis with
mother FE only to the population of children mathced with mothers,
noting that it is not a representative sample, and as such not
directly comparable to the municipal-level
difference-in-difference regressions presented in the paper based
on the entire universe of births.  Nevertheless, it acts as a
useful robustness check of the impact of ChCC based on different
identifying assumptions.\footnote{The two proposed strategies (the
  DD estimates in the body of the paper and the mother FE estimates
  in Appendices) are based on different identifying assumptions:
  strict (conditional) exogeneity for the family panel specification
  in equation \ref{eqn:panel} and parallel trends for the DD
  specification in equation  \ref{eqn:DD}.}
%and indeed, estimated effects should be diluted
%considerably at the municipal-level given that only a subset of
%a municipality enrolls in ChCC. However we can provide a
%rough comparison and consistency check of the effects if we inflate
%$\alpha_1$ to account for partial enrollment at the municipal level.
%We propose to conduct this comparison, following the methodology
%described in \citet{Almondetal2011} who conduct a similar adjustment
%in examining the roll-out of the Food Stamp Program at the municipal
%level in the United States.

\begin{figure}[htpb!]
  \begin{center}
    \centering
    \caption{Program Roll-out (early and late adopters)}
    \includegraphics[scale=1]{./../results/ChCCTrend.eps}
    \label{fig:trends}
  \end{center}
  \vspace{-5mm}
  \floatfoot{\textsc{Notes to figure \ref{fig:trends}}: Yearly averages are taken
    for all mothers who have ever participated in Chile Crece Contigo (although
    pre-2007 the program was not yet implemented) and all mothers who have never
    participated.  The vertical solid line indicates the beginning of the ChCC
    program.}
\end{figure}

\begin{table}[htpb!]
  \begin{center}
    \caption{Summary Statistics: Matched Mother, Child and Social Security Data}
    \label{tab:sumstatsMother}
    \begin{tabular}{lccccc} \toprule
      & N& Mean & Std. Dev. & Min & Max \\ \midrule
      \multicolumn{6}{l}{\textbf{Panel A: All Mothers}} \\
      \input{../results/SummaryMotherAll-update.tex}
      \multicolumn{6}{l}{\textbf{Panel B: Matched Mothers and Children}} \\
      \input{../results/SummaryMother-update.tex}
      \multicolumn{6}{l}{\textbf{Panel C: Unmatched Mothers and Children}} \\
      \input{../results/SummaryMotherNoChCC-update.tex}
      \bottomrule
      \multicolumn{6}{p{15.4cm}}{{\footnotesize \textsc{Notes to Table \ref{tab:sumstatsMother}}:
          Summary statistics are presented for all births matched with the mother's participation
          in social programs. Summary statistics are presented for all years from 2003-2010.
          \emph{Chile Crece Contigo} began in June of 2007, and so any mothers having all births
          prior to this date never participated in ChCC.  For additional notes on variable definitions
          and comparison with the full universe of births (collapsed by municipality) refer to Table
      \ref{tab:sumstats}.}}
      \end{tabular}
  \end{center}
\end{table}


\begin{figure}[htpb!]
  \begin{center}
    \caption{Birthweight Distributions Pre- and Post-Program Implementation}
    \label{dists}
    \begin{subfigure}{.5\textwidth}
      \centering
      \includegraphics[scale=0.64]{../results/Density_weightPre.eps}
      \caption{Birthweights Pre-ChCC}
      \label{predists}
    \end{subfigure}%
    \begin{subfigure}{.5\textwidth}
      \centering
      \includegraphics[scale=0.64]{../results/Density_weightPost.eps}
      \caption{Birthweights Post-ChCC}
      \label{postdists}
    \end{subfigure}
  \end{center}
  \floatfoot{\textsc{Notes to figure \ref{dists}}: Densities are plotted using
    an Epanechnikov kernel with a bandwidth of 5 grams.  Each panel separates
    distributions by whether the mother \emph{ever} participates in Chile Crece
    Contigo.  Panel (a) displays only pre-ChCC time periods, while panel (b)
    displays only post-ChCC time periods.  In both cases, Kolmogorov-Smirnov
    tests reject equality of distributions (in different directions).
  }
  \end{figure}

\input{../results/motherFE-controls.tex}
\clearpage

\section{Additional Program Details}
\label{app:program}
\begin{table}[htpb!]
  \caption{List of ChCC Policy Benefits}
  \label{ascn:benefits}
  \begin{tabular}{ll}
    \toprule
    Element (Program) & Benefit or Service \\ \midrule
    \multirow{4}{*}{Massive Education}
    &Weekly radio program ``Creciendo Juntos'' with national coverage\\
    &Chile Crece Contigo TV in waiting rooms of public health centres \\
    & ``Fono Infancia'' a free phone line providing support and information \\
    & Provision of children's books, and prenatal music CDs \\ \midrule
    \multirow{6}{*}{Socio-emotional Development}
    & Gestation and birth guide ``beginning to grow'' \\
    & Prenatal care protocol: check-ups  \\
    & Prenatal care support: fortified food, information\\
    & Educational support for pregnant mother and partner (4 sessions) \\ 
    & Personalised birth support, favouring rapid skin contact with mother \\ 
    & Integral puerperal and breastfeeding support \\ \midrule
    \multirow{6}{*}{Newborn Support Program}
    & Pack for safe attachment and clothing (multiple goods) \\
    & Pack for basic care and stimulation (mulitple goods) \\
    & Pack with crib/corral (multiple goods) \\
    & Integral care for newborn in neonatal and pediatric units \\
    & Regular health controls focused on integral development \\
    & Health controls for vulnerable or developmentally delayed children \\ \midrule
    \multirow{10}{*}{Means-Tested Elements}
    & Access to technical support for children with any type of disability \\
    & Guaranteed free access to ``sala cuna'' \\
    & Guaranteed free access to nursery school \\
    & Guaranteed access to ``Chile Solidario''\\
    & Support to finish education (mothers) \\
    & Support for labour market insertion (families) \\
    & Improvement of living conditions (families) \\
    & Mental Health Attention \\
    & Family dynamic attention (focused on domestic violence)\\
    & Judicial support \\  \bottomrule
    %\multicolumn{2}{p{15cm}}{{\footnotesize \textsc{Notes}}}
  \end{tabular}
\end{table}


\section{Data Agreement}
\label{app:agreement}
We provide the full confidentiality agreement between the principal investigator
of this project (Clarke) and the Ministry of Social Development (previously known
as Ministry of Planification) and the Ministry of Health.  This agreement is
displayed in full overleaf.
\\


\includepdf[pages=-,pagecommand={},width=\textwidth]{confidencialidad.pdf}

\clearpage
\section{Broader Context on Birth Outcomes and Maternal Characteristics in Chile}
\label{app:context}
Following the return to democratic rule in 1990, full microdata on
all births in Chile has been available from the Ministry of Health's
Department of Statistics and Health Information (DEIS).  These
vital statistics include each child's birth weight, weeks of gestation,
and a number of characteristics of the mother and father (when the
father is present).  This data is recognised to be of high quality
and very close to universal (see for example \citet{Mikkelsenetal2015}).

The average age of mothers in Chile has risen from slightly over 26 in
1990, to slightly under 28 in 2015 (Figure \ref{trendlongMother}).  The
average age of mothers increased constantly from 1990 until approximately
2004, before falling slightly, and ascending once again from 2009 onwards.
This reduction in maternal age occurred during a considerable slow-down
in growth, and an uptick in the nubmer of births each year (Figure
\ref{NBirths}), in line with results suggesting countercyclicality in
fertility.  Panel b of Figure \ref{trendlongMother} displays the proportion
of teenage births (among all births), which rose until the early 2000s,
began to fall until the growth slowdown in the mid-2000s, and has fallen
sharply from 2007.
\begin{figure}[htpb!]
  \begin{center}
    \caption{Trends in Maternal Characteristics in Chile}
    \label{trendlongMother}
    \begin{subfigure}{.5\textwidth}
      \centering
      \includegraphics[scale=0.64]{../results/trends/edad_m_year.eps}
      \caption{Mother's Age}
      \label{trendMAge}
    \end{subfigure}%
    \begin{subfigure}{.5\textwidth}
      \centering
      \includegraphics[scale=0.64]{../results/trends/teenbirth_year.eps}
      \caption{Proportion of Teenage Births}
      \label{placebo-lbw}
    \end{subfigure}
  \end{center}
  \floatfoot{\textsc{Notes to figure \ref{trendlongMother}}: Yearly averages of
    age and the proportion of all mothers aged under 20 years of age based on
    Ministry of Health (DEIS) microdata covering all births in Chile between
    1990 and 2015.
  }
\end{figure}

\begin{figure}[htpb!]
  \begin{center}
    \caption{Number of Births per Year}
    \label{NBirths}
    \includegraphics[scale=1]{../results/trends/N_year.eps}
  \end{center}
\end{figure}

We display descriptive plots of average birth outcomes across time in
figure \ref{trendlong}.  These indicators, particularly birth weight,
improved sharply following the transition to democracy in the early
1990s, and the implementation of a considerable public health reform.
Average birth weight increased by more than 60 grams, and the proportion
of low birth weight babies fell by a full percentage point (refer to
panels \ref{trendBW} and \ref{trendLBW}).  From the year 2000 onwards,
average outcomes have gradually worsened, in line with increases in
maternal age.

\begin{figure}[htpb!]
  \begin{center}
    \caption{Longer Term Trends in Birth Outcomes in Chile}
    \label{trendlong}
    \begin{subfigure}{.5\textwidth}
      \centering
      \includegraphics[scale=0.64]{../results/trends/peso_year.eps}
      \caption{Birth Weight}
      \label{trendBW}
    \end{subfigure}%
    \begin{subfigure}{.5\textwidth}
      \centering
      \includegraphics[scale=0.64]{../results/trends/lbw_year.eps}
      \caption{LBW}
      \label{trendLBW}
    \end{subfigure}
    \begin{subfigure}{.5\textwidth}
      \centering
      \includegraphics[scale=0.64]{../results/trends/semanas_year.eps}
      \caption{Gestation}
      \label{trendGest}
    \end{subfigure}%
    \begin{subfigure}{.5\textwidth}
      \centering
      \includegraphics[scale=0.64]{../results/trends/premature_year.eps}
      \caption{Prematurity}
      \label{trendPrem}
    \end{subfigure}
  \end{center}
  \floatfoot{\textsc{Notes to figure \ref{trendlong}}: Yearly averages of
    birth weight, the proportion of low birth weight births ($<$ 2500 grams),
    weeks of gestation, and the proportion of premature births ($<37$ weeks)
    from Ministry of Health (DEIS) microdata covering all births in Chile
    between 1990 and 2015.
  }
\end{figure}


\end{spacing}
\end{document}


















%\section*{One Page Summary}
%\noindent \textbf{Research Question}: What is the impact of participation in
%a large national-level prenatal and early-life social safety net program
%(Chile Crece Contigo) on infant health and survival in marginalised groups?
%
%\noindent \textbf{Data Abstract}: High-quality vital statistics data 
%are publicly available in Chile since the 1990s.  As well as reporting all
%births and deaths occurring in the country, indicators of health at birth
%including birth weight, gestation, and length at birth are available.  We have
%worked with the Ministries of Social Development and Health to merge all vital
%statistics data from 2000 to 2010 with an indicator of whether each mother
%participated in Chile Crece Contigo when her child was \emph{in utero}.  We
%are also able to match each child record to mortality data to observe whether
%the child survived the first year of life.  This results in a newly-created
%microdata set consisting of 2,651,075 births to 1,541,514 mothers, where all
%siblings are fully linked within mothers.  Although this database is private
%due to sensitive identifying information used to link children and mothers
%(national identity numbers), we will share a fully anonymised version of the
%dataset to permit replication and examination of our results.
%
%\noindent \textbf{Methodology Abstract}: We propose to use a number of
%characteristics of the data and program roll-out to identify the impact
%of participation in Chile Crece Contigo on neo-natal health outcomes.
%Firstly, given the sibling-level variation in births and program use,
%we propose to estimate a within-mother specification, controlling for
%age at birth and birth order fixed effects, to determine the
%micro-level impact of parental participation in the program on child
%health.  Secondly, we propose to use municipal variation in program
%roll-out to estimate a difference-in-difference (DD) specification.
%Chile Crece Contigo was piloted in 159 of Chile's 346 municipalities
%starting in June of 2007, before being rolled out to the remaining
%municipalities in April of 2008.  This temporal and spatial variation
%in coverage provides the identifying variation to examine the population-%
%level impacts of ChCC, as well as acting as a consistency check for the
%between-sibling regressions described above.  Finally, using the
%estimates from both parts, we propose to conduct a back-of-the-envelope
%calculation to determine the social willingness to pay for neonatal
%health.  As spending on ChCC currently exceeds 300 million USD per year,
%and over its life ChCC accounts for approximately 0.5\% of the national
%budget, efficiency considerations in this targeted health care program 
%are non-trivial.

\newpage
\section{Introduction}
The importance of early life health over the entire life course of an
individual has been extensively recognised in the economic (and non-economic)
literature \citep{Almondetal2017,AlmondCurrie2011,Barker1990}.  This has
lead to considerable investments in fetal and infant health in a wide range
of contexts (see, among many others, discussion in \citet{BitlerKaroly2015}
with reference to the USA and \citet{Bharadwajetal2013} applied to Chile).

An important motivation of these early-life health policies owes
to the dynamic complementary between the efficiency of investments in health
early in life and investments later in life.  In an influential series of
papers, \citet{HeckmanCunha2007,CunhaHeckman2009,Cunhaetal2010} argue that
early-life remedial investments are not only efficient, but equality promoting.
This has lead, at times explicitly, to policies targeting early-life health
outcomes as a basic column of the social safety net across the developed and
the developing world.\footnote{For example,
  in the context of Chile Crece Contigo---the policy we propose to examine in
  this paper---the policy's design explicitly references
  \citet{HeckmanCarneiro2003} as support for the implementation of a
  large-scale early-life social program (see the official materials in
  \citet{Arrietetal2013}).}

Such early-life health policies are particularly important in the context
of Latin America. Many Latin American countries are characterised by
irregular, rather than universally poor, infant health outcomes
\citep{Belizanetal2007}.  Indicators are particularly sub-standard among
socially isolated groups, including
low-income households, rural communities, and indigenous people.  These
early-life health differentials are only magnified over the life course
of individuals, partially explaining the emergence of significant gaps in
adulthood in education, salary, and morbidity and mortality.  This has been
documented in the Chilean context, where divergence of outcomes at a very
young age (birth weight) have important effects on adacemic achievement
up to 18 years later \citep{Bharadwajetal2017}.

The importance of investment in health---and early-life health in
particular---as a driver of individual and national outcomes in the
developed world has been flagged in various dimensions.  Influential
work points to the importance of health as a determinant of equality
within countries \citep{Deaton2003}, and document the long-shadow of
early life insults to health in the developing world
\citep{CurrieVogl2012}.   The social determinants of health starting
\emph{in utero} have lead to the recent design and implementation of
many large, targeted early life social safety-net programs throughout 
Latin America and the developing world. \citep{Monteiroetal2015}.


In this paper we propose to estimate the impact of one such large-scale,
national-level health program explicitly designed to target early-life
health in vulnerable groups.  This program: \emph{Chile Crece Contigo}
(hereafter ChCC) was implemented in Chile in 2007, offering a basket
of services, information and basic supplies to all children enrolled
in the public health system.\footnote{The Chilean health system consists
  of a private and public stream and users nominally choose between public or
  private care. An associated monthly payment is automatically deducted
  from all formal salaries as a previsional payment.  This payment is
  either made to the public health insurance (FONASA) or a private health
  insurer known as an ISAPRE.  Any individual unable to pay contributions
  is covered by the public FONASA system.  The private system is
  consderably more costly in terms of out of pocket costs.  Recent
  administrative data suggests that 76\% of the population is covered
  by public care.  Nationally, 67\% of beds are in the public system and
  the remaining 33\% are in the private system \citep{DEIS2016}.}  As well
as a transversal series of benefits, an additional series of means-tested
benefits were provided to families classified as part of the 60\% of most
vulnerable families, with additional benefits for those who were classified
as part of the 40\% of most vulnerable families.  ChCC also has a stated
aim of addressing divergent health outcomes in socially excluded groups,
releasing materials in both Spanish and native indigenous languages, given
the well-documented health disparities among indigenous people across the
world, and in Chile \citep{Andersonetal2016}.\footnote{Chile's population
  is 4.58\% indigenous, the majority of whom are Mapuche, and this group
  has been documented as having poorer birth, neonatal and child health
  outcomes \citep{Andersonetal2016}.}\textsuperscript{,}\footnote{In order to provide an idea of the
program's scope, we provide a brief list of program benefits in appendix
\ref{ascn:benefits} of this paper.}

\begin{figure}[htpb!]
\begin{center}
  \centering
  \caption{Timing of Program Rollout and Coverage}
  \includegraphics[scale=1]{./figures/ChCCcover.eps}
  \label{fig:coverage}
\end{center}
\vspace{-5mm}
\floatfoot{\textsc{Notes to figure \ref{fig:coverage}}: The first
  program rollout occurred in June of 2006, with the remainder of
  municipalities joining in the second rollout in Abril of 2007.
  All coverage figures are based on ChCC administrative records.
  %The program continued after 2010 and is due to do so indefinitely,
  %however we only consider the period up until 2010 given that our matched
  %birth data end in this year.
}
\end{figure}

The ChCC program was rolled-out progressively from June of 2007.  In
the first year the program covered 159 of Chile's 346 municipalities,
before being extended to all municipalities in mid-2008.  Program
participation among pregnant women also increased in line with geographic
coverage.  In figure \ref{fig:coverage} we plot administrative figures of
program usage over the life of the program.  In 2010 the program covered
greater than 200,000 pregnancies nationwide.\footnote{According to vital
  statistics data, in 2010 there were 250,643 live births.  We note
  however that a pregnancy will be contained in the ChCC administrative
  data plotted in figure \ref{fig:coverage} even if the mother miscarries
  or a fetal death is recorded.}  In terms of total cost, ChCC is one of
the largest health programs in Chile.  Recent figures (which we discuss
in slightly more detail in section \ref{sscn:spending} of this proposal)
suggest that ChCC spending currently accounts for almost 1\% of the
national budget.

Despite the size and scope of ChCC, few rigourous or well identified
studies have been conducted on the program's effectiveness, and none,
as far as we are aware,
have examined the policy's effect on birth outcomes or survival
during the first year.%\footnote{As well as searching the economics
 % literature, we conducted a search on PubMed using the keywords
 % ``Chile Crece Contigo'' AND ``Child Health'' OR ``Chile Crece Contigo''
 % AND ``Neonatal Health'' OR ``Chile Crece Contigo'' AND ``Birth weight''
 % OR ``Chile'' AND ``Neonatal Health'' over the span 2006 to 2014
% (ie after the design of the program).}
In this we take
advantage of newly matched administrative data to conduct the
first such study, drawing identification from two (different)
sources: the first, the within-mother variation in exposure produced
across siblings around the date of the policy's introduction, and 
the second, the geographic and time variation in municipal-level
participation.  The first strategy allows us to estimate a mother
fixed effects model (at the individual level), and the second a
difference-in-difference specification at the municipal level.

Our results suggest that this program has considerable effects on
neonatal health in Chile.  Depending on the specification examined,
we find that the effect of program participation on birth weight
is between a 12 and 13 gram increase, with this being particularly
important when considering the reduction of the proportion of low birth
weight babies.  Similarly, we observe a reduction in the frequency
of premature births.  Given the large coverage of the ChCC program,
these results are noteable in national level averages, and appear to
\emph{eliminate} the birth weight differential between the poorer
program participants and the less-poor non-participants. 

To put the program's effects in context, we calculate the inferred cost
of producing a gram of birth weight, and the implications of this to
educational attainment later in life.  When combined with the cost of
running Chile Crece Contigo, our estimates suggest that the government
spends around \$18 per gram of birth weight---a figure that is comparable
to other large neonatal health programs, even in developed countries
(such as the US Food Stamp Program).  What's more, given the well-known
positive effects of birth weight on later life outcomes, we estimate that
as an \emph{upper bound} cost, each \$1200 spent on Chile Crece Contigo
results in an additional 0.1 standard deviation of educational attainment
on later life test scores.  These results suggest that targeted public
health and social welfare programs can have large impacts in developing
and emerging economies, and that these impacts may last much longer than
the period in which an individual is enrolled in the program.

In what remains of this paper we briefly describe the newly
generated data that we will work with which provides universal
coverage of births and ChCC usage, discuss the proposed
estimation strategies to determine the impact of ChCC on neonatal health,
discuss estimated results, and in closing estimate the efficiency of
public spending on this program, benchmarking against other public
neonatal health programs, as well as the estimated value of imporvements in
health at birth in Chile.


\section{Data}
We have worked with the Chilean Ministry of Social Development
and Ministry of Health (MDS and MS respectively for their initials
in Spanish) to link administrative data on all births in the
country with an indicator of whether each mother was enrolled in
ChCC during pregnancy (as well as the vulnerability score of
each mother, which impacts the degree of benefits they will
receive).  As each person in Chile has a unique national identity
number, this has been used to link mothers and children between
administrative databases.  Given data privacy laws in Chile we have
signed a confidentiality declaration to protect all individual
level data with identifying features, however are able to release
anonymised registers at
the micro-level (refer to appendix \ref{ascn:agreement} for further
details).  The resulting data set is a unique universal source
of information on births in Chile which will allow us to estimate
the impact of the program, unlike other data sources on ChCC which
are small and do not cover pre- and post-reform time periods.

We match all births occurring between 2002 and 2010 with their
siblings, and, from 2007 onwards, whether the mother participated
in ChCC during gestation.  This results in a sample of 1,917,085
births occurring to 1,241,514 mothers. Of these births, 32.6\%
of mothers participated in ChCC for at least one of their births.
Vital Statistics data in Chile covers greater than 99\% of all
births, and coverage is stable over time.  We focus on the period
of 2001 to 2010 in order to have a sufficient pre-ChCC and post-ChCC
window for analysis, though below discuss a number of consistency checks
we run using a shorter pre-program window. Further deatils on
the Chilean Vital Statistics Data can be found in
\citet{Bharadwajetal2013}.

This data allows us to observe a range of human capital measures
at birth.  These include the weight of the baby, the baby's length
in centimetres, and the gestational length as recorded at birth. These
measures have been consistently shown to have large and long-lasting
effects on health and well-being \citep{AlmondCurrie2011b}.
Although Apgar and head circumference are measured at birth in
Chile, they are not currently available in administrative data.
Along with measures of health immediately at birth, we are able
to follow babies over 1 year of life to observe their survival
at one year.  Using the same unique national identity number
which is assigned at birth, we can match each child in the birth
register with any deaths under 1 year of age in the mortality
register in order to measure infant mortality.

\begin{table}[htpb!]
  \begin{center}
    \caption{Summary Statistics: Birth and Chile Crece Contigo Data}
    \label{tab:sumstats}
  \begin{tabular}{lccccc} \toprule
    \input{./../results/SummaryIndividual-clean.tex}
    \multicolumn{6}{l}{\textbf{Panel B: Municipal-Level Data}} \\
    \input{./../results/SummaryMunicipal-clean.tex} \bottomrule
    \multicolumn{6}{p{14cm}}{{\footnotesize \textsc{Notes}: All
        births from 2003
      to 2010 are included in the estimation sample.  Panel A
      presents individual-level statistics for all births.  Birth
      weights greater than 5,000 grams or less than 500 grams are
      removed from the sample, as are reported gestational times
      of less than 25 weeks or greater than 45 weeks.  Panel B
      presents municipal level averages collapsed to municipality
      and month$\times$ year cells.  ALl municipalities which
      have at least one birth in a given month have an observation
      (there are 345 municipalities in Chile).  The number of births
      in each cell is presented in the last row.
    }}
  \end{tabular}
  \end{center}
\end{table}

In table \ref{tab:sumstats} we provide summary statistics of
principal health indicators at birth, as well as rates of participation
in Chile Crece Contigo by mothers.  Panel A documents full micro-level
data, largely in agreement with values observed in vital statistics
data observed else where.  The average birth weight in the population
is approximately 3,300 grams, geation is on average 38.6 weeks, and
6 and 7\% of births are low birth weight or premature (respectively).
In administrative data from 2003 to 2010 15\% of mothers are observed
to ever participate in Chile Crece Contigo, though this value is
considerably lower than actual participation rates once the program
was implemented, as the program only began running from June of 2007
onwards (ie births ocurring in 2008 and onwards).  In panel B we
provide similar summary statistics, however now using averages in
municipal by month cells.  In Chile there are 345 municipalities
(the third level administrative district), and as we discuss further
below, using municipal-level variation in program roll-out we can
estimate the effects of Chile Crece Contigo on average birth outcomes.
In general municipal level averages line up with individual level
data.  In the final line of table \ref{tab:sumstats} we observe
that there is considerable variation in the size of municipalities.
Depending on the municipality, the number of births per months ranges
from as low as 1 birth (conditional on there not being 0 births) to
as high as 787 births.


In figure \ref{dists} we examine the full distribution of birth weights
split by those who eventually participated in ChCC and those who never
participated.  These are presented entirely \emph{before} the program's
implementation in figure \ref{predists}.
In this figure we observe that the distribution of birth weights for
those who eventually used ChCC (the solid line) was slightly lower
than the corresponding distribution for those who never used ChCC.
These distributions are statistically distinguishable using traditional
tests, and is particularly noticeable between approximately 2,500 and
3,000 grams.  We document similar distributions, however now based on the
post-ChCC time period in figure \ref{postdists}.  In this case we now
observe the reverse pattern: the ChCC user distribution appears to be
shifted to the right of the non-ChCC distribution, suggesting that
babies born to ChCC participants now have \emph{better} neonatal health
measures than non-participants.

\begin{figure}[htpb!]
  \begin{center}
    \caption{Birthweight Distributions Pre- and Post-Program Implementation}
    \label{dists}
    \begin{subfigure}{.5\textwidth}
      \centering
      \includegraphics[scale=0.54]{./figures/Kpesoprechcc.eps}
      \caption{Birthweights Pre-ChCC}
      \label{predists}
    \end{subfigure}%
    \begin{subfigure}{.5\textwidth}
      \centering
      \includegraphics[scale=0.54]{./figures/Kpesopostchcc.eps}
      \caption{Birthweights Post-ChCC}
      \label{postdists}
    \end{subfigure}
  \end{center}
  \floatfoot{\textsc{Notes to figure \ref{dists}}: Densities are plotted using
    an Epanechnikov kernel with a bandwidth of 5 grams.  Each panel separates
    distributions by whether the mother \emph{ever} participates in Chile Crece
    Contigo.  Panel (a) displays only pre-ChCC time periods, while panel (b)
    displays only post-ChCC time periods.  In both cases, Kolmogorov-Smirnov
    tests reject equality of distributions (in different directions).
  }
\end{figure}

Finally, in figure \ref{fig:trends} we examine time series of average birth
weights for the same two groups. We display
a vertical solid line to indicate the first roll-out period of ChCC, meaning
that expected impacts should be noted only  $\sim$9 months following this
point.  Using yearly averages it does appear that there is an increase in
birth outcomes among users following the reform, and approximately parallel
trends in the pre-reform period.  These illustrative trends are suggestive
that Chile Crece Contigo may have had significant impacts on health at
birth, which is something that we go on to test more formally in the following
sections.

\begin{figure}[htpb!]
\begin{center}
  \centering
  \caption{Program Roll-out (early and late adopters)}
  \includegraphics[scale=1]{./../results/ChCCTrend.eps}
  \label{fig:trends}
\end{center}
\vspace{-5mm}
\floatfoot{\textsc{Notes to figure \ref{fig:trends}}: Yearly averages are taken
  for all mothers who have ever participated in Chile Crece Contigo (although
  pre-2007 the program was not yet implemented) and all mothers who have never
  participated.  The vertical solid line indicates the beginning of the ChCC
  program.}
\end{figure}


\section{Methodology}
\subsection{Estimating the Impacts of ChCC}

Given the rich data available, we propose to follow two estimation
strategies to take advantage of different identifying features
inherent in data and the implementation of ChCC. The first is an
individual-level specification using variation within mothers
over time.  We propose to estimate:
\begin{equation}
  \label{eqn:panel}
  Infant Health_{ijt} = \beta_0 + \beta_1 ChCC_{jt} + \bm{X_{ijt}\beta}_{x} + \phi_t + \mu_j + \varepsilon_{ijt}
\end{equation}
where $InfantHealth$ refers to a measure of health at birth of
child $i$ born to mother $j$ at time $t$.  We will construct
a panel of mothers and their children, and our variable of
interest is $ChCC_{jt}$.  This measures for each mother at
time $t$ whether she participated in Chile Crece Contigo, and
under typical (fixed effect) panel assumptions, $\beta_1$
identifies the effect of participation on infant health.  We
include maternal fixed effects $\mu_j$ and year fixed effects
$\phi_t$, as well as a series of time-varying controls for
mothers including birth order dummies and mother's age at birth
dummies.  Identification takes advantage of the fact that there
are mothers who (a) participated in ChCC and had births both
before and after the introduction of the policy, and (b) never
participated in the policy and also had births both before and
after the policy's introduction.  As well as estimating this
specification with our full data (2000-2010), we will run a
number of consistency checks using tighter windows to ensure
that results aren't driven by children born at very different
maternal ages or birth orders, as well as augment mother
time-varying controls $\bm{X_{ijt}}$ to include age at birth
and birth order fixed effects for both ChCC participants and
non-participants separately.\footnote{We are also able to
  control for a number of other individual-level covariates
  including maternal education, however in our main specification
  do not propose to include this control given that ChCC
  explicitly aims to ensure that young mothers who are still
  enrolled in education finish their studies, and hence
  education is likely a bad control.
}

Our principal outcome measures of $InfantHealth$ consist of
birth weight in grams, low birth weight ($<$2,500 grams),
birth length in centimetres, gestation time in weeks, prematurity
($<$ 37 weeks gestation), and infant mortality.  Given that
we propose to use various outcome measures and a single
indepedent treatment variable we will correct for multiple
hypothesis testing.  We briefly return to this point in the
following subsection.

Our second strategy is a traditional difference-in-differences (DD)
model in which we take advantage of the time-varying roll-out
of the policy by geographic area.  As discussed above, ChCC
was first implemented in 159 municipalities before later being
implemented across the entire country.  If ChCC had a significant
effect on early-life child health outcomes in socially excluded
groups we should see that outcomes first improve in the early
treatment municipalities, and only later improve in the late
adopting areas.  We thus propose to estimate:
\begin{equation}
  \label{eqn:DD}
  Infant Health_{ct} = \alpha_0 + \alpha_1 ChCC_{ct} + \bm{W_{ct}\alpha}_{w} + \phi_t + \lambda_c + \eta_{ct}
\end{equation}
where $InfantHealth$ is now an average for each municipality $c$ in
year $t$.  The variable $CHCC_{ct}$ is a binary treatment measure
indicating if the program was available in the municipality 40
weeks prior (to account for gestation), and we include full municipality
and year fixed effects.  If implementation of the policy were
completely random, $\alpha_1$ should give us the unbiased effect
of ChCC on infant health measures.  However, as we may be concerned
that early adopting municipalities with better infrastructure
were following different trends over time, we propose to include
a series of time-varying controls for health infrastructure $\bm{W_{ct}}$,
and in supplementary regressions also examine the robustness of
results to municipal-specific linear time trends.  Despite these
considerations, we note that there is no particularly notable
geographic clustering of early- and late-adopted municipalities,
even within metropolitan areas such as Santiago (refer to figure
\ref{fig:map} overleaf to observe the variation in roll-out by area). As is
typical, we will cluster standard errors by municipality (346 municipalites)
to account for the well-known time-dependence in unobserved stochastic
errors by geographic area \citep{Bertrandetal2004,CameronMiller2015}.

These two proposed strategies are based on different identifying
assumptions\footnote{Strict (conditional) exogeneity for the family
  panel specification in equation \ref{eqn:panel} and parallel trends
  for the DD specification in equation  \ref{eqn:DD}.},
and indeed, estimated effects should be diluted
considerably at the municipal-level given that only a subset of
a municipality enrolls in ChCC. However we can provide a
rough comparison and consistency check of the effects if we inflate
$\alpha_1$ to account for partial enrollment at the municipal level.
We propose to conduct this comparison, following the methodology
described in \citet{Almondetal2011} who conduct a similar adjustment
in examining the roll-out of the Food Stamp Program at the municipal
level in the United States.

\begin{figure}[htpb!]
\begin{center}
  \centering
  \caption{Program Roll-out (early and late adopters)}
  \includegraphics[scale=0.41]{./figures/Grafico_ChCC_Final.pdf}
  \label{fig:map}
\end{center}
\vspace{-5mm}
\floatfoot{\textsc{Notes to figure \ref{fig:map}}: Chile consists
  of 346 municipalities (``\emph{comunas}'') which are the lowest
  geographic administrative level.  ChCC roll-out started with 159
  municipalities in June 2007 (chosen due to the availability of
  infrastructure) and then was rolled out to the remaining municipalities
  in April of the following year.  Early adopters are marked in green
and late adopters are marked in red.}
\end{figure}

\subsection{Inference, Robustness Strategies and Extensions}
In order to run a consistency check on DD results we propose
to estimate a full event study.  This event study is a test in the style
of \citet{Granger1969}.  We will examine precisely when indicators
diverge between early and late treatment areas, estimating the following
specification:
\begin{eqnarray}
  \nonumber
  Infant Health_{ct} &=& \gamma_0 + \sum_{k=1}^6 \gamma^{lead}_kChCC_{c}\times 1\{Year=-k\}  \\ &&+\sum_{l=1}^4 \gamma^{lag}_lChCC_{c}\times 1\{Year=+l\} \phi_t + \lambda_c + \nu_{ct}.
\end{eqnarray}
Here we interact a series of indicator variables indicating policy
implementation leads ($-k$) and lags ($+l$).  If the difference between
early and late implementing states only emerge when the policy is
implemented, there should be no differential impact in any of the
lead terms, suggesting an individual and joint test that each of the
$\gamma^{lead}$ terms are equal to zero.

Secondly, in order to ensure adequate size in all hypothesis testing,
we will correct for mulitple testing in each of models \ref{eqn:panel}
and \ref{eqn:DD}.  As described previously, we will examine 6 dependent
variables in each case.  In naive regressions we will be considerably
more likely to reject null hypotheses at a fixed level given that we
are conducting multiple tests. As such, we will
apply \citet{RomanoWolf2005}'s stepdown hypothesis testing algorithm
which fixes the Family Wise Error Rate at a set level $\alpha$.  This
hypothesis correction technique is considerably more powerful than
older techniques such as Bonferroni or Holm, and is increasingly used
in the social science literature (see for example \citet{Gertler2014}).
Similarly, it is more correct than setting False Discovery Rates at a
fixed level given the relatively small number of multiple tests.  We
will employ the algorithm available in \citet{Clarke2016}.

%\subsection{Distributional Effects of the Policy}
Finally, along with regressions examining birth weight, and low birth
weight we are able to observe the effects of the policy over the entire
range of the birth weight distribution, to examine precisely where
effects are observed (if effects are observed).  In order to do so
we propose to estimate specifications \ref{eqn:panel} and \ref{eqn:DD}
using quantile regression.  We will do this for the various
(approximately) continous outcomes available, namely birth weight,
length at birth, and gestational length.



\section{Results}
\subsection{Individual-Level Estimates with Mother Fixed Effects}
Estimates based on mother fixed effects are presented in table 2.
We present fixed effects estimates in each case also controlling
for mother's age and birth parity fixed effects which may vary
around the reform date.  Identification is thus driven by changes
in birth outcomes between siblings born before and after their
mothers began participating in Chile Crece Contigo, compared with
similar time siblings occurring to never-participating mothers.
In each case we cluster standard errors by mother to account for
the correlation of stochastic errors within a family over time.

Results are presented for birth weight, the likelihood that a birth
is low birth weight (less than 2,500 grams), total gestational
length, and the likelihood of prematurity (birth at less than 37
weeks).  Both LBW and prematurity are frequently used measures,
and associated with a range of poor health outcomes later in life
\citep{Almondetal2005}.  We observe large signficant effects on
birth weight and on gestational lengths, suggesting that participation
in Chile Crece Contigo impacted these two outcomes.  However,
we do not observe significant effects on low birth weight or
the probability of being born prematurely in this specification.

%1234694 mothers

\input{./../results/motherFE.tex}
%\input{./../results/motherFE-group.tex}

The effect sizes observed for birth weight and gestational weeks
are considerable.  An effect of 13 grams is equivalent to approximately
0.5\% of the mean birth weight in Chile over the time period
examined, and similar to the reported effects of large successful
programs world wide.  For example, recent evidence suggests that
participation in the Food Stamp Program in the United States, one
of the largest and most costly social security programs, increases
birth weight by approximately 20 grams \citep{Almondetal2011}.
Similarly, participation in the supplementary nutrition program
for Women, Infants and Children is estimated to increase birth weight
by around 17-30 grams \citep{Hoynesetal2011,RossinSlater2013}. 

These first results suggest that targeted social-security programs
can have considerable effects on early-life human capital in an
emerging country context.  Below we turn to an alternative estimation
strategy and a series of placebo and consistency checks to examine
the plausibility of these estimates.

\subsection{Municipal-Level Estimates}
Estimates based on municipality and time-varying exposure to the
Chile Crece Contigo program are presented in table 3.  As described
in the methodology section, identification is driven by the
differential roll-out to pilot (159) and non-pilot municipalities
(186), with municipal specific factors being captured by municipality
fixed effects.  Given that roll-out was timed by month, we generate
municipal by month averages for birth weight, LBW, gestation and
prematurity, as well as counts of the total number of births, which
we use to weight our principal specifications.

Results from table 3 once again suggest large and significant effects
of Chile Crece Contigo from this alternative identification strategy.
The estimates on birth weight (column 1 without maternal age and
parity controls, or column 2 with these controls) suggest that after
ChCC's arrival, average birth weights at the level of the municipality
increased by 7.5 grams.  While this is smaller than the value estimated
in table 2, this is expected given that only a proportion of each
municipality is ever enrolled in the program.  If we follow
\citet{Almondetal2011} and inflate estimates using participation
rates to arrive at an approximate individual-level estimate, this
suggests values of approximately 12 grams\footnote{This is calculated
  using average municipal participation rates of 65.1\% among pregnant
  women, and so for the birth weight result in column 1 the estimated
  inflated effect is given as $7.695/0.651=11.82$ grams.}, in quite
close agreement with our mother fixed effects estimates displayed in
the previous subsection.

In turning to weighted municipal-level estimates we \emph{do} observe
significant impacts on the frequency of low birth weight births and
premature deliveries.  For LBW babies, we see a reduction of 0.5 percentage
points, which is equivalent to nearly a 10\% reduction.  Similarly,
with premature deliveries we see a reduction of 0.4 percentage points,
or approximately a 7\% reduction in these pregnancies after the
implementation of ChCC.

We examine the plausibility using a series of placebo tests.
These placebo tests use the same early and late municipality groups,
however assigning the placebo reform date to a period entirely
\emph{before} the arrival of Chile Crece Contigo.  Thus, if there
is no general prevailing difference between the two groups of
municipalities we should observe that all placebo tests based on
pre-reform dates lead to insignificant estimates of the effect of
the placebo treatment on birth outcomes.

These results are displayed in figure \ref{placebo}.  Each point
estimate and confidence interval corresponds to a placebo reform
starting in the early municipalities at the date displayed on the
$x$-axis, and rolling out to the remaining municipalities 9 months
later (as occurred in the true reform).  We use all time periods
for which full coverage is available, until arriving to the date
of the true reform (indicated by the red dotted line in the figures).
We observe that in nearly all cases placebo reforms lead to smaller
and statistically insignificant estimates.  In the case of tests
using birth weight, we observe one statistically significant result
in the placebo tests (of approximately 40), and none when examining
low birth weight.  The effects of the reform only begin to be
observed when approaching the true reform date, reaching their
maximum estimates when the true reform dates are used.


\begin{landscape}
\input{./../results/comunaDifDif.tex}
\end{landscape}

\begin{figure}[htpb!]
  \begin{center}
    \caption{Placebo Tests for Municipal-Level Difference-in-Differences}
    \label{placebo}
    \begin{subfigure}{.5\textwidth}
      \centering
      \includegraphics[scale=0.54]{./../results/placebopeso_gest.eps}
      \caption{Birth weight placebo}
      \label{placebo-bwt}
    \end{subfigure}%
    \begin{subfigure}{.5\textwidth}
      \centering
      \includegraphics[scale=0.54]{./../results/placebolbw_gest.eps}
      \caption{Low birth weight placebo}
      \label{placebo-lbw}
    \end{subfigure}
  \end{center}
  \floatfoot{\textsc{Notes to figure \ref{placebo}}: Placebo tests
    consist of (falsely) assigning the initiation of the ChCC program
    during the pre-program period for the early and late adopters.  In
    each case the estimates and 95\% confidence intervals displayed
    correspond to the estimated effects of ChCC on early life outcomes
    if the early adoption municipalities adopted in the month displayed
    on the $x$-axis, and the late adopters adopted 8 months later (as
    was the case with the true adoption).  The actual adoption date was
    in June of 2007, and so the true estimates (ie estimates from table
    3) correspond to those indicated by the red vertical line.  The
    left-hand panel presents placebo tests for birth weight, while the
    right-hand panel presents tests for the proportion of low-birth weight
  babies.}
\end{figure}

\subsection{Robustness and Extensions}
In our principal specification we use weighted regressions, time and
municipal fixed effects, and infer exposure to the reform by subtracting
9 months from the date of birth.  We examine a number of specifications
to determine the robustness of these results to alternative measures and
specifications.

Firstly, as we observe the precise day of birth as well as the gestation
length in weeks, we can estimate the exact day of conception, and generate
exposure to Chile Crece Contigo as any pregnancy conceived after the
program's implementation.  In table 4 we replicate our main municipal-level
results from table 3, however now measuring exposure using conception rather
than birth dates.  In this table we observe that all results hold, and indeed
appear to be if anything slightly larger.  Our estimates for birth weight
are now 7.9 grams on average in each municipality, which when inflated to
give individual level estimates suggests an impact of 12.15 grams.

We present additional results in appendix tables documenting estimates
without municipal-specific population weighting, and where we augment
equation \ref{eqn:DD} to include municipality-specific linear time trends.
In both cases we find largely similar results, with comparable effect
sizes.

\begin{landscape}
\input{./../results/comunaDifDif-gest.tex}
\end{landscape}

Finally, throughout the paper we have estimated our main specifications
using four (related) measures of human capital at birth.  Despite the
fact that we are testing multiple hypothesis tests with a single independent
variable (Chile Crece Contigo exposure), we have not corrected for this
in baseline hypothesis testing.  As such, we examine the results' robustness
to \citet{RomanoWolf2005}'s stepdown algorithm which fixes the Family Wise
Error Rate.  Even when using this (demanding) criterion to test the
significance of results we observe that the estimated effects of ChCC
remain.  For example, in our main difference in difference results $p$-values
associated with each of the four outcome variables when using Romano Wolf
testing are 0.0434, 0.0876, 0.1833 and 0.1833 (in order of decreasing significance),
compared to 0.0225, 0.0351, 0.0978 and 0.1288 in naive tests.  All in all, these
results suggest that participation in the Chile Crece Contigo social security
program had economically and statistically important effects on neonatal
health outcomes.

\section{Efficiency of Public Healthcare Spending}
\label{sscn:spending}
Chile Crece Contigo is the flagship early life health program in the Chile,
and one of the largest social safety net programs of any type in the country.  It is also
one of the most important early life health programs in a middle or lower-middle
income country setting worldwide \citep{Richteretal2017}.  As such considerations
of efficiency in public health care spending are of considerable importance.
As we describe in table \ref{tab:spending}, spending on ChCC is approaching 1\%
of the fiscal budget per year, documenting the importance of this policy nation-wide.
Using the current exchange rate, spending on ChCC in 2010 was approximately USD 330
million.

\begin{table}[htpb!]
  \caption{Spending on ChCC as a Portion of National Spending}
  \label{tab:spending}
  \begin{tabular}{llcc} \toprule
    Year & Spending & Spending & Percent \\
    & (ChCC)        & (National) & \\ \midrule
    2007 & 67,903,331  & 17,883,154,418 & 0.380 \\
    2008 & 126,446,362 & 20,650,579,217 & 0.612 \\
    2009 & 159,660,473 & 23,406,879,324 & 0.689 \\    
    2010 & 214,505,550 & 25,651,969,793 & 0.836 \\
    \midrule
    \multicolumn{4}{p{7.1cm}}{{\footnotesize All values are displayed in 1000s of Chilean pesos. All national spending values are taken from the corresponding yearly budget, and ChCC spending
    is compiled from the ChCC final reports \citep{Arrietetal2013}.}} \\ \bottomrule
  \end{tabular}
\end{table}

To provide a broader consideration of the program's impacts and efficiency
given public investment, we calculate the inferred cost of producing one
gram of birth weight through this policy.  In order to do so we compare the
total cost of the pre-natal portion of Chile Crece Contigo with the total
grams of birth weight produced by the policy.  In order to estimate the
total costs, we assign pro-rata costs by years in the program.  Given that
the prenatal component is approximately 1 of 5 years of coverage, we assume
that one fifth of costs are apportioned to this sector.  In order to calculate
the total birth weight gained due to the program, we use our preferred estimate
of approximately 12 grams from table 3.  Using these values, as well as the
total number of pregnant women covered per year, the inferred cost of a
gram of birth weight is approximately 12,000 Chilean Pesos\footnote{This value
  is calculated using the costs, the total estimated impact, and the number
  of program recipients as:
  \[
  \text{Inferred Cost}=\frac{\frac{1}{5}\times 20,650,579,217,000}{13 grams\times161,834}=12020.52 pesos/gram
  \]} (or based on the
current exchange rate, approximately 18 US dollars.  Interestingly,
this value is similar in magnitude to that calculated from the US Food Stamp
Program and The WIC program \citep{Clarkeetal2017}.\footnote{In ongoing work
  we are collecting data to generate comparison values from early life health
  programs in  other Lower Middle Income Countries and Middle Income Countries.
  Examples of such programs from within Latin America include Plan Nacer (Argentina)
  and Qali Warma (Peru).}

While this value benchmarks the efficiency of the ChCC program compared to
other early life health programs, it provides less context on the implications
of these costs for social spending and development outcomes within the country.
In order to put these estimates in context, we can ask how investments in birth
weight compare to the \emph{returns} to birth weight in the country.  In Chile
there are a number of well-identified estimates of the value of birth weight to
later-life education, with significant and long-standing observed impacts
\citet{Bharadwajetal2013,Bharadwajetal2017}.  Using a similar within family
estimation strategy as proposed in specification \ref{eqn:panel} of this
paper, \citet{Bharadwajetal2017} estimate that a 10\% increase in weight at
birth increases child test scores by 0.05 standard deviations.  Using our
estimates from above, these values imply that each additional standard
deviation improvement on test scores costs 809,000 Chilean Pesos, or approximately
1200 USD.  What's more, these estimates are clearly a lower bound.  While birth
weight is a well known determiniant of educational attainment, birth weight is
also known to impact labour market outcomes \citep{JohnsonSchoeni2011,
  CookFletcher2015,BehrmanRosenzweig2004,RosenzweigZhang2013,Caseetal2005},
the prevalence of chronic morbidities \citep{Barker1995,AlmondMazumder2005,
  JohnsonSchoeni2011b}, mortality \citep{vandenBergetal2006}, and a range of
psychological outcomes \citep{Fletcher2011}.

%\begin{figure}[htpb!]
%\begin{center}
%  \centering
%  \caption{Placebo Tests for Municipal-Level Difference-in-Differences}
%  \includegraphics[scale=1]{./../results/placebopeso.eps}
%  \label{fig:placeboBW}
%\end{center}
%\vspace{-5mm}
%\floatfoot{\textsc{Notes to figure \ref{fig:placeboBW}}: }
%\end{figure}

\section{Conclusion}
We examine the importance of a large early life social safety net program
in a middle income country.  This program---Chile Crece Contigo---is one
of the largest social programs in Chile, reaching more than 150,000 pregnant
women each year, and accounting for nearly 1\% of the national budget.
Using newly generated administrative data matching all births with a program
participation indicator, as well as time and geographical variation in program
roll-out, we are able to combine a number of estimation strategies leading
to plausibly causal effects under varying assumptions.

We document, firstly, that this program has considerable effects on neonatal
health in Chile.  Depending on the specification examined, we estimate that
the program participation increases birth weight between 12 and 13 grams,
reduces the probability of being low birth weight by up to 10\% and reduces
premature births by as much as 7\%.  What's more, it appears to eliminate
the birth weight differential between the poorer program participants and
the less-poor non-participants.  Results appear to agree quite well whether
working with between-mother micro-level estimates, or difference-in-difference
estimates based on program roll-out nation-wide.

Combined with the cost of running Chile Crece Contigo, our estimates suggest 
that the government of Chile spends approximately \$18 per gram of birth
weight---a figure that is comparable to other large neonatal health programs,
even in developed countries.  What's more, given the well known positive
effects of birth weight on later life outcomes, we are able to estimate that
as an \emph{upper bound} cost, each \$1200 spent on Chile Crece Contigo
results in an additional 0.1 standard deviation of educational attainment
on later life test scores.  All told this paper suggests that public
investments in early life health in developing and emerging economies have
considerable returns when well targeted and well designed, and that these
impacts may propogate through the economy long after birth and program
implementation.

\end{spacing}
\newpage
\bibliography{references}

\setcounter{table}{1}
\renewcommand{\thetable}{A\arabic{table}}
\setcounter{figure}{1}
\renewcommand{\thetable}{A\arabic{figure}}
\clearpage
\appendix
\section*{Appendices}

\section{Appendix Figures and Tables}
\label{ascn:figstabs}
\begin{table}[htpb!]
  \caption{List of ChCC Policy Benefits}
  \label{ascn:benefits}
  \begin{tabular}{ll}
    \toprule
    Element (Program) & Benefit or Service \\ \midrule
    \multirow{4}{*}{Massive Education}
    &Weekly radio program ``Creciendo Juntos'' with national coverage\\
    &Chile Crece Contigo TV in waiting rooms of public health centres \\
    & ``Fono Infancia'' a free phone line providing support and information \\
    & Provision of children's books, and prenatal music CDs \\ \midrule
    \multirow{6}{*}{Socio-emotional Development}
    & Gestation and birth guide ``beginning to grow'' \\
    & Prenatal care protocol: check-ups  \\
    & Prenatal care support: fortified food, information\\
    & Educational support for pregnant mother and partner (4 sessions) \\ 
    & Personalised birth support, favouring rapid skin contact with mother \\ 
    & Integral puerperal and breastfeeding support \\ \midrule
    \multirow{6}{*}{Newborn Support Program}
    & Pack for safe attachment and clothing (multiple goods) \\
    & Pack for basic care and stimulation (mulitple goods) \\
    & Pack with crib/corral (multiple goods) \\
    & Integral care for newborn in neonatal and pediatric units \\
    & Regular health controls focused on integral development \\
    & Health controls for vulnerable or developmentally delayed children \\ \midrule
    \multirow{10}{*}{Means-Tested Elements}
    & Access to technical support for children with any type of disability \\
    & Guaranteed free access to ``sala cuna'' \\
    & Guaranteed free access to nursery school \\
    & Guaranteed access to ``Chile Solidario''\\
    & Support to finish education (mothers) \\
    & Support for labour market insertion (families) \\
    & Improvement of living conditions (families) \\
    & Mental Health Attention \\
    & Family dynamic attention (focused on domestic violence)\\
    & Judicial support \\  \bottomrule
    %\multicolumn{2}{p{15cm}}{{\footnotesize \textsc{Notes}}}
  \end{tabular}
\end{table}

\begin{landscape}
\input{./../results/comunaDifDif-nowt.tex}
\end{landscape}

\begin{landscape}
\input{./../results/comunaDifDif-trend.tex}
\end{landscape}

\section{Chile Crece Contigo}
\emph{Chile Crece Contigo}, or in English ``Chile Grows With You'', is a
large social safety net program focused exclusively on early childhood
development.



\section{Data Agreement}
\label{ascn:agreement}
We provide the full confidentiality agreement between the principal investigator
of this project (Clarke) and the Ministry of Social Development (previously known
as Ministry of Planification) and the Ministry of Health.  This agreement is
displayed in full overleaf.
\\

NOTE: This is suppressed for filesize limits on paper submission.
\includepdf[pages=-,pagecommand={},width=\textwidth]{confidencialidad.pdf}


\end{document}
%\newpage
%\begin{table}
%\caption{Requirements for Causality}
%\begin{tabular}{lcc}\toprule
%Instrument & Positive Bias & Negative Bias  \midrule
%Twins & $\Cov(Twin,U)>0$ & & $\Cov(Twin,U)<0$ & 
%\end{tabular}
%\end{table}
\begin{landscape}
\begin{figure}[htpb!]
\begin{center}
  \centering
  \caption{Contraceptive Reforms in the Economic Literature}
  \includegraphics[scale=0.77]{./figures/timeline3.png}
  \label{fig:timeline}
\end{center}
\vspace{-5mm}
\floatfoot{\textsc{Notes to figure \ref{fig:timeline}}: Dates of contraceptive reforms
are as listed in papers which use the reforms.  The definition of the first state
anti-abortion laws and their relaxation are used in \citet{Lahey2014}.  Affordable
Care Act (ACA) mandates are used in \citet{Mulligan2015}.}
\end{figure}
\end{landscape}


